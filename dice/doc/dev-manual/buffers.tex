\section{Memory and Buffer Management}
\label{sec:mem-mgmt}

There are different requirements on the buffer management for IDL derived
code.  The CORBA Specification, for instance, defines that any memory
necessary for non-simple types is allocated by stubs automatically.  When
conforming to this requirement, a working implementation of \verb|malloc|
and \verb|free| has to be provided by the user.

Another part, which defines requirements, is the communication platform.
L4 allows the usage of indirect parts in messages, which are only the
start address and size of a memory region to be transferred.  This memory
region is not part of the message buffer.  Therefore the receiver has 
to provide a corresponding memory region into which the content of the
sender's memory region can be copied.  This feature is widely used in
L4 applications to avoid copying large blocks of data from a message buffer
into other memory structures.  The start of an area in the memory 
structure is given and the message transfer copies data directly into 
that structure.


