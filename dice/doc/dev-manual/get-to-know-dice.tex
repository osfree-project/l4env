\section{Get to know \dice}

When you start working on \dice{}, first get to know it's structure and source
code.  \dice{} source code makes intensive use of inheritance to implement
features specific to a certain target language or platform.  You have to
always be aware of this fact.

\emph{TODO: class names}

\emph{TODO: class inheritance}

\emph{TODO: the basic structure}

\emph{TODO: in depth look: front-end}

\emph{TODO: in depth look: parser}

\emph{TODO: in depth look: back-end}

Walk the source code! Let's say you want to alter the way a class declaration
is written, the most natural place to first look at is the implementation of
the \texttt{Write()} method.  There you see the---what I thought---logical
step when writing a class.  Look for the method writing the declaration of the
class. There might be different \texttt{Write()} methods for header and
implementation files, where the declaration is written only for the header
file.  Or, there might be a seperate method to be called if there is a
declaration to be written. You cannot find code that would produce the output
you see when a class declaration is written? Then the respective method might
be overloaded. Search for a back-end class, which derives from
\texttt{CBEClass}, for instance \texttt{CLangCClass}. Maybe the respective
code can be found there.

When adding code, which should go back into the \dice{} source tree, chack if
this code is platform dependent, language dependent, neither, or both.  Create
classes respective to your decision.  The higher the created classes are in
the inheritance tree, the more independent they are.  If necessary, create new
subdirectories and create your classes there (do not forget to integrate them
into the make process).

