\chapter{Using the Generated Code}
To illustrate the usage of the generate code, the following example
is used. To look up the syntax of the example consult the
previous section. This example IDL is contained in a
file named \verb|test.idl|.

\begin{verbatim}
library example
{
  interface first
  {
    void foo([in] int parameter);
  };
};
\end{verbatim}

\section{Invoking \dice{}}
\dice{} generates per default code for both client and server side. To
translate the above example into target code for the C language and the
L4.Fiasco ABI on a \verb|x86| architecture, call:

\begin{verbatim}
dice test.idl
\end{verbatim}

The target language C, as well as the L4.Fiasco ABI and the \verb|x86|
architecture are default setting for the respective options. The above call
will generate five files.

\begin{enumerate}
\item \verb|test-client.h|
\item \verb|test-client.c|
\item \verb|test-server.h|
\item \verb|test-server.c|
\item \verb|test-sys.h|
\end{enumerate}

The file \verb|test-sys.h| contains the opcodes for the interface as described
in Section~\ref{sec:opcode}. The client and server side code is contained in
the respective files.

\section{Client Code}
The client side code packs the \verb|in| data into a message buffer that is
send via IPC to the server.  The declaration of the client function looks like
this:

\begin{verbatim}
void example_first_foo_call(CORBA_Object _dice_corba_obj,
     int parameter,
     CORBA_Environment *_dice_corba_env);
\end{verbatim}

There are several things noteworthy. Two additional parameters have been added
to the function. According to the CORBA C Language Mapping \cite{corba-clm}.
The first identifies the server, which is on L4 a pointer to the thread ID
(\verb|l4_threadid_t|) of the server.  {\em \verb|CORBA_Object| is a type
alias for a pointer to an L4 thread ID.} The last parameter is the CORBA
environment, which is a collection of information denoting the context of the
call. For more detailed information about the \verb|CORBA_Environment| see
Section~\ref{sec:environment}.

If using C as target language, the name of the function is composed of the
name of the library or module (\verb|example|), the name of the interface
(\verb|first|), and the name of the function (\verb|foo|).

\section{Server Code}
The function template generated for the server looks similar to the client
side code. The only difference is the function's name. The
parameter list is almost the same as the client side function.

\begin{verbatim}
void example_first_foo_component(CORBA_Object _dice_corba_obj,
     int parameter,
     CORBA_Server_Environment *_dice_corba_env);
\end{verbatim}

At the server side the \verb|CORBA_Object| parameter does not contain the ID
of the server, but instead the ID of the sender. It can be used to identify
the sender of a request.

By default the server function is only declared in the server header file, but
no implementation exists. The option \verb|-t| or \verb|--template| will add
an additional file which ends on \verb|-template.c| to the list of generated
files. It contains function skeletons for the server functions. Copy these
skeletons and add the functionality of the server. Note that every new
invocation of \dice{} with the mentioned option will regenerate the template
file.

\section{Server Loop}
The server's side of the communication implements
functionality, which receives messages, dispatches them to the
appropriate server side functions, and sends replies
to the client. This process can be separated into the
following steps:

\begin{enumerate}
\item wait for a new message
\item get opcode from message
\item check opcode, and depending on it
\item unmarshal the \verb|in| parameters
\item call the server side function
\item marshal the \verb|out| parameters
\item send the reply
\item start from the beginning (step 1)
\end{enumerate}

For performance reasons step 7 and 1 are usually combined into one step.
Accordingly to the above steps, \dice{} generates helper functions which have
the following endings:

\begin{enumerate}
\item[1.] and 2. \verb|<interface>_wait_any|
\item[3.] is a switch statement inside of \verb|<interface>_dispatch|
\item[4.] \verb|<function>_unmarshal|
\item[5.] \verb|<function>_component|
\item[6.] \verb|<function>_marshal|
\item[7.] and 8. \verb|<interface>_reply_and_wait|
\end{enumerate}

The functions \verb|<interfac>_wait_any|, \verb|<interface>_dispatch|,
and \verb|<interface>_reply_and_wait| are called inside of the
\verb|<interface>_server_loop| function. The \verb|<function>_*|
functions are called in the \verb|<interface>_dispatch| function.

The place holders \verb|<interface>| and \verb|<function>| show that
the name is either interface of function specific.
The aforementioned example produces
an \verb|<interface>|-specific function with the name
\verb|example_first_wait_any| and a \verb|<function>|-specific function
with the name \verb|example_first_foo_unmarshal|.

%TODO:
%The aforementioned functions, except the \verb|<function>_component| and
%\verb|<interface>_server_loop| function are declared \verb|static inline|
%and are only defined in the server implementation file.

The \verb|<interface>_server_loop| function does take a \verb|void *|
parameter. It does not return any value.

For an interface which is derived from other interfaces, the
server loop will also be able to distinguish messages for functions
of the base interface. It dispatches these messages to the appropriate
server functions.

\section{Function Identifiers --- Opcodes}
\label{sec:opcode}
To distinguish the messages from one another, e.g. to know which function
should be called, the message contains at a defined position in the message a
number, consisting of an interface identifier and a function identifier ---
the {\em opcode}. This opcode is a number, which is determined by the sequence
of the declaration of the functions inside the interface definition. For
example does \dice{} generate for the following IDL:

\begin{verbatim}
interface simple
{
    void func1();
    void func2();
    void func3();
};
\end{verbatim}

the function identifiers:

\begin{enumerate}
\item \verb|func1|: 1,
\item \verb|func2|: 2, and
\item \verb|func3|: 3.
\end{enumerate}

If you derive an interface from \verb|simple|, the function identifiers
would have to continue starting with 4. To avoid this, the interface identifier
is used. For example do the functions of the following interface

\begin{verbatim}
interface derived : simple
{
    void func4();
    void func5();
};
\end{verbatim}

map to the following function identifiers:

\begin{enumerate}
\item \verb|func4|: 1 and
\item \verb|func5|: 2.
\end{enumerate}

The functions \verb|func1| and \verb|func4| are only different in the
interface identifier, which is 1 for \verb|simple| and 2 for
\verb|derived|. The resulting opcode is generated by shifting the
interface identifier left by \verb|DICE_IID_BITS|\footnote{The value
is currently set to 20 and is defined in one of the dice header files.}
and then AND the function identifier bitwise. Thus the opcode for
\verb|func1| is \verb|0x100001| and for \verb|func4| is \verb|0x200001|.

\subsection{Defining Opcodes}
\label{sec:uuid}
To determine the interface and function identifiers yourself,
you may use the \verb|uuid|\footnote{{\tt uuid} stands for Universal
Unique Identifier and is supposed to contain a 128 bit number. The DCE
Specification allows the {\tt uuid} attribute only for interfaces. We
(mis)use it for interface and function identifiers.} attribute for the
interface or functions. If used with the interface, for example:

\begin{verbatim}
[uuid(0xC00)]
interface other
{
...
\end{verbatim}

it will only effect the interface identifier. Valid values for interface
identifiers range from 1 to $0xFFF$. If you specify an interface
identifier, which is already used by a base interface, the function
identifiers of the derived interface will be counted starting
after the biggest function identifier of the base interface. For example does:

\begin{verbatim}
[uuid(1)]
interface derived : simple
{
...
\end{verbatim}

generate the function identifiers:

\begin{enumerate}
\item \verb|func4|: 4 and
\item \verb|func5|: 5.
\end{enumerate}

You may use the \verb|uuid| attribute with functions as
well. Valid values are within the range of $0$ to $0xF'FFFF$.
Function identifiers are first assigned
using the \verb|uuid| attribute. Then the remaining functions are
numbered starting from the lowest, not-assigned number. For example:

\begin{verbatim}
interface simple
{
    void func1();
    void func2();
    [uuid(1)] func3();
};
\end{verbatim}

will generate:

\begin{enumerate}
\item \verb|func1|: 2,
\item \verb|func2|: 3, and
\item \verb|func3|: 1.
\end{enumerate}

If a derived interface has the same identifier as the base interface
it will start to enumerate its own functions with the highest number
of the base interfaces. The example:

\begin{verbatim}
interface simple
{
  void func1();
  [uuid(4)] void func2();
  [uuid(1)] void func3();
};

[uuid(1)]
interface derived : simple
{
  void func4();
  void func5();
};
\end{verbatim}

will generate the following identifiers:

\begin{enumerate}
\item \verb|func1|: 2,
\item \verb|func2|: 4,
\item \verb|func3|: 1,
\item \verb|func4|: 5, and
\item \verb|func5|: 6.
\end{enumerate}

This implies that any changes to the IDL of a base interface imply
regeneration of code generated for derived interfaces.

It is possible to use constants or enumeration instead of numbers with the
\verb|uuid| attribute.  If you derive an new interface from multiple base
interfaces use the \verb|uuid| to distinguish the different base interfaces.
This is necessary because base interfaces can be compiled separately.

\section{CORBA Environment}
\label{sec:environment}

The CORBA Environment is split into two types.  One environment for the client
side and one for the server side.  This is done because client wrapper
functions often declare the environment on the stack.  To keep the memory
footprint small, the client side environment is a reduced version of the
server side environment.

The respective definitions can be found in Appendix~\ref{app:corba_env}.

The first elements of the \verb|CORBA_Environment| are defined
by the CORBA C language mapping. The first is of type
\verb|CORBA_exception_type| (an \verb|int|), which consists of
a \verb|major| exception number, and a \verb|repos_id|. The major
number defines whether the exception was a system exception (e.g.
communication error) or a user defined exception. The second
member defines a minor exception number.

The \verb|repos_id| can also be used as an index into a
repository with descriptions of the exceptions. It can be
used to print precise error messages. The exception description
can be accessed using the method \verb|CORBA_exception_id|,
which takes the \verb|repos_id| as an argument.

The third parameter is also defined by the CORBA C language
mapping. It is a void pointer to user defined data. User defined
means here, that it is defined by the exception setting instance.
An exception can be set using the method \verb|CORBA_exception_set|.
Functions to manipulate the CORBA Environment are explained
in the following sections.

This parameter is used disjunct with the IPC error code.  If there
was an IPC error, no exception could be transmitted.

\subsection{CORBA Environment Functions}

There are a few functions defined by the CORBA C Language
mapping \cite{corba-clm} to manipulate the CORBA Environment.

\begin{verbatim}
void CORBA_exception_set(
  CORBA_Environment *ev,
  CORBA_exception_type major,
  CORBA_char *except_repos_id,
  void *param);
\end{verbatim}

The above function is used to set the exception part
of the environment. It receives a pointer to an
environment structure to be initialized. It first calls
\verb|CORBA_exception_free| for the environment. Then
it sets the \verb|major| member of the environment and
if it is not equal to \verb|CORBA_NO_EXCEPTION| the
\verb|repos_id| and \verb|param| members are set as well.

\begin{verbatim}
CORBA_char* CORBA_exception_id(CORBA_Environment *ev);
\end{verbatim}

This function retrieves the stored exception description
for the \verb|repos_id| of the environment. The exception
descriptions are stored in the global variable
\verb|__CORBA_Exception_Repository| and are currently set
as described in table~\ref{tab:reposid}.

\begin{table}[ht]
\caption{\label{tab:reposid} Stored exception descriptions.}
\begin{tabular}{|l|l|} \hline
\verb|repos_id| & stored string \\ \hline \hline
\verb|CORBA_DICE_EXCEPTION_NONE| & {\tt none} \\ \hline
\verb|CORBA_DICE_EXCEPTION_WRONG_OPCODE| & {\tt wrong opcode} \\ \hline
\verb|CORBA_DICE_EXCEPTION_IPC_ERROR| & {\tt ipc error} \\ \hline
\verb|CORBA_DICE_INTERNAL_IPC_ERROR| & {\tt internal ipc error} \\ \hline
\end{tabular}
\end{table}

\begin{verbatim}
void* CORBA_exception_value(CORBA_Environment *ev);
\end{verbatim}

This function returns the value of the \verb|param| member of
the environment.

\begin{verbatim}
void CORBA_exception_free(CORBA_Environment *ev);
\end{verbatim}

This functions is supposed to free any allocated memory of
the resource. Since \verb|CORBA_exception_set| does not
allocate any memory, the members are simply set to the default
values: \verb|major| is set to \verb|CORBA_NO_EXCEPTION|,
\verb|repos_id| is set to \verb|CORBA_DICE_EXCEPTION_NONE|,
and \verb|param| is set to zero.

\subsection{\dice{} Specific Extensions}
\label{sec:env-dice}

\dice{} currently adds some members to the environment.
They are \verb|timeout|, \verb|rcv_fpage|, \verb|ipc_error|,
\verb|user_data|, \verb|malloc|, \verb|free| and memory
pointers \verb|ptrs|.

The \verb|timeout| value is used to set the timeout
of an IPC. It is of type \verb|l4_timeout_t|.

The \verb|rcv_fpage| value is used to set the receive
window for received flexpages. It is of type
\verb|l4_fpage_t|.

The \verb|ipc_error| value is set if an IPC error
occurred to the value of \verb|L4_IPC_ERROR(result)|,
where \verb|result| is the result variable of the
IPC\footnote{An L4 IPC returns a result variable,
which contains information about the transferred number
of double words and indirect parts. If there occurred an
error during the IPC, the result variable contains
further information about the kind of the error.}.
If there was no IPC error the value is undefined.
To determine if there was an IPC error test the
\verb|major| member of the environment for
\verb|CORBA_SYSTEM_EXCEPTION|.

The \verb|user_data| member is of type \verb|void *| and
can be used freely by the user. It is ignored by the generated
stubs and not transferred from the client to the server,
but consistent within the scope of the server loop.

The \verb|malloc| and \verb|free| members are a function pointers
to memory allocation reoutines of type
\begin{verbatim}
void* (*dice_malloc_func) (unsigned long);
void (*dice_free_func) (void*);
\end{verbatim}
which are used to allocate and free memory. These function pointers
are currently used by default.

When using an \verb|CORBA_Environment| you should always
initialize it with \verb|dice_default_environment|. This
ensures, that all members are set to default values.
Regard that the memory function pointers are set to
default functions, which enter the kernel debugger.
The default environment of \verb|CORBA_Server_Environment| is
\verb|dice_default_server_environemnt|.

If you assign the memory function members of the environment new functions,
some compilers complain about mismatching types.  Simply cast your function
using \verb|dice_malloc_func| and \verb|dice_free_func| respectively.

\section{Buffer Management}

\section{Memory Management for Indirect Parts}
\label{sec:mem-indirect-parts}
Since the server loop is auto generated, it has to use heuristics to provide
the receive buffer for indirect parts. The default approach is to dynamically
allocate the memory using the \verb|malloc| member of the
\verb|CORBA_Environment| of the server loop.  If the
\verb|-fforce-corba-alloc| option is specified the \verb|CORBA_alloc| function
is used (see Section~\ref{sec:force-corba-alloc}).

\subsection{{\tt malloc} and {\tt CORBA's alloc}}

The \verb|malloc| member of \verb|CORBA_Environment| and the
\verb|CORBA_alloc| functions have the same signature as the libc \verb|malloc|
function. They take the desired size of the memory as parameter and return a
\verb|void *| to the allocated memory.

Example for \verb|malloc| member of \verb|CORBA_Environment|:
\begin{verbatim}
CORBA_Environment env = dice_default_environment;
env.malloc = (dice_malloc_func)malloc;
env.free = (dice_free_func)free;
server_loop(&env);
\end{verbatim}

Example for \verb|CORBA_alloc|:
\begin{verbatim}
void* CORBA_alloc(unsigned long size)
{
  return malloc(size);
}
\end{verbatim}

The server loop will call the \verb|malloc| or \verb|CORBA_alloc| function for
every receive buffer it should set. It defines the size of the buffer by the
maximum of all indirect parts that can be received into this buffer.  To
determine the size of an indirect part, \dice{} uses the maximum values
defined either in brackets or using the \verb|max_is| attribute. If neither is
given, a heuristic is used to determine the maximum size: all variable sized
strings are ``defined'' to have a maximum size of $512$ bytes. All other
variable sized arrays are defined to have a maximum size of $1024$ bytes. If
you want to send data larger than these sizes use the \verb|max_is| attribute
to give \dice{} a hint.

\subsection{\tt init rcvstring}
Example for \verb|init_rcvstring|: first the IDL file, then the
needed additional code:
\begin{verbatim}
[init_rcvstring(my_alloc)]
interface simple
{
  void foo([in, ref] long data[200]);
};
\end{verbatim}

\dice{} generates a server loop, which initializes each receive
buffer for indirect parts by calling the named (\verb|my_alloc|)
function. A possible implementation for this function might look
like this:

\begin{verbatim}
void
my_alloc(int nb,
  l4_umword_t *addr,
  l4_umword_t *size,
  CORBA_Environment *env)
{
  *addr = my_buffers[nb];
  *size = BUFFER_SIZE;
}
\end{verbatim}

It receives four parameters:
\begin{enumerate}
\item a zero based index indicating which of the strings should
be set (\verb|nb|),
\item a pointer to the variable which should hold the address of the
buffer (\verb|addr|),
\item a pointer to a variable which should hold the size of the
buffer (\verb|size|), and
\item a pointer to the current \verb|CORBA_Environment|.
\end{enumerate}

% XXX Todo
%If you want to implement a simple protocol, where the client first calls a
%function to specify the size of the next message with an indirect part
%parameter, you can call the specified functions yourself, using the size
%send by the client.

