\chapter{Pagefault Handling}
\label{app:pagefault-handling}

L4 servers imply special semantics with certain {\em opcodes}. One is the pagefault
IPC. \dice{} provides some mechanisms to deal with these special semantics.

\section{L4 version 2}
A pagefault IPC consists of two double word values. The first contains the
pagefault address and the second the current instruction pointer. Since by 
convention there are no pagefaults in the micro-kernel's portion of the 
address space, any pagefault address above \verb|0xC000'0000| is not possible,
or handled by the kernel directly (will not involve an IPC). \dice{} generated 
code uses the first double word for its opcode.

This knowledge can be used to write a server, which is able to handle 
pagefault IPC. As described in section~\ref{sec:uuid}, you may specify that
the interface identifier and thus the base opcode is \verb|0xC00|, which
generates opcodes larger than \verb|0xC000'0000|.

The server loop now ``accepts'' only opcodes larger than \verb|0xC000'0000|.
To handle the pagefaults, one has to specify a call-back function, which is
invoked for every ``unknown'' opcode --- which are all opcodes smaller
than \verb|0xC000'0000|. This is done using the \verb|default_function|
attribute for interfaces:

\begin{verbatim}
[uuid(0xC00), default_function(pf_handler)]
interface pagefault
{
...
\end{verbatim}

The call-back function has to have the format:

\begin{verbatim}
CORBA_int pf_handler(CORBA_Object src,
    <interface>_msg_buffer_t* msg_buffer,
    CORBA_Environment *env)
\end{verbatim}

This is basically the format of a normal \verb|<function>_component|
function, but instead of parameters it receives the message buffer.
The implementation can access the members of the message buffer using
the macros described in appendix~\ref{app:msgbuf_macros}. An example
implementation of the pagefault handler is provided in 
appendix~\ref{app:pf_handler}.

The default function has a return value. This is used to determine
whether the server loop should respond to the message or not. If you 
return \verb|DICE_REPLY| the server loop will send a reply message.
To skip the reply message, the default function returns \verb|DICE_NO_REPLY|.

