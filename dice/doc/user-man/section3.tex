\chapter{A Short Reference to Compiler Options}

This chapter will provide you with a short overview about the available
compiler options. It will also show you what you can do with these 
options.

Most of the information given here can also be found in the manual
page available for \dice{}.

\section{Compiler Options}
There are several compiler options which are roughly grouped
into general compiler options, preprocessor options, and back-end
options.

\subsection{Pre-Processing Options (Front-End Options)}
These options affect the preprocessing of the input files.

\subsubsection{{\tt -x <language>}}
Specify the language of the input IDL files. Possible values are {\tt dce} and
{\tt corba}. This option defaults to {\tt dce}. Include header files are
recognized by their extension.

\subsubsection{{\tt --preprocess, -P}}
Passes the arguments given to this option to the preprocessor.

Example: \verb|-P--nostdinc|

\subsubsection{{\tt -Wp,}}
Same as {\tt -P}. {\tt -P} and {\tt -Wp,} are scanned for
{\tt -I} and {\tt -D}.

\subsubsection{{\tt -I}}
Provides the preprocessor with include directories to search
for the files specified with the \verb|#include| and \verb|import|
directives. (Same as {\tt -P-I}.)

Example: \verb|-I/usr/include|

\subsubsection{{\tt -nostdinc}}
Passed as is to the preprocessor.

\subsubsection{{\tt -D}}
Provides the preprocessor with additional symbols. (Same as
{\tt -P-D}.)

Example: \verb|-DL4API_l4v2|

\subsubsection{{\tt -E}}
Stop ``compilation'' after preprocessing. The output of the 
preprocessor is printed to stdout. Consider that the preprocessor
does not resolve \verb|import| statements and therefore these
statements will appear in the generated output.

\subsubsection{{\tt -EXML}}
Stop ``compilation'' after the parsing. The intermediate representation
(the abstract syntax tree -- if you like) is printed into an XML file. 
The name of the XML file is the same as the IDL file's with the \verb|idl|
extension exchanged against the \verb|xml| extension.

\subsubsection{{\tt -M}}
Print include file tree and stop after doing that.

\subsubsection{{\tt -MM}}
Print include file tree for files included with '' and stop
after doing that.

\subsubsection{{\tt -MD}}
Print include file tree into \verb|.d| files and compile.

\subsubsection{{\tt -MMD}}
Print include file tree for file included with '' and compile.

\subsubsection{{\tt -MF <filename>}}
Generates the dependency tree into the file specified by the filename. This
option only works if one of {\tt -M}, {\tt -MM}, {\tt -MD}, or {\tt -MMD} is
given.

\subsubsection{{\tt -MP}}
Generates for all files, which the generated files depend on, a phony
dependency target in the dependency list. This option also requires one of the
dependency generation options.

\subsubsection{{\tt --with-cpp=<argument>}}
Specify your own preprocessor. This will override environment
variable \verb|CC| or \verb|CXX|.

Example: \verb|--with-cpp=/usr/bin/cpp-3.0|

\subsection{Back-End Options}

\subsubsection{{\tt --client, -c}}
Create client side code only. Default is to create both client and server side code.

\subsubsection{{\tt --server, -s}}
Create server side code only. Default is to create both client and server side code.

\subsubsection{{\tt --template, -t}}
Create server skeleton/template file.

\subsubsection{{\tt --no-opcodes, -n}}
Do not generate the opcode file. This is useful if you wish to specify the opcodes
yourself or with the parameter {\tt -fopcodesize}.

\subsubsection{{\tt --filename-prefix, -F}}
Prefix each file-name of the \dice{} generated files with the given
string.

Example: \verb|-FRun1| lets the includes of the generated files
look like this: \verb|#include "Run1<file>-client.h"|.

\subsubsection{{\tt --include-prefix, -p}}
Prefix each file-name inside the generated include statements
with the given string, which is interpreted as path.

Example: \verb|-p/tmp/dice| lets all includes be prefixed with
the string: \verb|#include "/tmp/dice/<file>-client.h"|.

Together with \verb|-F| the generated include statement looks like this
\verb|#include "/tmp/dice/Run1<file>-client.h"|.

\subsubsection{{\tt -o}}
Specify output directory. All generated files are placed into the 
given directory. If an invalid directory is given you receive error
messages stating that files cannot be opened.

\subsubsection{{\tt --create-inline=<mode>, -i<mode>}}
Generate client stubs as inline. You may specify a mode for
inline. Use \verb|static| or \verb|extern|. The mode is optional.
This will only generate header files, since the implementations
will appear there.

Example: \verb|-istatic| generates \verb|static inline| functions.

\subsubsection{{\tt -B<argument>}}

Defines the back-end which is used to generate the target code. There are
three categories, which define a back end. The first is the {\it target
platform}, which is denoted by the {\tt p} suffix to the {\tt -B} option.  The
second is the {\it target kernel interface}, denoted by {\tt i}. And last is
the {\it target language mapping}, denoted by {\tt m}.

The target platform can be one of {\tt ia32} {\tt arm}, or {\tt amd64}.  If
none of the mentioned platforms is chosen, {\tt ia32} is used.

The {\tt arm} platform is currently only supported with the X0 kernel
interface.

The {\tt amd64} platform is currently only supported with the V2 kernel
interface.

Example: \verb|-Bpia32|

The target kernel interface can be one of {\tt v2}, {\tt x0}, {\tt x0adapt},
{\tt v4}, {\tt sock}, or {\tt cdr}. Currently {\tt v4} is not fully supported.
The kernel interface {\tt x0adapt} uses the X0 kernel interface, but presents
a V2 interface to the user. The difference between the kernel interfaces is in
the size of the \verb|l4_threadid_t| type. Using {\tt x0adapt} you can
continue to use your applications written for use with a V2
\verb|l4_threadid_t| type on top of an X0 kernel.  The {\tt x0} kernel
interface is badly tested, because there currently exists no L4Env for native
X0.

Example: \verb|-Bix0adapt|

The target language mapping can be one of {\tt C} or {\tt CPP} -- for C++.
Using the C++ back-end will generate code that can be compiled with g++. It
does not contain classes for interfaces yet.

Example: \verb|-BmC|

\subsubsection{{\tt -O<level>}}
This option sets the optimization level of the back-end.
\emph{deprecated! do not use!}

\subsubsection{{\tt --testsuite, -T}}
Generates a test-suite for the declared interfaces. The test-suite
will start an L4 thread with the server, and call all functions
with random values for the parameters. The \verb|in| parameter
values are checked at the server for conformance with the values
set at the client side. The \verb|out| parameter values are set
in the server implementation function and checked at the client's
side.

This option will generate the \verb|-template.c| file with 
implementations of the server function which test the values.
It will also generate a \verb|-testsuite.c| file which initializes
the parameters and checks \verb|out| parameters. It also contains
code to start the server thread and initialize the server loop.

If you wish to write your own test-suite, don't use this
option, because this option makes \dice{} generate target
code which is complete and won't use your test functions.

%TODO: thread and task argument

\subsubsection{{\tt --message-passing, -m}}
Generate message passing functions for RPC functions as well.
See Section~\ref{sec:message-passing} for more details.

\subsection{General Options}
This section describes some of the options, which do not fit into one 
of the above categories.

\subsubsection{{\tt --help}}
Displays a verbose help screen, showing all of the available
options.

\subsubsection{{\tt --version}}
Displays version information, including the build date and the
user who built this version.

\subsubsection{{\tt --verbose, -v<level>}}
Displays verbose output.  The optional value specifies the amount of
verboseness. The higher values the more output.

\subsubsection{{\tt -f}}
Specifies additional compiler flags, which are hints for the compiler
on how to generate code. See section~\ref{sec:comp-flags} for details.

\subsection{Compiler Flags}
\label{sec:comp-flags}
The mentioned flags are specified as argument to the {\tt -f} option.
So the argument {\tt ctypes} is used as {\tt -fctypes}.

\subsubsection{{\tt F} - Filetype}
This is a ``nested'' option, which itself takes arguments. These arguments
are described in table~\ref{tab:filetype}.

\begin{table}[htb]
\begin{center}
\begin{tabular}{|l|l|p{6cm}|} \hline
Argument & Alternative & Meaning \\ \hline \hline
idlfile & 1 & Generate one client implementation file per input IDL file. \\ \hline
module & 2 & Generate one client implementation file per specified module. \\ \hline
interface & 3 & Generate one client implementation file per specified interface. \\ \hline
function & 4 & Generate one client implementation file per specified function. \\ \hline
all & 5 & Generate one client implementation file for all IDL files. \\ \hline
\end{tabular}
\caption{\label{tab:filetype} Filetype Options}
\end{center}
\end{table}

\subsubsection{{\tt ctypes}}
This option specifies that the generated code should use C types rather than
CORBA's C types. \verb|long| is used instead of \verb|CORBA_long|.
CORBA types, which have no C expression, are used as CORBA types.

{\it This option is respected in the Name Factory when generating type names.}

\subsubsection{{\tt l4types}}
This option specifies that the generated code should use L4 types\footnote{L4 Types
are type aliases for commonly used C types, which might have different size across
platforms. The long type is sometimes 16, 32, or 64 bits wide. These types are defined
in the {\tt l4/sys/l4int.h} header.} rather than
CORBA C types. \verb|l4_int32_t| is used instead of \verb|CORBA_long|.

This implies that the generated code is used within an L4 environment, which
knows these types. (L4Env does.)

{\it This option is respected in the Name Factory when generating type names.}

\subsubsection{{\tt opcodesize$=<$size$>$}}
This is a nested option. It can be used to determine the
size used by the opcode within the message. Its possible values are shown in
table~\ref{tab:opcodesize}.

\begin{table}[htb]
\begin{center}
\begin{tabular}{|l|l|l|} \hline
Argument & Alternative & Meaning \\ \hline \hline
byte & 1 & uses only 1 byte for the opcode \\ \hline
short & 2 & uses 2 bytes for the opcode \\ \hline
long & 4 & uses 4 bytes for the opcode (default) \\ \hline
longlong & 8 & uses 8 bytes for the opcode \\ \hline
\end{tabular}
\caption{\label{tab:opcodesize} Opcode Size Options}
\end{center}
\end{table}

\dice{} generated opcodes assume the opcode size of 4 bytes. If you specify
other sizes than this one, you should also use the {\tt --no-opcode} option,
so you can specify appropriate opcodes.

\subsubsection{{\tt server-parameter}}
If you specify the \verb|-fserver-parameter| switch the server
loop interprets the \verb|void *| parameter as  a pointer to a
\verb|CORBA_Environment| variable. It then uses the values of the 
environment to specify a
receive window for flexpages, timeouts, etc. 
Without this option the parameter is ignored.

\subsubsection{{\tt no-server-loop}}
If specified no server loop will be generated. Instead only the
dispatch function will be generated.

\subsubsection{{\tt init-rcvstring$=<$function-name$>$}}
Specifies a function to be used to initialize the receive buffers of indirect
strings. This is the same as the \verb|init-rcvstring| attribute you may specify
with an interface. If you specify this option, the function is applies to all 
generated server loops, which have no \verb|init_rcvstring| attribute.

\subsubsection{{\tt force-corba-alloc}}
\label{sec:force-corba-alloc}

\dice{} uses by default the function \verb|malloc| of
the \verb|CORBA_Environment| to dynamically allocate 
the memory for variable-sized receive parameters. To enforce
the usage of the \verb|CORBA_alloc|
function, specify the option {\tt -fforce-corba-alloc}. The usage of
\verb|CORBA_alloc| implies that there
have to be potentially two implementations of \verb|CORBA_alloc| --- one
for the client side and one for the server side. Also consider that
an implementation of \verb|CORBA_alloc| in a client library may collide with
a different implementation in another client library.
If you use the \verb|malloc| member of the \verb|CORBA_Environment|,
you can assign your implementation of malloc as you wish, e.g.
\verb|liba_malloc|.

You may detect the usage of malloc using the
\verb|-Wprealloc| option with \dice{}.


\subsubsection{{\tt force-c-bindings}}
This option enforces the usage of the L4 API C bindings. Otherwise \dice{}
may decide to generate inline assembler code for IPCs.

\subsubsection{{\tt trace-server=<function>}}
If specified the generated server code contains tracing code, which prints
status information to the LOG server.

If specified the $<$function$>$ is used to print the trace messages. The function
has to have the same signature as the printf function.

\subsubsection{{\tt trace-client=<function>}}
If specified the generated client code contains tracing code, which prints
status information to the LOG server.

If specified the $<$function$>$ is used to print the trace messages. The function
has to have the same signature as the printf function.

\subsubsection{{\tt trace-dump-msgbuf=<function>}}
This option makes \dice{} generate code which dumps the content of the message
buffer just before and after each IPC. {\it This may produce an immense amount
of status output.} 

If specified the $<$function$>$ is used to print the trace messages. The function
has to have the same signature as the printf function.

\subsubsection{{\tt trace-dump-msgbuf-dwords$=<$number$>$}}
This option restricts the number of dumped dwords to {\tt number}.

\subsubsection{{\tt trace-function$=<$function$>$}}
Each of the {\tt trace$-*$} options may be followed by {\tt $=<$function$>$}
to specify an output function for this class of traces. The specified function
has to follow the \verb|printf| syntax, which means it has to take as first
argument an format string and then a variable number of arguments.

To specify one function for all of the options, use {\tt trace-function}.
The default function is \verb|printf|.

\subsubsection{{\tt zero-msgbuf}}
This option lets \dice{} generate server code which zeros and then
re-initializes the message buffer, just before marshaling the return parameter 
in the \verb|wait| or 
\verb|reply-and-wait| functions. This provides
the marshaling code with a clean message buffer.

\subsubsection{{\tt use-symbols} or {\tt use-defines}}
\dice{} generates IPC code which contains a lot if \verb|#ifdef|
directives for profiling, frame-buffer use, etc. Since these \verb|#if|
statements mostly only test for the existence of a symbol, \dice{}
does a pre-evaluation of the symbols given via the {\tt -D} option,
and generates code only for the specified options.

This has the advantage that the generated code is smaller, because it
contains only one code snippet with IPC code (for each function) instead
of three or four.

This option should only be used if you know that \dice{} is invoked with
the same set of symbols used to compile the generated code.

\subsubsection{{\tt test-no-success-message}}
The test-suite generally prints a message if a parameter is transmitted 
correctly and it prints a message if data is not transmitted correctly.
With this option only error messages are printed.

\subsubsection{\tt const-as-define}
Constants are usually printed in the generated header files as const
declarations (\verb|const <type> <var> = <value>;|). Using this option,
they are written as define statements (\verb|#define <var> <value>|).

\subsubsection{\tt no-l4dir-include-path}
To know the CORBA Types, \dice{} imports the file \verb|dice/dice-corba-types.h|
before parsing the IDL file. The file can (usually) be found in
the L4 trees include dir. Therefore the L4 include dir has to be
known to include this file. This directory is automatically added to
the include search paths. If you do not wish to set this path, then
use the {\tt no-l4dir-include-path} option. The L4 tree's root is
determined by the configure switch \verb|--with-l4dir| which defaults
to \verb|../..|.

\subsubsection{\tt align-to-type}
On some architectures (e.g. ARM) it is necessary to align parameters 
when marshaling to type size (or word size). To turn this feature on,
use the switch \verb|-falign-to-type|. This may waste some space in the
message buffer. Since all parameters are sorted by size before marshaling,
the padding should be minimal.

\subsubsection{\tt generate-line-directive}
Generates line information in the target code using source file
information.  So, if a operation was declare on line $10$ in the
IDL file, the generated functions for this operation will be preceeded
be a pre-processor statement containing the IDL file's name and
line number of the original declaration.

\section{Warnings}
\dice{} will print warnings for different conditions if the respective
option is given. This section gives an overview of the available warning
options. All warning options start with \verb|-W|, the following options
have to be added to \verb|-W|.

\subsubsection{ignore-duplicate-fids}
This option will print warnings if there exist duplicate function identifiers
within an interface. This will normally cause an error and compilation 
abort.

\subsubsection{prealloc}
Print warnings if the malloc member of the CORBA Environment or the
\verb|CORBA_alloc| function are used.

\subsubsection{no-maxsize}
Print warnings if a parameter has no \verb|max_is| attribute assigned
to assign a maximum size. \dice{} uses heuristics to determine its
maximum size. Using this option you may detect parameters which may
need an \verb|max_is| attribute to increase or decrease the memory
allocate for them.

\subsubsection{all}
Turns on all of the above warnings.

