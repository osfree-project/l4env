% gives a short introduction on what the IDL file is
% what the generated code looks like and how to use
% it
\chapter{Interface Definitions}
\label{section1}

The use of IDL compilers is well established in distributed systems. An IDL
compiler provides you with a level of abstraction for the communication with a
remote service.

Assuming you want to write a server for L4 manually, you have to solve several
problems. You have to communicate your request from a client to the server,
which means that you build a message and send an IPC containing this message
to the server. The server has to wait for request, extract the data from the
message, and dispatches the request to the appropriate function, which does
the requested work. After this function returns, the server has to pack the
response into another message and reply to the client.

With the help of an IDL compiler you can automate all of the above
steps---except the implementation of the server function. All you have to do
is to write a description of the server's interface using an interface
description language (IDL) and translate this description with \dice{}.

Currently, \dice{} supports two different interface description
languages---CORBA IDL and DCE IDL. Since we think that DCE IDL is more
powerful, we added options to express L4 specific properties to the DCE IDL.
These attributes are described in detail in section~\ref{sec:attributes}.

For the CORBA IDL there exist multiple well-defined language mappings. Since
these are widely accepted, \dice{} generates code that conforms to the CORBA C
language mapping \cite{corba-clm}.

\section{A Simple Interface}

To illustrate the use of \dice{} we show a simple example here. The language
of our IDL examples is DCE IDL. See the appendix for a CORBA IDL version of
the example.

\subsection{Interface Description Language}

An interface description file consists of at least an {\em interface
specification} which includes one or more {\em function declarations}.  As
described above a client may send data to the server and receive a response
containing data. We have to differentiate these different direction of data
transfer from each other. Data sent to the server has the attribute \verb|in|
and data sent back to the client has the attribute \verb|out|.

\begin{verbatim}
interface simple
{
  void foo([in] int parameter);
}
\end{verbatim}

The above example sends a simple parameter to the server and does not expect
return values. Nonetheless, the client continues its work only after it
received an (empty) answer from the server. These calls are always
synchronous.  Asynchronous calls are described in detail in
section~\ref{sec:asynchronous}.

\section{Supported Types}

\subsection{Integer Types}
The DCE IDL supports the following types:

\begin{center}
\begin{longtable}{|l|l|p{6cm}|}%
  \multicolumn{3}{r}{next page \dots}\\\endfoot%
  \multicolumn{3}{l}{\dots continued from last page}\\\endhead%
  \endfirsthead%
  \endlastfoot%
  \hline
type & size & value range \\ \hline
$[$ signed $]$ small $[$ int $]$ & 8 bit & -128... 128 \\
unsigned small $[$ int $]$ & 8 bit & 0... 255 \\ 
$[$ signed $]$ short $[$ int $]$ & 16 bit & -32'768... 32'767 \\
unsigned short $[$ int $]$ & 16 bit & 0... 65'565 \\
signed & 32 bit & -2'147'483'648... 2'147'483'647 \\
$[$ signed $]$ int & 32 bit & -2'147'483'648... 2'147'483'647 \\
$[$ signed $]$ long $[$ int $]$ & 32 bit & -2'147'483'648... 2'147'483'648 \\
unsigned $[$ int $]$ & 32 bit & 0... 4'294'967'295 \\
unsigned long $[$ int $]$ & 32 bit & 0... 4'294'967'295 \\
$[$ signed $]$ long long $[$ int $]$ & 64 bit & -9'223'372'036'854'775'808... 9'223'372'036'854'775'807 \\
unsigned long long $[$ int $]$ & 64 bit & 0... 18'446'744'073'709'551'615 \\
$[$ signed $]$ hyper $[$ int $]$ & 64 bit & -9'223'372'036'854'775'808... 8'223'372'036'854'775'807 \\
unsigned hyper $[$ int $]$ & 64 bit & 0... 18'446'744'073'709'551'615 \\
\hline
\end{longtable}
\end{center}

The CORBA IDL supports the following types:

\begin{center}
\begin{tabular}{|l|l|p{5cm}|}
\hline
type & size & value range \\ \hline
short & 16 bit & -32'768... 32'767 \\
unsigned short & 16 bit & 0... 65'565 \\
long & 32 bit & -2'147'483'648... 2'147'483'648 \\
unsigned long & 32 bit & 0... 4'294'967'295 \\
long long & 64 bit & -9'223'372'036'854'775'808... 9'223'372'036'854'775'807 \\
unsigned long long & 64 bit & 0... 18'446'744'073'709'551'615 \\
\hline
\end{tabular}
\end{center}

Note that the interface description language do \emph{not} provide a type to
describe a machine word. To use parameters of the type machine word, import a
header file defining such a type an use this type.

\subsection{Floating Point Types}
The DCE and CORBA IDL support the following types:

% TODO: precision

\begin{center}
\begin{tabular}{|l|l|}
\hline
type & size \\ \hline
float & 32 bit \\
double & 64 bit \\
long double & 80 bit \\
\hline
\end{tabular}
\end{center}

\subsection{Other Types}

\dice{} also supports miscellaneous types for DCE IDL:

\begin{center}
\begin{tabular}{|l|l|l|}
\hline
type & size & values \\
\hline
byte & 8 bit & 0... 255 \\
void & undefined & none \\
unsigned char & 8 bit & 0... 255 \\
$[$ signed $]$ char & 8 bit & -128... 127 \\
boolean & 8 bit & {\tt true}, {\tt false} \\
\hline
\end{tabular}
\end{center}

The type {\tt boolean} is mapped to a {\tt unsigned char} type and therefore
has a size of 8 bit. Its values are {\tt false} if zero and {\tt true}
otherwise.

\dice{} supports the following miscellaneous types for the
CORBA IDL:

\begin{center}
\begin{tabular}{|l|l|l|}
\hline
type & size & values \\
\hline
char & 8 bit & -128... 127 \\
wchar & 16 bit & -32'768... 32'767 \\
boolean & 8 bit & {\tt true}, {\tt false} \\
octet & 8 bit & 0... 255 \\
\hline
\end{tabular}
\end{center}

The types {\tt char} and {\tt wchar} have a special semantic in the CORBA IDL.
Even though they might contains integer values, they are interpreted as
characters. The CORBA specification allows the communication code to transform
character sets and thereby change the integer value of a character. If you do
not intend to transfer string, but rather sequences of 8 Bit values, use the
{\tt octet} type instead.

\subsection{L4 specific types}

There are some L4 specific types, which have been added to the IDLs to support
L4 specific semantics.  One of these types is the flexpage type, which
expresses the mapping of memory pages from one address space into another.  A
flexpage can be transmitted using the {\tt flexpage} (or {\tt fpage}) type.

\begin{verbatim}
void map([in] fpage page);
\end{verbatim}

\section{Constructed Types}

\dice{} transmits constructed types as well. You may either include or import
a type via the mechanisms described in section~\ref{sec:import} or you define
it within the IDL file. The latter approach has the advantage that you can
give hints to the IDL compiler on how to transmit the type. The disadvantage
is, that you have to either use the generated data type in your wrapper code
or cast your own data type to the generated data type when using the generated
stubs.

\subsection{Aliased Types}

You may use {\tt typedef} to define alias names for types, which can be used
as types of parameters or members of other constructed types.

\begin{verbatim}
typedef int buffer50[50];
\end{verbatim}

The specified types are provided in the generated header files and can
therefore be used in your code.

\subsection{Arrays and Sequences}
You may specify arrays in your IDL files.

Arrays are denoted by brackets, and may contain lower and upper bounds.  You
may also specify a variable sized array by using an asterisk instead of an
empty pair of brackets.

\begin{verbatim}
typedef int unboundArray[];
typedef int unboundArray2[*];
\end{verbatim}

The following example describes an array with 20 elements of type \verb|long|.
There are different ways to express this. As mentioned you may specify lower
bounds for arrays. The generated stub code will then transmit the data
starting with the lower bound of the array ({\em currently not implemented}). 

\begin{verbatim}
void array1([in] long param[20]);
void array2([in] long param[0..20]);

// currently not supported
void array3([in] long param[10..30]);
\end{verbatim}

The following example shows some possibilities to specify variable
sized arrays. Variable sized arrays may have a fixed size, but the 
number of transmitted elements is determined at run-time. ({\em The
shown examples will not work, because the \verb|size_is| attribute
is missing.})

\begin{verbatim}
void var_array1([in] long param[*]);
void var_array2([in] long param[]);
void var_array3([in] long *param);
\end{verbatim}

In a CORBA IDL file you may use the {\tt sequence} keyword to define
array types. The CORBA C Language Mapping defines that {\tt sequence}
is translated into a {\tt struct}. Consider the following example:
\begin{verbatim}
IDL:
typedef sequence<long, 10> my_long_10;

C:
typedef struct
{
  unsigned long _maximum;
  unsigned long _length;
  long *_buffer;
} my_long_10;
\end{verbatim}

You may have noticed that the boundary of the sequence is silently dropped
when converting the sequence to the C type. You are responsible for enforcing
the boundary yourself\footnote{If you think this is unfair, please contact the
OMG.}.

The syntax of CORBA allows you to specify a sequence as type of a parameter.
When mapping this to C, this will produce a struct definition inside a
function declaration. This implies that the struct type is only known inside
the function, which means you cannot declare a variable of this type outside
the function.

\subsubsection{Strings}

Strings are regarded as a variable sized array of characters, which is
zero-terminated.  This allows to omit the specification of length attributes
for the parameter. Instead you specify an \verb|string| attribute. The
generated code will use string functions to determine the length of the
string.

\begin{verbatim}
void hello([in, string] char* world);
\end{verbatim}

In CORBA IDL you may specify strings with boundary (similar to the sequence
type). The CORBA C Language Mapping defines to map strings and wide-strings to
\verb|char*| and \verb|wchar*| respectively, ignoring the boundaries. This
implies that you may have to enforce the boundaries yourself.

\begin{verbatim}
IDL:
typedef string<25> my_name;

C:
typedef char* my_name;
\end{verbatim}

\subsection{Structured Types}

Within the IDL file you may specify a structured type---a \verb|struct|---the
same way as you would within a C/C++ header file. Furthermore it is possible
to specify attributes with each member of the defined \verb|struct|.  One of
the allowed attributes is \verb|size_is|, which is described in more detail in
Section~\ref{sec:attributes}. It allows you to specify the run-time size of an
array. The definition of such a structure could look like this:

\begin{verbatim}
struct _string
{
  [size_is(length), string] char* buffer;
  unsigned length;
};
\end{verbatim}
This is similar to the usage as parameters:
\begin{verbatim}
[in, size_is(length), string] char* buffer,
[in] unsigned length
\end{verbatim}

% Valid attributes are:
% ...

A \verb|struct| is mostly marshaled by copying it as is into the message
buffer. If the \verb|struct| contains variable sized members, such as the
\verb|buffer| member of the above example, these are marshaled separately
after the parameter of the structured type.

% packed structs!

\subsubsection{Bit-fields}

You may specify bit-fields in a structured type. Consider that this is highly
platform dependent and the usage of bit-fields may lead to unwanted results.

\begin{verbatim}
struct {
  unsigned int first : 3;
  unsigned int second : 5;
  unsigned int third : 8;
};
\end{verbatim}

\subsection{Unions}

Within your IDL specification you may define a union in C syntax or in IDL
syntax. The difference is made when marshaling the union into the message
buffer. A ``C-style'' union is simply copied into the message buffer.
Therefore it will consume the space of the largest of its members.

The ``IDL-style'' union declares a decision making variable, which is then
used to select the member to transmit. Such a union could look like this:

\begin{verbatim}
union switch(long which_member) _union
{
case 1:
  long l_mem;
case 2:
  double d_mem;
case 3:
case 4:
  long array[100];
default:
  byte status;
};
\end{verbatim}

If the variable \verb|which_member| has the value 1, the member \verb|l_mem|
is transmitted. When defining a switch variable and the case statement, you
have to consider that they are used in a C switch statement for comparison.
Thus it is valid to use a \verb|char s_var| and define the cases with
character values.

The generated code for an ``IDL-style'' union includes a switch statement to
decide which member to transmit. This may be relevant if the union may
tremendously vary in size and a comparison would possibly minimize the amount
of data to be copied. But, such a union will always need the switch variable
to be transmitted as well --- at the receiver's side there is also a switch
statement which decides what to unmarshal from the message buffer.

A union may also have constructed or variable sized members. Members of
variable size are marshaled after the union.

\section{Deriving Interfaces}
You can derive an interface from another interface. This is done
using similar syntax as with C++ classes. You name the interface to derive
followed by a colon and then the list of base interfaces.
\begin{verbatim}
interface derived : base1, base2
{
...
\end{verbatim}

The main difference to C++ class derivation is, that you mostly use derived
classes to overload functionality of the base classes. Since there is no
implementation in an IDL file, which could be overloaded, the main purpose of
interface derivation is the expansion of the base interface' functionality.
This is done by specifying new functions in the derived interface.

You may use interface derivation to unify multiple interface definitions
into one interface. The derived interface may be empty. 
\begin{verbatim}
interface all_in_one : 
  base1, base2, 
  another_scope::base3
{};
\end{verbatim}

The generated server loop will be able to receive message from \verb|base1|,
\verb|base2|, and \verb|another_scope::base3|. The server loop distinguishes
the different function calls from another as described in the following
section (Section~\ref{sec:opcode}).

\section{Constants}

You may specify constant by either using the \verb|#define| statement or a
{\tt const} declaration. The \verb|#define| statement is parsed by the
pre-processor and will not appear in the generated code. The {\tt const}
declaration will appear in the generated code as they have been defined in the
IDL file.

\begin{verbatim}
const int foo = 4;
\end{verbatim}

You may use the declared {\tt const} variables when specifying the sizes of
arrays, etc.

\section{Attributes}
\label{sec:attributes}

This section describes some attributes, which are available with the DCE IDL.
If an attribute is available for CORBA IDL as well this is specifically
mentioned.

Attributes are used to propagate knowledge about the target environment or
usage context of the IDL specification to \dice{}.  Some attributes are used
to generate code optimized for a specific platform or architecture. Others
will influence the generation of the target code to exploit features of a
specific L4 ABI. Therefore the attributes are important to provide \dice{}
with knowledge about the IDL specification and optimization potential.

However, there are attributes which might be ignored on some platforms.

\subsection{Array Attributes}
Attributes describing parameters of an array mostly end on
\verb|_is| and usually take a parameter. They are:
\verb|first_is|, \verb|last_is|, \verb|length_is|, \verb|min_is|,
\verb|max_is|, and \verb|size_is|. Currently only \verb|length_is|,
\verb|size_is|, and \verb|max_is| are supported.

Starting with the latter, the attribute is used to determine
the maximum size of an array. For example do
\begin{verbatim}
[in] int parameter[100]
\end{verbatim}
and
\begin{verbatim}
[in, max_is(100)] int parameter[]
\end{verbatim}
express the same semantics.

You may also specify a parameter or constant as the argument of the attribute.

The \verb|length_is| or \verb|size_is| attributes are used to determine the
length of a variable sized array at run-time.  The generated code will
transmit only the specified number of elements instead of the maximum size.
The \verb|str| parameter does not have a maximum size to keep the example
simple. A maximum size will enable the server loop or client stubs to know the
size of memory to allocate for the array.

The example:
\begin{verbatim}
[in, size_is(len)] char str[], 
[in] int len
\end{verbatim}

specifies a variable sized array of bytes, which has a length of \verb|len|.
\verb|len| has to be either a parameter or a constant. \dice{} first searches
for parameters with the specified name and, if it does not find one, it
searches for a constant with that name. An error is generated if neither is
given.

%TODO: exact semantics of _is attributes
%TODO: are there other attributes relevant for arrays

\subsection{Strings}

As described above is it possible to define the run-time length of a variable
sized array using the \verb|length_is| or \verb|size_is| attribute. This may
also be used for strings.  If you know that the parameter you will transmit is
a zero-terminated string, you can also let the generated code calculate the
size of the string. You simply specify the \verb|string| attribute.

\begin{verbatim}
[in, string] char* str
\end{verbatim}

In this example the \verb|str| parameter does not have a maximum size, because
it can be any size. 

\subsection{Indirect Parts IPC}

L4 allows you to transmit any data using indirect part IPC. You specify the
address and size of the original data and at receiver's side the address and
size of the receive buffer and the kernel copies the data directly. Without
indirect parts the data is copied from the source address to the message
buffer, the kernel copies the message buffer from senders side to the
receiver's message buffer, and the receiver copies the data from the message
buffer to the target variables.  The downside of the indirect part IPC is,
that the kernel may have to establish additional temporary mappings of the
data areas.

To make \dice{} use indirect part IPC for a specific parameter you may use the
\verb|ref| attribute. You have to combine it with either the \verb|size_is|
attribute or the \verb|string| attribute to specify the size of a variable
sized array.

\begin{verbatim}
[in, size_is(size), ref] long data[],
[in] long size
\end{verbatim}

You may also transmit any other kind of data using indirect part IPC. Only
specify the \verb|ref| attribute with the respective parameter and the size of
the data to be transmitted. 

With the L4 version 4 back-end all constructed data types will be transmitted
using indirect parts, even if no \verb|ref| attribute has been specified (or
rather, the compiler may choose parameters to transmit as indirect parts as it
seems fit).

Be aware that for indirect parts with the \verb|[out]| attribute the receive
buffer has to be allocated before the IPC is initiated.  Thus, the \verb|size|
parameter also has to have an \verb|[in]| attribute. The \verb|[out]|
attribute has to be present, so that the \verb|size| parameter is used to set
the send size of the indirect part.

\subsection{Receiving Indirect Parts}

The CORBA C Language Mapping specifies for the reception of variable sized
parameters the allocation of memory using the \verb|CORBA_alloc| function.  If
you know the receive buffer for the variable sized parameter before calling
the client stub, you may want to give this buffer to the client stub, so the
data is copied directly into this buffer. This can be done using
\verb|prealloc_client| attribute:

\begin{verbatim}
[out, prealloc_client, size_is(len), ref] char** str,
[out, in] int *len
\end{verbatim}

{\em The Section~\ref{sec:mem-indirect-parts} contains further information on
the memory management of indirect parts.}

Even though, the receive buffer for the indirect part has been allocated
beforehand, the size element of the indirect part is set to the value of the
\verb|len| parameter.  Thus, this parameter should have the \verb|[in]|
attribute to indicate this situation.  Also, the very parameter is used to set
the actual transmitted size when sending from the server to the client.
Therefore, you should set this variable to the actual size of the buffer to be
sent back to the client.  Otherwise the kernel might abort the IPC with an
error (if the size specified is larger than the buffer).

{\em Also note that the \verb|str| parameter has an additional reference as
\verb|out| parameter even though it is preallocated and thus no memory
allocation is necessary in the stub.}

Note: The \verb|prealloc_client| attribute also has the counter-part
\verb|prealloc_server|.  Using this attribute will generate code that
allocates memory for the parameter in the server side dispatcher function.
The allocated memory is then handed to the component function for using.
Because the environment's \verb|malloc| (or the \verb|CORBA_alloc|) function
is used for this parameter on every invocation of the respective operation,
the memory should eventually be freed.  This is automatically done by the
generated code.  It registers the allocated memory pointer in the
environment's pointer list.  This list is iterated after the
receipt\footnote{Freeing the memory before sending the message is not possible,
because the memory is needed during the reply.} of the next message and freed.
If you want to keep the memory for further processing, you have to remove the
memory from the list.  Usually the usage of \verb|prealloc_server| is not
necessary.

\subsection{Error Function}

Usually the server ignores IPC errors when sending a reply to a client or
waiting for a new request. Sometimes it is useful to use timeouts when waiting
for new request to ``do something else'' meanwhile. The wait timeout can be
specified in the \verb|CORBA_Environment| parameter of the server loop. But
how do you know when the timeout has been triggered and what does the server
loop do?

For this scenario, there exists an attribute --- \verb|error_function| ---
which makes the server loop call the specified function on every IPC error.
This allows you to implement a specific behavior for the IPC error. The called
function takes only one parameter --- an \verb|l4_msgdope_t| variable.  This
can be used to check the origin of the error and take appropriate measures.
To distinguish client and server error functions, one can use the
\verb|error_function_client| and \verb|error_function_server| attributes
respectively. Using these attributes overrides \verb|error_function|.

% XXX Todo
% The other parameter is the message buffer, which should be transmitted and
% the current \verb|CORBA_Environment|. These can be used to obtain some
% context information of the error

% describes some L4 specific scenarios and how to make
% the IDL compiler do what you want

\subsection{Using other types for transmition}

Sometimes it is convenient to use other types to transmit the data, than are
used when calling a function. An example is the usage of a constructed type,
which contains variable sized members, which are known only at run-time. If
this type is declared in a C header file, it is not possible to add attributes
to the variable sized member to define the run-time size. 

The attribute \verb|transmit_as| lets you define the type to use
when transmitting the data:
\begin{verbatim}
typedef [transmit_as(unsigned long)] 
my_C_union idl_union;
\end{verbatim}
This example will use the type \verb|unsigned long| instead of
\verb|my_C_union| when transmitting parameters of type \verb|idl_union|.

The example:
\begin{verbatim}
[in, transmit_as(unsigned char),
 size_is(size)] var_struct *t,
[in] unsigned size
\end{verbatim}
will transmit \verb|t| as a variable sized array of type \verb|unsigned char|
with size \verb|size| instead of \verb|var_struct|.

{\em Consider that the latter example requires the specification of a pointer
parameter. Without the \verb|transmit_as| attribute this would have the
semantic of an array of type \verb|var_struct| and \verb|size| members. But
with the \verb|transmit_as| attribute the parameter \verb|t| is considered a
variable sized array with \verb|size| elements of type \verb|unsigned char|.}

\section{Imported and Included Files}
\label{sec:import}

\dice{} provides you with two mechanisms to include other IDL or C/C++ files
into your IDL file. The first is the well known preprocessor directive
\verb|#include|. The second is the \verb|import| statement. Both are resolved
by \dice{}'s preprocessor.  Including a file using \verb|import| implies that
the content of this file is only scanned for IDL syntax. This includes C/C++
style type and constant definitions, but excludes C style function
declarations and definitions.

\dice{} automatically scans the file \verb|dice/dice-corba-types.h| to know
all the CORBA types. This file is searched in the given include paths.  If you
receive the error '\verb|dice-corba-types.h| cannot be found', please check
your include directories.

\subsection{C/C++ Attributes}

If you specify attributes (\verb|__attribute__((...))|) with a type
definition, they are ignored by the IDL compiler. This is important if you
specify attributes responsible for the memory layout of a data type. Thus, if
you define C types using attributes check the generated C code to see if you
data type is really copied the way you intended. (For example, the
\verb|packed| attribute of structs.)

\section{Message Passing}
\label{sec:message-passing}

Traditional RPC semantics include the sequence of actions of sending a request
to a server, processing the request and waiting for a reply.  Opposed to that,
exist programming languages, which use communication mechanisms to synchronize
or exchange data at specific points in the program. In Ada these points are
called ``rendezvous''. With this technique it is possible to write programs,
where activity is associated the flow of data.

To provide the level of abstraction we introduced for RPCs for message passing
also, we extended the DCE IDL to let \dice{} generate simple communication
stubs.

A ``rendezvous'' point is given on L4 when one thread waits for a specific
message and another thread is sending this message to the waiting thread.
Since a message is defined in the IDL by an operation of an interface, we
associate an attribute with the operation to denote the message passing
semantic. It is possible for a message to be consumed by a thread or emitted
by it. This is similar to the semantic for parameters of a RPC. Thus we used
the \verb|in| and \verb|out| attribute for message passing.

\begin{verbatim}
interface mp
{
  [in] void send(int param);
  [out] int receive(int *param);
};
\end{verbatim}

The parameters of an \verb|in| operation are automatically \verb|in| as
well. The usage of return types other than void or \verb|out| parameters
will cause an error. Accordingly are all parameters of an \verb|out|
operation \verb|out| themselves.

\dice{} generates a different set of functions for message passing
operations. At the sender's side\footnote{Beware, this can be the client's
side \emph{and} the server's side.} the functions end on \verb|_send|, because
they only involve the send phase of the IPC. At the receiver's side
there are two functions, one ending on \verb|_wait| and the other ending
on \verb|_recv|. These functions receive the specified message from
any sender or a specific sender respectively. For the receiver's side
there also exists an \verb|_unmarshal| function.

If the receiver's side is also the server's side the message
can also be received by a server loop. For this case \dice{} also
generates the before-mentioned functions (\verb|_unmarshal| and 
\verb|_component|). 

A synonym for a function's \verb|in| attribute is the \verb|oneway|
attribute, known from DCE IDL.

\subsection{Optimizations}
If you would like to use message passing stubs and know what you are
doing, you might want to use some extra attributes. Assuming you do
use the message passing function as \emph{send} and \emph{receive}
pairs only---meaning no single spot where you wait for more than one
message. You can specify the \verb|noopcode| attribute for a function.
This attribute makes \dice{} skip opcode marshaling and unmarshaling.
This save one message word for some kernel interfaces (V2 and X0).
But it bares the generated code to validate if the message you received
is really the message you wanted to get (which is otherwise done using
the opcode).

A similar function attribute is the \verb|noexceptions| attribute. When
using simple RPC mechanisms, e.g. function calls, an exception is 
usually returned. This allows the caller to check, for instance, if the
server recognized the opcode. Using the \verb|noexceptions| attribute
generates function stubs, which do not transfer this exception code
back from the server to the calling client. This provides an additional
message word for data transfer.

\section{Asynchronous Servers}
\label{sec:asynchronous}
Asynchronous servers are servers, which receive a request, enqueue or
propagate it and immediately wait for the next request. At a later 
moment (asynchronously) the reply is delivered to the client. To
enable asynchronous processing of a function, use the attribute 
\verb|allow_reply_only| for this function. The \verb|*_component| function
will have an additional parameter (\verb|short *_dice_reply|) which is
set to the value of \verb|DICE_REPLY|. This implies that the component
function does work synchronously per default. To delay the reply
set the value of \verb|_dice_reply| to \verb|DICE_NO_REPLY| before
returning.

When the work is finished you can sent a reply to this request to 
the client using the \verb|_reply| function generated for this
function. It expects the \verb|CORBA_Object| for the client, the
returned parameters, and a reference to a \verb|CORBA_Server_Environment|.

\subsubsection{Example}
The following asynchronous server has one function, which receives a
parameters and returns a result. This function should be processed
asynchronously:

\begin{verbatim}
[allow_reply_only]
void foo([in] long p1, [out] long* p2);
\end{verbatim}

To return the result to the client, the server has to respond from the same
thread as the client sent the request to. Therefore we will need a
function, which does handle the replies:

\begin{verbatim}
void notify([in] long p1, [in] long p2);
\end{verbatim}

\verb|p2| is not a reference, because it is a simple \verb|in| parameter
for the \verb|notify| function. \verb|p1| is used as an identifier for the
job -- it has to be unique.

The \verb|_component| function may create a worker thread, which does all the
processing of the job. To be able to send the result of the job back to the
client the server will need to store the association of client ID to job ID.
(In fact the client ID could be used as job ID if a client ID cannot be
used again while the job is still being processed, e.g. the client did not
receive a reply yet.) The worker thread should be informed about the job to 
process. 

After the job is finished the worker thread should invoke the \verb|notify| 
function with the job ID (\verb|p1|) and the result (\verb|p2|). The
implementation (\verb|_component| function) of this method calls the \verb|_reply|
function of the \verb|foo| method.

\subsubsection{Remarks}
To send a reply to a client you have to store its ID and an identifier
for the job/request. (Remember: \verb|CORBA_Object| is a pointer.)

For some architectures (e.g. L4 version 2) the client initiated a
call which expects the reply from the same thread as the request
has been sent to. Therefore, you will need for each asynchronous function
a function in the server loop which delivers the replies.

Inspect the example \verb|async| for an implementation example.

\section{Thoughts About \dice{} Specific Types}
When using IDL types, they mostly can be mapped to L4 specific 
message types using attributes. An indirect part is for 
instance denoted by the DCE \verb|ref| attribute. It thereby
follows the semantic declared in the DCE IDL specification.

However, there are some L4 specific IPC semantics, which
are hard to express using attributes. The following few
sections will try to explain the semantic of some of the L4
specific types and ideas on how to represent them using other 
methods then currently used ones ({\tt flexpage} type).

\subsection{\tt flexpage}
\label{sec:flexpage}
A flexpage is a memory region consisting of at least one 
memory frame which can be transferred from one address space
to another (or rather the rights to access it). It consists
of a base address, a size, and the transferred rights, including
the ownership. There also apply restrictions to the address ---
it has to be page size aligned --- and the size of a flexpage.
To express this using attributes, we could write
something like:
\begin{verbatim}
[in, memory, size_is(pages), 
rights(access)] void* base_address, 
[in] unsigned int pages,
[in] unsigned int access
\end{verbatim}
This scheme would add two new attributes (\verb|memory| and
\verb|rights|) and require the user to know all necessary
parameters. This can be simplified by using a predefined 
type, such as:
\begin{verbatim}
typedef struct {
[memory, size_is(pages), rights(access)]
void *base_address;
unsigned int pages;
unsigned int access;
} flexpage;
\end{verbatim}
An IDL compiler should map this constructed type 
onto the existing \verb|l4_fpage_t| type of L4 and use it
appropriately.
Since one of the goals of the IDL compiler is the usage of 
existing data types --- such as \verb|l4_fpage_t| --- this would
be contra productive.

Therefore we decided to use the special type \verb|flexpage| or
\verb|fpage|. It is directly mapped to the \verb|l4_fpage_t| type
and implies the above mentioned characteristics of a memory
region. It is simpler in the usage, because the user does
not have to transfer the L4 type to the predefined IDL type.

%TODO: make dice recognize l4_fpage_t

\subsection{Interrupts}
One of the future goals is to make the interaction with the
kernel transparent to the user as well. One could image to
describe the kernel interface using an IDL file, which defines
all type and system calls and the IDL compiler generates the
appropriate code to invoke the system calls.

This idea involves the abstraction of interrupt IPC. Therefore
an IDL type for interrupts should exist, which can be used to
describe an interrupt IPC.

{\it These are abstract ideas, which have not been implemented
and will not be in the near future.}

