\documentclass[twocolumn,10pt]{article}
\usepackage[a4paper,dvips]{geometry}
\usepackage{fancyvrb}

\title{Towards A Common Environment For L4 Applications}
\author{Lars Reuther}
\date{}

\begin{document}
\maketitle

\section{The Problem}

Currently, most L4 applications have their very own idea about the
environment (libraries, interfaces and so on) in which they are
executed. Almost every programmer has his own set of libraries he uses
to build his applications, which results in huge problems if someone
tries to combine components developed by different authors. Frequent
problems are conflicting implementations of common functions (like
{\tt printf}) or conflicts caused by the lack of a central management
of resources like threads or virtual memory. 
  
\section{A Common Environment}

The intention of this section is to define a set of functions which
describe a minimal environment. This minimal environment is available
for every L4 application. Hence, all applications and especially all
libraries can use these functions. Libraries which are intended to be
used by many different applications should only use this functions to
avoid dependencies to other libraries.

\subsection{Console Input/Output}

Basic console functions to read/write characters:
\begin{Verbatim}[fontsize=\small]
int console_putchar(int c);
int console_getchar(void);
int console_puts(const char * s);
int console_putbytes(const char * s, int len);
\end{Verbatim}

\subsection{Minimal libc}

\begin{Verbatim}[fontsize=\small]
int printf(const char * __format, ...);
int perror(const char * __format, ...);
\end{Verbatim}
(Print error message. It can be used to redirect error messages to a
special error console instead of the standard console.)

\begin{Verbatim}[fontsize=\small]
void * memcpy(void * __to, const void * __from, 
              unsigned int __n);
char * strcpy(char * __dest, const char * __src);
...
\end{Verbatim}
And all the other string manipulation functions defined by libc.

\begin{Verbatim}[fontsize=\small]
int getopt(int argc, char * const argv[], 
           const char * optstring);
\end{Verbatim}
(Parse command line options)

\subsection{Memory Allocation}

What's a reasonable abstraction for memory allocation?
\begin{Verbatim}[fontsize=\small]
get_free_pages
sbrk
malloc
\end{Verbatim}

\subsection{Virtual Memory Management}

We use the abstraction of vm regions backed by dataspaces for the
management of the virtual memory of a task. Hence, operations to
manipulate the virtual memory are attach and detach dataspaces to vm
regions.

\begin{Verbatim}[fontsize=\small]
int attach_to_region(dataspace_t * ds, 
                     unsigned long addr,
                     unsigned long size, 
                     unsigned long ds_offs,
                     region_t * region); 
\end{Verbatim}
(Attach dataspace to region $<$addr,addr+size$>$, starting at offset
{\tt ds\_offs} in dataspace.)

\begin{Verbatim}[fontsize=\small]
int attach(dataspace_t * ds,
           unsigned long size,
           unsigned long ds_offs,
           unsigned long * addr,
           region_t * region);
\end{Verbatim}
(Attach dataspace to any appropriate region.)

\begin{Verbatim}[fontsize=\small]
int detach(region_t region);
\end{Verbatim}

\subsection{Thread Management}

\emph{To be defined in detail, see proposal for thread library}

\subsection{Name Service}
 
A simple root name service is part of the standard environment.

\begin{Verbatim}[fontsize=\small]
int names_register(const char * name);
int names_unregister(const char * name);
int names_query_name(const char * name, 
                     l4_threadid_t * id);
int names_waitfor_name(const char * name, 
                       l4_threadid_t * id, 
                       const int timeout);
\end{Verbatim}

\subsection{L4 Kernel Information}

Information about the L4 microkernel the application runs on is
provided by a L4 system library.  It offers
\begin{itemize}
\item information about the L4 kernel version 
\item an interface to the L4 kernel info page
\end{itemize}

\section{Implementation Notes}

The functions described in the previous section are implemented in a
set of libraries: 
\begin{itemize}
\item {\tt l4cio}
\item {\tt l4c}
\item {\tt l4mem}
\item {\tt l4rm}
\item {\tt l4thread}
\item {\tt l4names}
\item {\tt l4kernel}
\item {\tt l4env}\\
(Startup and glue code.)
\end{itemize}
Ideally, all these libraries are implemented as dynamically loadable
libraries. They can be replaced by enhanced implementation as long as
those implementations still provide all the defined
functions. E.g. l4c can be replaced by a fully featured libc.
 
\subsection{System Configuration}

System configuration data is provided by l4env. Following information
can be requested:
\begin{itemize}
\item size of the virtual memory
\item number of threads used by the task
\item default stack size for new threads
\item id of the root name server
\item id of the default memory server
\end{itemize}
The values are initially set by the loader of a task.
 
\section{Summary}

This short paper tries to define a common environment for L4
applications. This environment should enable programmers to write
generic components or libraries without further dependencies to other
libraries.

%\onecolumn
%
%\begin{appendix}
%
%\section{Common environment API}
%
%\emph{to be done...}
%
%\end{appendix}

\end{document}
