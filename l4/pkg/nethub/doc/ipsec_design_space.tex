%% LyX 1.3 created this file.  For more info, see http://www.lyx.org/.
%% Do not edit unless you really know what you are doing.
\documentclass[english]{scrartcl}
\usepackage[T1]{fontenc}
\usepackage[latin1]{inputenc}
\setlength\parskip{\medskipamount}
\setlength\parindent{0pt}
\usepackage{array}
\usepackage{rotating}
\usepackage{graphicx}

\makeatletter

%%%%%%%%%%%%%%%%%%%%%%%%%%%%%% LyX specific LaTeX commands.
%% Bold symbol macro for standard LaTeX users
\newcommand{\boldsymbol}[1]{\mbox{\boldmath $#1$}}

%% Because html converters don't know tabularnewline
\providecommand{\tabularnewline}{\\}

\usepackage{babel}
\makeatother
\begin{document}

\section{Design space of Separated IPSec Implementations}

The design-space problem arises in scenarios where incoming IPSec
streams need to be securely diverted to different security domains.
Figure~\ref{fig:Multiplexing-in-Linux} and Figure~\ref{fig:Multiplexing-in-Viaduct}
show examples for this scenario.

%
\begin{figure}[tbhp]
\begin{center}\includegraphics[%
  width=0.70\columnwidth]{viaduct_mux}\vspace{-4ex}\end{center}


\captionbelow{Multiplexing in Linux protocol stubs. The L$^{4}$Linuxes on the
left side are inside the protected networks. The L$^{4}$Linux on
the right side is outside of the protected networks. The two inside
L$^{4}$Linux instances can be in a different security level each,
where packets of one domain must not be seen in the other domain.
The routing of incomming packets is done in the outside L$^{4}$Linux,
which handles only encrypted traffic. \label{fig:Multiplexing-in-Linux}}
\end{figure}


%
\begin{figure}[tbhp]
\begin{center}\includegraphics[%
  width=0.70\textwidth,
  keepaspectratio]{viaduct_vr}\vspace{-4ex}\end{center}


\captionbelow{Multiplexing in Viaduct. Similar to Figure~\ref{fig:Multiplexing-in-Linux}
the L$^{4}$Linuxes on the left side are inside distinct protected
networks and the rightside L$^{4}$Linux is outside of the protection
domains. The diagram inside the Viaduct shows how routing can be done
inside the Viaduct. \label{fig:Multiplexing-in-Viaduct}}
\end{figure}



\subsection{The design space has the following dimensions}

\begin{description}
\item [Point~of~Multiplexing]Outside Linux, Viaduct, inside Linux
\item [Multiplexing~Strategy]Broadcast packets to all receivers, route
packets to designated receiver
\end{description}
Table~\ref{tab:Seperated-IPSec-Design} shows the design space with
pros and cons for the different implementations. The variations where
the inside Linux, which sees unencrypted traffic, should do the multiplexing
is nonsense, because Linux is not trustworthy and therefore one Linux
instance must only see unencrypted traffic of exactly one security
domain. The variation where broadcasting is implemented in the Viaduct
is of no use because it adds complexity to the Viaduct with no benefit
--- broadcasting could be done more simple in the Linux protocol stub. 

Useful are only the variations with broadcasting in the Linux protocol
stub (LB) and routing either in the Linux protocol (LR) stub or in
the Viaduct (VR).

%
\begin{table}[tbhp]
\begin{center}\begin{tabular}{|l||c>{\raggedright}p{0.25\columnwidth}|c>{\raggedright}p{0.25\columnwidth}|>{\raggedright}p{0.25\columnwidth}|}
\hline 
\multicolumn{1}{|l||}{}&
\multicolumn{2}{l|}{Outside Linux}&
\multicolumn{2}{l|}{Viaduct}&
Inside Linux\tabularnewline
\hline
\hline 
\begin{sideways}
\hspace{-10em}Broadcast%
\end{sideways}&
\multicolumn{2}{l|}{LB}&
\multicolumn{2}{l|}{VB}&
\tabularnewline
\cline{2-3} \cline{4-5} 
\begin{sideways}
%
\end{sideways}&
+&
Simple Viaduct (1-1)&
--&
More complex Viaduct&
Nonsense --- trust moved to Linux\tabularnewline
&
+&
Simple Linux protocol stub&
+&
Simple Linux protocol stub&
\tabularnewline
&
-~-&
\textsl{n}~SADB lookups (each Viaduct)&
--&
\textsl{n}~SADB lookups&
\tabularnewline
&
+&
Simple SA management (each SA in one Viaduct)&
+&
Simple SA management (each SA in one SADB)&
\tabularnewline
\hline 
\begin{sideways}
\hspace{-14.5em}Route%
\end{sideways}&
\multicolumn{2}{l|}{LR}&
\multicolumn{2}{l|}{VR}&
\tabularnewline
\cline{2-3} \cline{4-5} 
\begin{sideways}
%
\end{sideways}&
+&
Simple Viaduct (1-1)&
--&
More complex Viaduct&
Nonsense --- trust moved to Linux\tabularnewline
\begin{sideways}
%
\end{sideways}&
--&
Complicated Linux protocol stub --- must do routing based on SA-Key
(needs SADB)&
+&
Simple Linux protocol stub&
\tabularnewline
&
{*}&
2~SADB lookups (Linux + Viaduct)&
+&
1~SADB lookup in Viaduct&
\tabularnewline
&
--&
Duplicate SA management (same SA in Linux and Viaduct)&
+&
Simple SA management (each SA in one SADB)&
\tabularnewline
\hline
\end{tabular}\vspace{-3ex}\end{center}


\caption{Separated IPSec Design Space\label{tab:Seperated-IPSec-Design}}
\end{table}



\subsection{Performance point of view}

\begin{description}
\item [LB]\textsl{n} times packet header parsing, \textsl{n} times SADB
lookup (\textsl{n}-1 lookups result in drops) \textsl{(worst)}
\item [LR]2 times packet header parsing, 2 times SADB lookup
\item [VR]1 time packet header parsing, 1 SADB lookup \textsl{(best)}
\end{description}

\subsection{Implementation complexity}

\begin{description}
\item [LB]Viaduct as it is, implementation of \textsl{n} times packet enqueue
in Linux protocol stub \textsl{(simple)}
\item [LR]Viaduct as it is, implementation of simple routing table based
on SA-Keys in Linux protocol stub \textsl{(complex w.r.t. Linux drivers)}
\item [VR]Support of multiple interfaces in the Viaduct additional interface
pointer in each SA, Linux protocol stub as it is \textsl{(complex
w.r.t Viaduct)}
\end{description}

\subsection{Conclusion}

Only LR and VR are practicable. Differences between LR and VR are:

\begin{itemize}
\item The point of routing
\item Distribution/Locality of SADB and SPD (and the performance implications)
\item Code complexity in Viaduct (seems to be a minor issue)
\end{itemize}

\subsection{Questions}

How often are SPIs changed for one connection?

How often are SAs re-keyed (without changing SPIs)?
\end{document}
