\documentclass[twocolumn]{article}
\usepackage[latin1]{inputenc}

\title{Protocol of discussion between Jork, Lars and Sebastian on 02/01/2001}
\date{02/02/2001}
\author{Jork L�ser}

\newcommand{\BI}{\begin{itemize}}
\newcommand{\EI}{\end{itemize}}
\newcommand{\BE}{\begin{enumerate}}
\newcommand{\EE}{\end{enumerate}}
\newcommand{\IT}{\item}


\begin{document}

\maketitle

\section{Problem}

Currently, there is no uniform way for components to signal the end of
data-transmission to the controlling application.

\section{Ad-hoc ideas}

One solution could be to offer an additional flick-interface at the
components in separate threads. For security and isolation reasons and
because of flick limitations, this requires one additional thread per socket
and component. At the controlling application this also requires an
additional thread per socket.

\section{Solution}

DSI offers events. DSI-Events (referenced to as events in this
document) can be sent from components to applications. Sending the
events is done using IPCs with timeout 0, preventing to be blocked by
DoS-attacks. Receiving events imposes blocking until the event is set.


\section{Implementation}

The number of different event types is limited, enumerated from 0 to
E-1. A component can set an event on a socket, the event can be
received and reset by the creator of the socket. The events are
counted, which means a component can set the same event multiple
times, and the receiver can receive the same amount of events of this
type.

\subsection{Component Side}

DSI defines a function to be used by the component implementation to
set events. DSI also creates a thread at the component, called
signalling thread. This thread serves all requests regarding
event-handling at this component.

\subsubsection{Event-related Function}

DSI defines the following function to be used by a component
implementation:

\verb'dsi_set_event(socket, event_type)': This function increments
the event-counter for the event of the specified type for the
specified socket. It also sends an request with timeout never(!) to the
signalling thread. The timeout imposes a break of limited, but unknown
length. Thus, this function should not be used in a time-critical
environment.

\subsubsection{The Signalling thread at the component}

The signalling thread uses an open-wait function to receive IPCs from
the threads of the component and to receive IPCs from DSI
applications.  The requests are not necessarily answered in-order.

The following component-requests (request from the component) is defined:
\BI
\IT \verb'notify-event-change(event)': The
    \verb'dsi_set_event()'-function generates this request to notify
    the signalling thread about a changed event-status. In reaction,
    the signalling thread checks for pending application-requests and
    answers as appropriate.
\EI

The following application-requests (requests from applications) are defined:
\BI
\IT \verb'wait-for-event(socket, event_type)': This request will be
    answered immediately if the event-counter of the specified
    event-type and socket is 1 or greater. Otherwise, the answer will
    be postponed until a notify-event-change request is received and
    the event-counter is 1 or greater at this time.

\IT \verb'reset-event(socket, event_type)': This request will
    decrement the event-counter of the specified event-type and socket.
\EI

The application-requests are sent with an IPC-call allowing the
signalling thread to answer with a timeout 0. This ensures that
component-requests will be answered by the signalling thread at some
point in time. But the maximum delay is influenced by incoming
client-requests and cannot be calculated.

\subsubsection{Coding Issues}

As far as we see, the implementation is straight-forward.

Just make sure to check if the application-requests are sent from the
same task that created the socket.


\subsection{Application Side}


\subsubsection{Event-related Functions}

DSI offers the following functions to deal with events at
applications:

\BI
\IT \verb'dsi_get_events(stream)': This function returns the events
     that are set and received for the given stream (as a bit-mask,
     separate for send and receive). This does not necessarily
     correspond to the current situation at the component. This
     function should be called only after calling \verb'dsi_select()'.

\IT \verb'dsi_select(in stream[], out ev_stream[]': This function
    operates similar to the Unix-select() system-call. It blocks until
    an event on one of the given streams in \verb'stream' is set. Some
    of the streams having events set are returned in
    \verb'ev_stream[]', at least one. It is up to the caller to
    determine which events are set by iterating through
    \verb'ev_stream' using \verb'dsi_get_events'.

\IT \verb'dsi_reset_event(stream, event_type, send_receive)': This
    function decrements the event-counter for the given event-type of
    the specified socket of the stream at the according component.
\EI


\subsubsection{Coding issues}

The implementation of \verb'dsi_get_events()' is straight-forward,
the same holds for \verb'dsi_reset_event'.

\verb'dsi_select()' is implemented using multiple threads and a semaphore.
This function does the following steps:

\BE
\IT create a semaphore (\verb'sem')
\IT creates a thread for each socket of the given streams in \verb'stream'
\IT do \verb'down(sem)'
\IT kills all the created threads
\EE

Each created thread does the following steps:

\BE
\IT issue a wait-for-event-request to the component owning the socket
\IT after reply, mark the event in the local socket-reference in the
    stream
\IT do \verb'up(sem)'
\IT wait
\EE

\paragraph{Discussion}
Bear in mind, that the threads can be stopped every time. As a
result, we can miss some of the event notifications. Thats why, the
components must not reset an event themselves. Instead, we need the
explicit reset-request to signal that we got the event.

For the signalling-thread at the components it is no problem that we
kill the requesting threads. They are sending their reply with timeout
0 and ignore any errors.


\section{Evaluation}

This solution allows to send a number of events from components to
applications. This is suited to implement the end-notification, and
might be used to pass other events.

The API follows well-established methods, such as the select-call.

The implementation requires the creation of a number of threads on
every select-call, which is costly. However, this effort is reasonable
for end-of-transmission-notifications.

\end{document}
