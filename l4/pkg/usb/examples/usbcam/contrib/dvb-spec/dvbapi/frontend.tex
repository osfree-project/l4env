\devsec{DVB Frontend API}

The DVB frontend device controls the tuner and DVB demodulator hardware.
It can be accessed through \texttt{/dev/dvb/adapter0/frontend0}.
Data types and and ioctl definitions can be accessed by including
\texttt{linux/dvb/frontend.h} in your application.

DVB frontends come in three varieties: DVB-S (satellite), DVB-C (cable)
and DVB-T (terrestrial). Transmission via the internet (DVB-IP) is
not yet handled by this API but a future extension is possible.
For DVB-S the frontend device also supports satellite equipment control
(SEC) via DiSEqC and V-SEC protocols. The DiSEqC (digital SEC) specification
is available from Eutelsat \texttt{http://www.eutelsat.org/}.

Note that the DVB API may also be used for MPEG decoder-only PCI cards,
in which case there exists no frontend device.

\devsubsec{Frontend Data Types}

\devsubsubsec{frontend type}
\label{frontendtype}

For historical reasons frontend types are named after the
type of modulation used in transmission.

\begin{verbatim}
typedef enum fe_type {
        FE_QPSK,   /* DVB-S */
        FE_QAM,    /* DVB-C */
        FE_OFDM    /* DVB-T */
} fe_type_t;
\end{verbatim}

\devsubsubsec{frontend capabilities}
\label{frontendcaps}

Capabilities describe what a frontend can do. Some capabilities
can only be supported for a specific frontend type.

\begin{verbatim}
typedef enum fe_caps {
        FE_IS_STUPID                  = 0,
        FE_CAN_INVERSION_AUTO         = 0x1,
        FE_CAN_FEC_1_2                = 0x2,
        FE_CAN_FEC_2_3                = 0x4,
        FE_CAN_FEC_3_4                = 0x8,
        FE_CAN_FEC_4_5                = 0x10,
        FE_CAN_FEC_5_6                = 0x20,
        FE_CAN_FEC_6_7                = 0x40,
        FE_CAN_FEC_7_8                = 0x80,
        FE_CAN_FEC_8_9                = 0x100,
        FE_CAN_FEC_AUTO               = 0x200,
        FE_CAN_QPSK                   = 0x400,
        FE_CAN_QAM_16                 = 0x800,
        FE_CAN_QAM_32                 = 0x1000,
        FE_CAN_QAM_64                 = 0x2000,
        FE_CAN_QAM_128                = 0x4000,
        FE_CAN_QAM_256                = 0x8000,
        FE_CAN_QAM_AUTO               = 0x10000,
        FE_CAN_TRANSMISSION_MODE_AUTO = 0x20000,
        FE_CAN_BANDWIDTH_AUTO         = 0x40000,
        FE_CAN_GUARD_INTERVAL_AUTO    = 0x80000,
        FE_CAN_HIERARCHY_AUTO         = 0x100000,
        FE_CAN_MUTE_TS                = 0x80000000,
        FE_CAN_CLEAN_SETUP            = 0x40000000
} fe_caps_t;
\end{verbatim}

\devsubsubsec{frontend information}
\label{frontendinfo}

Information about the frontend ca be queried with
FE\_GET\_INFO (\ref{fegetinfo}).

\begin{verbatim}
struct dvb_frontend_info {
        char       name[128];
        fe_type_t  type;
        uint32_t   frequency_min;
        uint32_t   frequency_max;
        uint32_t   frequency_stepsize;
        uint32_t   frequency_tolerance;
        uint32_t   symbol_rate_min;
        uint32_t   symbol_rate_max;
        uint32_t   symbol_rate_tolerance;     /* ppm */
        uint32_t   notifier_delay;            /* ms */
        fe_caps_t  caps;
};
\end{verbatim}

\devsubsubsec{diseqc master command}
\label{diseqcmastercmd}

A message sent from the frontend to DiSEqC capable equipment.

\begin{verbatim}
struct dvb_diseqc_master_cmd {
        uint8_t msg [6]; /*  { framing, address, command, data[3] } */
        uint8_t msg_len; /*  valid values are 3...6  */
};
\end{verbatim}

\devsubsubsec{diseqc slave reply}
\label{diseqcslavereply}

A reply to the frontend from DiSEqC 2.0 capable equipment.

\begin{verbatim}
struct dvb_diseqc_slave_reply {
        uint8_t msg [4]; /*  { framing, data [3] } */
        uint8_t msg_len; /*  valid values are 0...4, 0 means no msg  */
        int     timeout; /*  return from ioctl after timeout ms with */
};                       /*  errorcode when no message was received  */
\end{verbatim}

\devsubsubsec{SEC voltage}
\label{secvoltage}

The voltage is usually used with non-DiSEqC capable LNBs to
switch the polarzation (horizontal/vertical). 
When using DiSEqC epuipment this voltage has to be switched consistently 
to the DiSEqC commands as described in the DiSEqC spec.

\begin{verbatim}
typedef enum fe_sec_voltage {
        SEC_VOLTAGE_13,
        SEC_VOLTAGE_18
} fe_sec_voltage_t;
\end{verbatim}

\devsubsubsec{SEC continuous tone}
\label{sectone}

The continous 22KHz tone is usually used with non-DiSEqC capable LNBs to
switch the high/low band of a dual-band LNB.
When using DiSEqC epuipment this voltage has to be switched consistently 
to the DiSEqC commands as described in the DiSEqC spec.

\begin{verbatim}
typedef enum fe_sec_tone_mode {
        SEC_TONE_ON,
        SEC_TONE_OFF
} fe_sec_tone_mode_t;
\end{verbatim}

\devsubsubsec{SEC tone burst}
\label{sectoneburst}

The 22KHz tone burst is usually used with non-DiSEqC capable
switches to select between two connected LNBs/satellites.
When using DiSEqC epuipment this voltage has to be switched consistently 
to the DiSEqC commands as described in the DiSEqC spec.

\begin{verbatim}
typedef enum fe_sec_mini_cmd {
        SEC_MINI_A,
        SEC_MINI_B
} fe_sec_mini_cmd_t;
\end{verbatim}


\devsubsubsec{frontend status}
\label{frontendstatus}

Several functions of the frontend device use the fe\_status data 
type defined by
\begin{verbatim}
typedef enum fe_status {
        FE_HAS_SIGNAL     = 0x01,   /*  found something above the noise level */
        FE_HAS_CARRIER    = 0x02,   /*  found a DVB signal  */
        FE_HAS_VITERBI    = 0x04,   /*  FEC is stable  */
        FE_HAS_SYNC       = 0x08,   /*  found sync bytes  */
        FE_HAS_LOCK       = 0x10,   /*  everything's working... */
        FE_TIMEDOUT       = 0x20,   /*  no lock within the last ~2 seconds */
        FE_REINIT         = 0x40    /*  frontend was reinitialized,  */
} fe_status_t;                      /*  application is recommned to reset */
\end{verbatim}
to indicate the current state and/or state changes of 
the frontend hardware. 


\devsubsubsec{frontend parameters}
\label{frontendparameters}

The kind of parameters passed to the frontend device for tuning 
depend on the kind of hardware you are using.
All kinds of parameters are combined as a union in the 
FrontendParameters structure:
\begin{verbatim}
struct dvb_frontend_parameters {
        uint32_t frequency;       /* (absolute) frequency in Hz for QAM/OFDM */
                                  /* intermediate frequency in kHz for QPSK */
        fe_spectral_inversion_t inversion;
        union {
                struct dvb_qpsk_parameters qpsk;
                struct dvb_qam_parameters  qam;
                struct dvb_ofdm_parameters ofdm;
        } u;
};
\end{verbatim}

For satellite QPSK frontends you have to use the \verb|QPSKParameters| member 
defined by
\begin{verbatim}
struct dvb_qpsk_parameters {
        uint32_t        symbol_rate;  /* symbol rate in Symbols per second */
        fe_code_rate_t  fec_inner;    /* forward error correction (see above) */
};
\end{verbatim}
for cable QAM frontend you use the \verb|QAMParameters| structure
\begin{verbatim}
struct dvb_qam_parameters {
        uint32_t         symbol_rate; /* symbol rate in Symbols per second */
        fe_code_rate_t   fec_inner;   /* forward error correction (see above) */
        fe_modulation_t  modulation;  /* modulation type (see above) */
};
\end{verbatim}
DVB-T frontends are supported by the \verb|OFDMParamters| structure
\begin{verbatim}
struct dvb_ofdm_parameters {
        fe_bandwidth_t      bandwidth;
        fe_code_rate_t      code_rate_HP;  /* high priority stream code rate */
        fe_code_rate_t      code_rate_LP;  /* low priority stream code rate */
        fe_modulation_t     constellation; /* modulation type (see above) */
        fe_transmit_mode_t  transmission_mode;
        fe_guard_interval_t guard_interval;
        fe_hierarchy_t      hierarchy_information;
};
\end{verbatim}

In the case of QPSK frontends the \verb|Frequency| field specifies the 
intermediate frequency, i.e. the offset which is effectively added to the 
local oscillator frequency (LOF) of the LNB.
The intermediate frequency has to be specified in units of kHz.
For QAM and OFDM frontends the Frequency specifies the absolute frequency
and is given in Hz.

The Inversion field can take one of these values:
\begin{verbatim}
typedef enum fe_spectral_inversion {
        INVERSION_OFF,
        INVERSION_ON,
        INVERSION_AUTO
} fe_spectral_inversion_t;
\end{verbatim}
It indicates if spectral inversion should be presumed or not. 
In the automatic setting (\verb|INVERSION_AUTO|) the hardware will
try to figure out the correct setting by itself.

\noindent
The possible values for the \verb|FEC_inner| field are
\begin{verbatim}
typedef enum fe_code_rate {
        FEC_NONE = 0,
        FEC_1_2,
        FEC_2_3,
        FEC_3_4,
        FEC_4_5,
        FEC_5_6,
        FEC_6_7,
        FEC_7_8,
        FEC_8_9,
        FEC_AUTO
} fe_code_rate_t;
\end{verbatim}
which correspond to error correction rates of 1/2, 2/3, etc.,
no error correction or auto detection.

\noindent 
For cable and terrestrial frontends (QAM and OFDM) one also has to 
specify the quadrature modulation mode which can be one of the following:
\begin{verbatim}
typedef enum fe_modulation {
	QPSK,
        QAM_16,
        QAM_32,
        QAM_64,
        QAM_128,
        QAM_256,
        QAM_AUTO
} fe_modulation_t;
\end{verbatim}

Finally, there are several more parameters for OFDM:
\begin{verbatim}
typedef enum fe_transmit_mode {
        TRANSMISSION_MODE_2K,
        TRANSMISSION_MODE_8K,
        TRANSMISSION_MODE_AUTO
} fe_transmit_mode_t;
\end{verbatim}

\begin{verbatim}
typedef enum fe_bandwidth {
        BANDWIDTH_8_MHZ,
        BANDWIDTH_7_MHZ,
        BANDWIDTH_6_MHZ,
        BANDWIDTH_AUTO
} fe_bandwidth_t;
\end{verbatim}

\begin{verbatim}
typedef enum fe_guard_interval {
        GUARD_INTERVAL_1_32,
        GUARD_INTERVAL_1_16,
        GUARD_INTERVAL_1_8,
        GUARD_INTERVAL_1_4,
        GUARD_INTERVAL_AUTO
} fe_guard_interval_t;
\end{verbatim}

\begin{verbatim}
typedef enum fe_hierarchy {
        HIERARCHY_NONE,
        HIERARCHY_1,
        HIERARCHY_2,
        HIERARCHY_4,
        HIERARCHY_AUTO
} fe_hierarchy_t;
\end{verbatim}


\devsubsubsec{frontend events}
\label{frontendevents}

\begin{verbatim}
struct dvb_frontend_event {
        fe_status_t status;
        struct dvb_frontend_parameters parameters;
};
\end{verbatim}

\clearpage


\devsubsec{Frontend Function Calls}

\function{open()}{
  int open(const char *deviceName, int flags);}{
  This system call opens a named frontend device (/dev/dvb/adapter0/frontend0)
  for subsequent use. Usually the first thing to do after a successful open
  is to find out the frontend type with FE\_GET\_INFO.

  The device can be opened in read-only mode, which only allows
  monitoring of device status and statistics, or read/write mode, which allows 
  any kind of use (e.g. performing tuning operations.)
  
  In a system with multiple front-ends, it is usually the case that multiple
  devices cannot be open in read/write mode simultaneously.  As long as a 
  front-end device is opened in read/write mode, other open() calls in 
  read/write mode will either fail or block, depending on whether 
  non-blocking or blocking mode was specified.
  A front-end device opened in blocking mode can later be put into non-blocking
  mode (and vice versa) using the F\_SETFL command of the fcntl system call.
  This is a standard system call, documented in the Linux manual page for fcntl.
  When an open() call has succeeded, the device will be ready for use in the 
  specified mode. This implies that the corresponding hardware is powered up, 
  and that other front-ends may have been powered down to make that possible.
  
  }{
  const char *deviceName & Name of specific video device.\\
  int flags & A bit-wise OR of the following flags:\\
            & \hspace{1em} O\_RDONLY read-only access\\
            & \hspace{1em} O\_RDWR read/write access\\
            & \hspace{1em} O\_NONBLOCK open in non-blocking mode \\
            & \hspace{1em} (blocking mode is the default)\\
  }{
  ENODEV    & Device driver not loaded/available.\\
  EINTERNAL & Internal error.\\
  EBUSY     & Device or resource busy.\\
  EINVAL    & Invalid argument.\\
}

\function{close()}{
  int close(int fd);}{
  This system call closes a previously opened front-end device.  
  After closing a front-end device, its corresponding hardware might be
  powered down automatically.
  }{
  int fd & File descriptor returned by a previous call to open().\\
  }{
  EBADF & fd is not a valid open file descriptor.\\
}

\ifunction{FE\_READ\_STATUS}{
  int ioctl(int fd, int request = FE\_READ\_STATUS, fe\_status\_t *status);}{
  This ioctl call returns status information about the front-end.
  This call only requires read-only access to the device.
  }{
  int fd      & File descriptor returned by a previous call to open().\\
  int request & Equals FE\_READ\_STATUS for this command.\\
  struct fe\_status\_t *status & Points to the location where the front-end
  status word is to be stored.
  }{
  EBADF& fd is not a valid open file descriptor.\\
  EFAULT& status points to invalid address.\\
}

\ifunction{FE\_READ\_BER}{
  int ioctl(int fd, int request = FE\_READ\_BER, uint32\_t *ber);}{
  This ioctl call returns the bit error rate for the signal currently 
  received/demodulated by the front-end. For this command, read-only access 
  to the device is sufficient.
  }{
  int fd      & File descriptor returned by a previous call to open().\\
  int request & Equals FE\_READ\_BER for this command.\\
  uint32\_t *ber & The bit error rate is stored into *ber.\\
  }{
  EBADF& fd is not a valid open file descriptor.\\
  EFAULT& ber points to invalid address.\\
  ENOSIGNAL& There is no signal, thus no meaningful bit error
             rate.  Also returned if the front-end is not turned on.\\
  ENOSYS&         Function not available for this device.
}

\ifunction{FE\_READ\_SNR}{
  int ioctl(int fd, int request = FE\_READ\_SNR, int16\_t *snr);}{
  This ioctl call returns the signal-to-noise ratio for the signal currently 
  received by the front-end. For this command, read-only access to the device
  is sufficient.
  }{
  int fd      & File descriptor returned by a previous call to open().\\
  int request & Equals FE\_READ\_SNR for this command.\\
  int16\_t *snr& The signal-to-noise ratio is stored into *snr.\\
}{
  EBADF& fd is not a valid open file descriptor.\\
  EFAULT& snr points to invalid address.\\
  ENOSIGNAL&        There is no signal, thus no meaningful signal
  strength value.  Also returned if front-end is not  turned on.\\
  ENOSYS&         Function not available for this device.
}

\ifunction{FE\_READ\_SIGNAL\_STRENGTH}{
  int ioctl( int fd, int request = FE\_READ\_SIGNAL\_STRENGTH, int16\_t *strength);
}{
This ioctl call returns the signal strength value for the signal currently 
received by the front-end. For this command, read-only access to the device 
is sufficient.
}{
int fd      & File descriptor returned by a previous call to open().\\
int request & Equals FE\_READ\_SIGNAL\_STRENGTH for this command.\\
int16\_t *strength & The signal strength value is stored into *strength.\\
}{
  EBADF& fd is not a valid open file descriptor.\\
  EFAULT& status points to invalid address.\\
  ENOSIGNAL&        There is no signal, thus no meaningful signal
  strength value.  Also returned if front-end is not  turned on.\\
  ENOSYS&         Function not available for this device.
}

\ifunction{FE\_READ\_UNCORRECTED\_BLOCKS}{
  int ioctl( int fd, int request = FE\_READ\_UNCORRECTED\_BLOCKS, uint32\_t *ublocks);  }{
  This ioctl call returns the number of uncorrected blocks detected by 
  the device driver during its lifetime.
  For meaningful measurements, the increment in 
  block count during a specific time interval should be calculated. 
  For this command, read-only access to the device is sufficient.\\
  Note that the counter will wrap to zero after its maximum count has 
  been reached.
}{
int fd      & File descriptor returned by a previous call to open().\\
int request & Equals FE\_READ\_UNCORRECTED\_BLOCKS for this command.\\
uint32\_t *ublocks & The total number of uncorrected blocks seen
by the driver so far.
}{
  EBADF& fd is not a valid open file descriptor.\\
  EFAULT& ublocks points to invalid address.\\
  ENOSYS&         Function not available for this device.
}


\ifunction{FE\_SET\_FRONTEND}{
  int ioctl(int fd, int request = FE\_SET\_FRONTEND, struct dvb\_frontend\_parameters *p);}{
        This ioctl call starts a tuning operation using specified parameters.  
        The result of this call will be successful if the parameters were valid and 
        the tuning could be initiated.
        The result of the tuning operation in itself, however, will arrive 
        asynchronously as an event (see documentation for FE\_GET\_EVENT 
        and FrontendEvent.)
        If a new FE\_SET\_FRONTEND operation is initiated before the previous
        one was completed,
        the previous operation will be aborted in favor of the new one.
        This command requires read/write access to the device.
  }{
  int fd & File descriptor returned by a previous call to open().\\
  int request & Equals FE\_SET\_FRONTEND for this command.\\
  struct dvb\_frontend\_parameters *p& Points to parameters for tuning operation.\\
  }{
  EBADF& fd is not a valid open file descriptor.\\
  EFAULT& p points to invalid address.\\
  EINVAL& Maximum supported symbol rate reached.\\
}

\ifunction{FE\_GET\_FRONTEND}{
  int ioctl(int fd, int request = FE\_GET\_FRONTEND, struct dvb\_frontend\_parameters *p);}{
  This ioctl call queries the currently effective frontend parameters.  
  For this command, read-only access to the device is sufficient.
  }{
  int fd & File descriptor returned by a previous call to open().\\
  int request & Equals FE\_SET\_FRONTEND for this command.\\
  struct dvb\_frontend\_parameters *p& Points to parameters for tuning operation.\\
  }{
  EBADF& fd is not a valid open file descriptor.\\
  EFAULT& p points to invalid address.\\
  EINVAL& Maximum supported symbol rate reached.\\
}

\ifunction{FE\_GET\_EVENT}{
  int ioctl(int fd, int request = QPSK\_GET\_EVENT, struct dvb\_frontend\_event *ev);}{
  This ioctl call returns a frontend event if available. If an event
  is not available, the behavior depends on whether the device is in blocking 
  or non-blocking mode.  In the latter case, the call fails immediately with 
  errno set to EWOULDBLOCK. In the former case, the call blocks until an event
  becomes available.\\
  The standard Linux poll() and/or select() system calls can be used with the 
  device file descriptor to watch for new events.  For select(), the file 
  descriptor should be included in the exceptfds argument, and for poll(), 
  POLLPRI should be specified as the wake-up condition.
  Since the event queue allocated is rather small (room for 8 events), the queue
  must be serviced regularly to avoid overflow.   If an overflow happens, the 
  oldest event is discarded from the queue, and an error (EOVERFLOW) occurs 
  the next time the queue is read. After reporting the error condition in this 
  fashion, subsequent FE\_GET\_EVENT calls will return events from the queue as
  usual.\\
  For the sake of implementation simplicity, this command requires read/write 
  access to the device.
  }{
  int fd & File descriptor returned by a previous call to open().\\
  int request & Equals FE\_GET\_EVENT for this command.\\
  struct dvb\_frontend\_event *ev & Points to the location where the event,\\
                                  & if any, is to be stored.
  }{
  EBADF& fd is not a valid open file descriptor.\\
  EFAULT& ev points to invalid address.\\
  EWOULDBLOCK &              There is no event pending, and the device is in
                                non-blocking mode.\\
  EOVERFLOW &\\
&   Overflow in event queue - one or more events were lost.\\
}

\ifunction{FE\_GET\_INFO}{
\label{fegetinfo}
  int ioctl(int fd, int request = FE\_GET\_INFO, struct dvb\_frontend\_info *info);}{
  This ioctl call returns information about the front-end.
  This call only requires read-only access to the device.
  }{
  int fd & File descriptor returned by a previous call to open().\\
  int request & Equals FE\_GET\_INFO for this command.\\
  struct dvb\_frontend\_info *info & Points to the location where the front-end
  information is to be stored.
  }{
  EBADF& fd is not a valid open file descriptor.\\
  EFAULT& info points to invalid address.\\
}

\ifunction{FE\_DISEQC\_RESET\_OVERLOAD}{
  int ioctl(int fd, int request = FE\_DISEQC\_RESET\_OVERLOAD);}{
  If the bus has been automatically powered off due to power overload, this
  ioctl call restores the power to the bus. The call requires read/write 
  access to the device.
  This call has no effect if the device is manually powered off.
  Not all DVB adapters support this ioctl.
  }{
  int fd & File descriptor returned by a previous call to open().\\
  int request & Equals FE\_DISEQC\_RESET\_OVERLOAD for this command.\\
  }{
  EBADF      & fd is not a valid file descriptor.\\
  EPERM      & Permission denied (needs read/write access).\\
  EINTERNAL  & Internal error in the device driver.\\
}


\ifunction{FE\_DISEQC\_SEND\_MASTER\_CMD}{
int ioctl(int fd, int request = FE\_DISEQC\_SEND\_MASTER\_CMD, struct dvb\_diseqc\_master\_cmd *cmd);}{
  This ioctl call is used to send a a DiSEqC command.\\
  }{
  int fd & File descriptor returned by a previous call to open().\\
  int request & Equals FE\_DISEQC\_SEND\_MASTER\_CMD for this command.\\
  struct dvb\_diseqc\_master\_cmd *cmd & Pointer to the command to be transmitted.\\
  }{
  EBADF      & fd is not a valid file descriptor.\\
  EFAULT     & Seq points to an invalid address.\\
  EINVAL     & The data structure referred to by seq is invalid in some way.\\
  EPERM      & Permission denied (needs read/write access).\\
  EINTERNAL  & Internal error in the device driver.\\
}

\ifunction{FE\_DISEQC\_RECV\_SLAVE\_REPLY}{
int ioctl(int fd, int request = FE\_DISEQC\_RECV\_SLAVE\_REPLY, struct dvb\_diseqc\_slave\_reply *reply);}{
This ioctl call is used to receive reply to a DiSEqC 2.0 command.\\
  }{
  int fd & File descriptor returned by a previous call to open().\\
  int request & Equals FE\_DISEQC\_RECV\_SLAVE\_REPLY for this command.\\
  struct dvb\_diseqc\_slave\_reply *reply & Pointer to the command to be received.\\
  }{
  EBADF      & fd is not a valid file descriptor.\\
  EFAULT     & Seq points to an invalid address.\\
  EINVAL     & The data structure referred to by seq is invalid in some way.\\
  EPERM      & Permission denied (needs read/write access).\\
  EINTERNAL  & Internal error in the device driver.\\
}

\ifunction{FE\_DISEQC\_SEND\_BURST}{
int ioctl(int fd, int request = FE\_DISEQC\_SEND\_BURST, fe\_sec\_mini\_cmd\_t burst);}{
This ioctl call is used to send a 22KHz tone burst.\\
  }{
  int fd & File descriptor returned by a previous call to open().\\
  int request & Equals FE\_DISEQC\_SEND\_BURST for this command.\\
  fe\_sec\_mini\_cmd\_t burst & burst A or B.\\
  }{
  EBADF      & fd is not a valid file descriptor.\\
  EFAULT     & Seq points to an invalid address.\\
  EINVAL     & The data structure referred to by seq is invalid in some way.\\
  EPERM      & Permission denied (needs read/write access).\\
  EINTERNAL  & Internal error in the device driver.\\
}


\ifunction{FE\_SET\_TONE}{
int ioctl(int fd, int request = FE\_SET\_TONE, fe\_sec\_tone\_mode\_t tone);}{
This call is used to set the generation of the continuous 22kHz tone. 
This call requires read/write permissions.
}{
int fd & File descriptor returned by a previous call to open().\\
int request & Equals FE\_SET\_TONE for this command.\\
fe\_sec\_tone\_mode\_t tone & The requested tone generation mode (on/off).\\
}{
ENODEV    & Device driver not loaded/available.\\
EBUSY     & Device or resource busy.\\
EINVAL    & Invalid argument.\\
EPERM     & File not opened with read permissions.\\
EINTERNAL & Internal error in the device driver.\\
}


\ifunction{FE\_SET\_VOLTAGE}{
int ioctl(int fd, int request = FE\_SET\_VOLTAGE, fe\_sec\_voltage\_t voltage);}{
This call is used to set the bus voltage.
This call requires read/write permissions.
}{
int fd & File descriptor returned by a previous call to open().\\
int request & Equals FE\_SET\_VOLTAGE for this command.\\
fe\_sec\_voltage\_t voltage & The requested bus voltage.\\
}{
ENODEV    & Device driver not loaded/available.\\
EBUSY     & Device or resource busy.\\
EINVAL    & Invalid argument.\\
EPERM     & File not opened with read permissions.\\
EINTERNAL & Internal error in the device driver.\\
}

\ifunction{FE\_ENABLE\_HIGH\_LNB\_VOLTAGE}{
int ioctl(int fd, int request = FE\_ENABLE\_HIGH\_LNB\_VOLTAGE, int high);}{
If high != 0 enables slightly higher voltages instead of 13/18V
(to compensate for long cables).
This call requires read/write permissions.
Not all DVB adapters support this ioctl.
}{
int fd & File descriptor returned by a previous call to open().\\
int request & Equals FE\_SET\_VOLTAGE for this command.\\
int high & The requested bus voltage.\\
}{
ENODEV    & Device driver not loaded/available.\\
EBUSY     & Device or resource busy.\\
EINVAL    & Invalid argument.\\
EPERM     & File not opened with read permissions.\\
EINTERNAL & Internal error in the device driver.\\
}


%%% Local Variables: 
%%% mode: latex
%%% TeX-master: "dvbapi"
%%% End: 
