\devsec{DVB Demux Device}

The DVB demux device controls the filters of the DVB hardware/software.
It can be accessed through \texttt{/dev/adapter0/demux0}.
Data types and and ioctl definitions can be accessed by including
\texttt{linux/dvb/dmx.h} in your application.

\devsubsec{Demux Data Types}


\devsubsubsec{dmx\_output\_t}
\label{dmxoutput}

\begin{verbatim}
typedef enum
{
        DMX_OUT_DECODER,
        DMX_OUT_TAP,
        DMX_OUT_TS_TAP
} dmx_output_t;
\end{verbatim}

\noindent\texttt{DMX\_OUT\_TAP} delivers the stream output to the demux device
on which the ioctl is called.

\noindent\texttt{DMX\_OUT\_TS\_TAP} routes output to the logical DVR device
\texttt{/dev/dvb/adapter0/dvr0}, which delivers a TS multiplexed from
all filters for which \texttt{DMX\_OUT\_TS\_TAP} was specified.


\devsubsubsec{dmx\_input\_t}
\label{dmxinput}

\begin{verbatim}
typedef enum
{
        DMX_IN_FRONTEND,
        DMX_IN_DVR
} dmx_input_t;
\end{verbatim}


\devsubsubsec{dmx\_pes\_type\_t}
\label{dmxpestype}

\begin{verbatim}
typedef enum
{
        DMX_PES_AUDIO,
        DMX_PES_VIDEO,
        DMX_PES_TELETEXT,
        DMX_PES_SUBTITLE,
        DMX_PES_PCR,
        DMX_PES_OTHER
} dmx_pes_type_t;
\end{verbatim}


\devsubsubsec{dmx\_event\_t}
\label{dmxeventt}

\begin{verbatim}
typedef enum
{
        DMX_SCRAMBLING_EV,
        DMX_FRONTEND_EV
} dmx_event_t;
\end{verbatim}


\devsubsubsec{dmx\_scrambling\_status\_t}
\label{dmxscramblingstatus}

\begin{verbatim}
typedef enum
{
        DMX_SCRAMBLING_OFF,
        DMX_SCRAMBLING_ON
} dmx_scrambling_status_t;
\end{verbatim}


\devsubsubsec{struct dmx\_filter}
\label{dmxfilter}

\begin{verbatim}
typedef struct dmx_filter
{
        uint8_t         filter[DMX_FILTER_SIZE];
        uint8_t         mask[DMX_FILTER_SIZE];
} dmx_filter_t;
\end{verbatim}


\devsubsubsec{struct dmx\_sct\_filter\_params}
\label{dmxsctfilterparams}

\begin{verbatim}
struct dmx_sct_filter_params
{
        uint16_t            pid;
        dmx_filter_t        filter;
        uint32_t            timeout;
        uint32_t            flags;
#define DMX_CHECK_CRC       1
#define DMX_ONESHOT         2
#define DMX_IMMEDIATE_START 4
};
\end{verbatim}


\devsubsubsec{struct dmx\_pes\_filter\_params}
\label{dmxpesfilterparams}

\begin{verbatim}
struct dmx_pes_filter_params
{
        uint16_t            pid;
        dmx_input_t         input;
        dmx_output_t        output;
        dmx_pes_type_t      pes_type;
        uint32_t            flags;
};
\end{verbatim}


\devsubsubsec{struct dmx\_event}
\label{dmxevent}

\begin{verbatim}
struct dmx_event
{
        dmx_event_t          event;
        time_t               timeStamp;
        union
        {
                dmx_scrambling_status_t scrambling;
        } u;
};
\end{verbatim}

\devsubsubsec{struct dmx\_stc}
\label{dmxstc}

\begin{verbatim}
struct dmx_stc {
        unsigned int num;       /* input : which STC? 0..N */
        unsigned int base;      /* output: divisor for stc to get 90 kHz clock */
        uint64_t stc;           /* output: stc in 'base'*90 kHz units */
};
\end{verbatim}

\clearpage

\devsubsec{Demux Function Calls}
\function{open()}{
  int open(const char *deviceName, int flags);}{
  This system call, used with a device name of /dev/dvb/adapter0/demux0,
  allocates a new filter 
  and returns a handle which can be used for subsequent control of that 
  filter. This call has to be made for each filter to be used, i.e. every
  returned file descriptor is a reference to a single filter.
  /dev/dvb/adapter0/dvr0 is a logical device to be used for retrieving Transport 
  Streams for digital video recording. When reading from  
  this device a transport stream containing the packets from all PES
  filters set in the corresponding demux device (/dev/dvb/adapter0/demux0) 
  having the output set to DMX\_OUT\_TS\_TAP.  A recorded Transport Stream 
  is replayed by writing to this device.
%  This device can only be opened in read-write mode.

  The significance of blocking or non-blocking mode is described in the
  documentation for functions where there is a difference. It does not 
  affect the semantics of the open() call itself. A device opened in 
  blocking mode can later be put into non-blocking mode
  (and vice versa) using the F\_SETFL command of the fcntl system call.
  }{
  const char *deviceName & Name of demux device.\\
  int flags & A bit-wise OR of the following flags:\\
            & \hspace{1em} O\_RDWR read/write access\\
            & \hspace{1em} O\_NONBLOCK open in non-blocking mode \\
            & \hspace{1em} (blocking mode is the default)\\
  }{
  ENODEV    & Device driver not loaded/available.\\
  EINVAL    & Invalid argument.\\
  EMFILE    & ``Too many open files'', i.e. no more filters available.\\
  ENOMEM    & The driver failed to allocate enough memory.\\
}

\function{close()}{
  int close(int fd);}{
  This system call deactivates and deallocates a filter that was previously
  allocated via the open() call.
  }{
  int fd & File descriptor returned by a previous call to open().\\
  }{
  EBADF & fd is not a valid open file descriptor.\\
}

\function{read()}{
  size\_t read(int fd, void *buf, size\_t count);
  }{
  This system call returns filtered data, which might be section or PES 
  data. The filtered data is transferred from the driver's internal circular
  buffer to buf. The maximum amount of data to be transferred is implied by
  count.\\
  When returning section data the driver always tries to return a complete 
  single section (even though buf would provide buffer space for more data).
  If the size of the buffer is smaller than the section as much as possible
  will be returned, and the remaining data will be provided in subsequent 
  calls.\\
  The size of the internal buffer is 2 * 4096 bytes (the size of two maximum
  sized sections) by default. The size of this buffer may be changed by 
  using the DMX\_SET\_BUFFER\_SIZE function. If the buffer is not large enough,
  or if the read operations are not performed fast enough, this may result 
  in a buffer overflow error. In this case EOVERFLOW will be returned,
  and the circular buffer will be emptied.
  This call is blocking if there is no data to return, i.e. the process 
  will be put to sleep waiting for data, unless the O\_NONBLOCK flag is 
  specified.\\
  Note that in order to be able to read, the filtering process has to be
  started by defining either a section or a PES filter by means of the 
  ioctl functions, and then starting the filtering process via the DMX\_START
  ioctl function or by setting the DMX\_IMMEDIATE\_START flag.
  If the reading is done from a logical DVR demux device, the data will 
  constitute a Transport Stream including the packets from all PES filters
  in the corresponding demux device /dev/dvb/adapter0/demux0 having the output set 
  to DMX\_OUT\_TS\_TAP.
  }{
  int fd        & File descriptor returned by a previous call to open().\\
  void *buf     & Pointer to the buffer to be used for returned filtered data.\\
  size\_t count  & Size of buf.\\
  }{
  EWOULDBLOCK   & No data to return and O\_NONBLOCK was specified.\\
  EBADF         & fd is not a valid open file descriptor.\\
  ECRC          & Last section had a CRC error - no data returned. 
                  The buffer is flushed.\\
  EOVERFLOW & \\
& The filtered data was not read from the buffer in
                    due time, resulting in non-read data being lost.
                    The buffer is flushed.\\
  ETIMEDOUT       & The section was not loaded within the stated
                    timeout period. See ioctl DMX\_SET\_FILTER for
                    how to set a timeout.\\
  EFAULT          & The driver failed to write to the callers buffer
                    due to an invalid *buf pointer.\\
}

\function{write()}{
  ssize\_t write(int fd, const void *buf, size\_t count);
  }{
  This system call is only provided by the logical device /dev/dvb/adapter0/dvr0, 
  associated with the physical demux device that provides the actual 
  DVR  functionality.  It is used for replay of a digitally recorded 
  Transport Stream. Matching filters have to be defined in the 
  corresponding physical demux device, /dev/dvb/adapter0/demux0.
  The amount of data to be transferred is implied by count.
  }{
  int fd        & File descriptor returned by a previous call to open().\\
  void *buf     & Pointer to the buffer containing the Transport Stream.\\
  size\_t count & Size of buf.\\
  }{
  EWOULDBLOCK   & No data was written. This might happen if
                  O\_NONBLOCK was specified and there is no more
                  buffer space available (if O\_NONBLOCK is not
                  specified the function will block until buffer 
                  space is available).\\
  EBUSY         & This error code indicates that there are
                  conflicting requests. The corresponding demux
                  device is setup to receive data from the front-
                  end. Make sure that these filters are stopped
                  and that the filters with input set to DMX\_IN\_DVR
                  are started.\\
  EBADF         & fd is not a valid open file descriptor.\\
}

\ifunction{DMX\_START}{
  int ioctl( int fd, int request = DMX\_START);
  }{
  This ioctl call is used to start the actual filtering operation 
  defined via the ioctl calls DMX\_SET\_FILTER or DMX\_SET\_PES\_FILTER.
  }{
  int fd & File descriptor returned by a previous call to open().\\
  int request & Equals DMX\_START for this command.\\
  }{
  EBADF  & fd is not a valid file descriptor.\\
  EINVAL & Invalid argument, i.e. no filtering parameters
           provided via the DMX\_SET\_FILTER or
           DMX\_SET\_PES\_FILTER functions.\\
  EBUSY  & This error code indicates that there are
           conflicting requests. There are active filters
           filtering data from another input source. Make
           sure that these filters are stopped before starting
           this filter.\\
}

\ifunction{DMX\_STOP}{
  int ioctl( int fd, int request = DMX\_STOP);
  }{
  This  ioctl call is used to stop the actual filtering operation defined 
  via the ioctl  calls DMX\_SET\_FILTER or DMX\_SET\_PES\_FILTER and started via
  the DMX\_START command.
  }{
  int fd & File descriptor returned by a previous call to open().\\
  int request & Equals DMX\_STOP for this command.\\
  }{
  EBADF  & fd is not a valid file descriptor.\\
}

\ifunction{DMX\_SET\_FILTER}{
  int ioctl( int fd, int request = DMX\_SET\_FILTER, struct dmx\_sct\_filter\_params *params);
  }{
  This ioctl call sets up a filter according to the filter and mask 
  parameters provided. A timeout may be defined stating number of seconds 
  to wait for a section to be loaded. A value of 0 means that no timeout 
  should be applied. Finally there is a flag field where it is possible to
  state whether a section should be CRC-checked, whether the filter should
  be a "one-shot" filter, i.e. if the filtering operation should be stopped
  after the first section is received, and whether the filtering operation 
  should be started immediately (without waiting for a DMX\_START ioctl call).
  If a filter was previously set-up, this filter will be canceled, and the
  receive buffer will be flushed.
  }{
  int fd & File descriptor returned by a previous call to open().\\
  int request & Equals DMX\_SET\_FILTER for this command.\\
  struct dmx\_sct\_filter\_params *params
  & Pointer to structure containing filter parameters.\\
  }{
  EBADF  & fd is not a valid file descriptor.\\
  EINVAL & Invalid argument.\\
}

\ifunction{DMX\_SET\_PES\_FILTER}{
  int ioctl( int fd, int request = DMX\_SET\_PES\_FILTER, 
  struct dmx\_pes\_filter\_params *params);
  }{
  This ioctl call sets up a PES filter according to the parameters provided.
  By a PES filter is meant a filter that is based just on the packet 
  identifier (PID), i.e. no PES header or payload filtering capability is 
  supported.\\
  The transport stream destination for the filtered output may be set. Also
  the PES type may be stated in order to be able to e.g. direct a video 
  stream directly to the video decoder. Finally there is a flag field where
  it is possible to state whether the filtering operation should be started
  immediately (without waiting for a DMX\_START ioctl call).
  If a filter was previously set-up, this filter will be cancelled, and the
  receive buffer will be flushed.
  }{
  int fd & File descriptor returned by a previous call to open().\\
  int request & Equals DMX\_SET\_PES\_FILTER for this command.\\
  struct dmx\_pes\_filter\_params *params
  & Pointer to structure containing filter parameters.\\
  }{
  EBADF  & fd is not a valid file descriptor.\\
  EINVAL & Invalid argument.\\
  EBUSY  & This error code indicates that there are
           conflicting requests. There are active filters
           filtering data from another input source. Make
           sure that these filters are stopped before starting
           this filter.\\
}

\ifunction{DMX\_SET\_BUFFER\_SIZE}{
  int ioctl( int fd, int request = DMX\_SET\_BUFFER\_SIZE, unsigned long size);
  }{
  This ioctl call is used to set the size of the circular buffer used
  for filtered data. The default size is two maximum sized sections, i.e. 
  if this function is not called a buffer size of 2 * 4096 bytes will be
  used.
  }{
  int fd & File descriptor returned by a previous call to open().\\
  int request & Equals DMX\_SET\_BUFFER\_SIZE for this command.\\
  unsigned long size & Size of circular buffer.\\
  }{
  EBADF  & fd is not a valid file descriptor.\\
  ENOMEM & The driver was not able to allocate a buffer of the requested size.\\
}

\ifunction{DMX\_GET\_EVENT}{
  int ioctl( int fd, int request = DMX\_GET\_EVENT, struct dmx\_event *ev);
  }{
  This ioctl call returns an event if available. If an event is not 
  available, the behavior depends on whether the device is in blocking or 
  non-blocking mode.  In the latter case, the call fails immediately with
  errno set to EWOULDBLOCK. In the former case, the call blocks until an
  event becomes available.\\
  The standard Linux poll() and/or select() system calls can be used with
  the device file descriptor to watch for new events.  For select(), the 
  file descriptor should be included in the exceptfds argument, and for 
  poll(), POLLPRI should be specified as the wake-up condition.
  Only the latest event for each filter is saved.
  }{
  int fd & File descriptor returned by a previous call to open().\\
  int request & Equals DMX\_GET\_EVENT for this command.\\
  struct dmx\_event *ev & Pointer to the location where the event is to be stored.\\
  }{
  EBADF  & fd is not a valid file descriptor.\\
  EFAULT & ev points to an invalid address.\\
  EWOULDBLOCK & There is no event pending, and the device is in non-blocking mode.\\
}

\ifunction{DMX\_GET\_STC}{
  int ioctl( int fd, int request = DMX\_GET\_STC, struct dmx\_stc *stc);
  }{
  This ioctl call returns the current value of the system time counter
  (which is driven by a PES filter of type DMX\_PES\_PCR). Some hardware
  supports more than one STC, so you must specify which one by setting
  the num field of stc before the ioctl (range 0...n). The result is returned in form
  of a ratio with a 64 bit numerator and a 32 bit denominator, so the
  real 90kHz STC value is \begin{ttfamily}stc->stc / stc->base\end{ttfamily}.
  }{
  int fd & File descriptor returned by a previous call to open().\\
  int request & Equals DMX\_GET\_STC for this command.\\
  struct dmx\_stc *stc & Pointer to the location where the stc is to be stored.\\
  }{
  EBADF  & fd is not a valid file descriptor.\\
  EFAULT & stc points to an invalid address.\\
  EINVAL & Invalid stc number.\\
}

%%% Local Variables: 
%%% mode: latex
%%% TeX-master: "dvbapi"
%%% End: 
