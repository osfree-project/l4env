\documentclass{article}
\usepackage[a4paper]{geometry}
\usepackage[dvips]{epsfig}
\usepackage{times}
\usepackage{listings}

% formating stuff
\emergencystretch 10mm
\clubpenalty 10000
\widowpenalty 10000

% some macros

% display eps figure
% arg 1 ... eps file
% arg 2 ... minipage width
% arg 3 ... caption 
% label -> fig:<eps file>
\newcommand{\epsfigure}[3]
{
  \begin{figure}[ht]
    \centering
    \begin{minipage}{#2}
      \epsfig{file=#1.eps,width=\columnwidth}
      \caption{#3}
      \label{fig:#1}
    \end{minipage}
  \end{figure}
}

% customization of listings
\lstset{
  language=C,
  basicstyle=\footnotesize,
  keywordstyle=\tt\bf,
  nonkeywordstyle=\tt,
  labelstyle=\tiny,
  labelstep=10}

\title{DROPS Block Device Driver Interface Specification\\
--- DRAFT ---}
\author{Lars Reuther}
\date{February 2001 \\ \small $Revision$}

\begin{document}
\maketitle

\section*{Change History}

\begin{description}
\item[2000-02-01]
  Created. Author: Lars Reuther
\end{description}

\clearpage
\tableofcontents
\clearpage

\part{Project Info}

\section{Change Control}

\begin{description}
\item[Current owner:] Lars Reuther
\item[Change Control Board:] 
\item[User components:] Ext2fs-Server
\item[Dependencies:]
\end{description}

\section{Risks}

\begin{enumerate}
\item This document specifies a preliminary version of the DROPS Block 
  Device Driver Interface. It might change once we have the next version of 
  the DROPS Streaming Interface (DSI), then the Block Device Driver Interface 
  might use the DSI instead of the RPC protocol to send requests to a device
  driver.
\item The Quality of Service management in DROPS is not yet finally 
  specified. Once it is specified, the Block Device Driver Interface might
  need to be changed to use the correct type for the quality argument of 
  real-time streams.
\item A naming scheme is still missing in DROPS. Drivers use the standard 
  DROPS nameserver to register itself, different devices of a driver (e.g. 
  different disks) are specified using the minor number scheme of Linux.
\item Flick cannot create asynchronous interfaces directly. The driver RPC 
  interface is defined using two different Flick interfaces to emulate 
  an asynchronous interface.
\item Advanced driver functionality like \emph{ioctls} are not yet 
  considered in this specification.
\end{enumerate}

\clearpage
\part{DROPS Block Device Driver Interface Specification} 

\section{Introduction}

The DROPS Block Device Driver Interface specifies how block device drivers
like a SCSI or IDE disk driver provide their functionality to DROPS 
components like a real-time filesystem. Fig. \ref{fig:drops} shows such an  
application scenario in DROPS. 

\epsfigure{drops}{7.6cm}{Application Scenario}

\section{Block Request Scheduling}

To provide the context of real-time disk requests, this section briefly 
describes the request scheduling algorithm used in block device drivers.

\emph{to be done...}

\subsection{Filesystem Metadata}

To create the request list for a file, the filesystem must access metadata
like inodes and blocklists of the file. This metadata is typically also stored 
on the disk, thus the accesses must be considered in the request scheduling. 
Because of the structure of the metadata all requests to metadata blocks must 
be mandatory. A disk block containing a part of a blocklist can hold up to 
several thousand block numbers, thus the loss of a single blocklist block 
would affect a huge part of the data stream and therefore is not acceptable.

To avoid an overreservation of disk bandwidth for metadata requests, the
requests are spread across a large time interval. By that, we only need 
to reserve a few requests in a request period (idealy just one) for 
metadata requests.

\epsfigure{metadata}{8cm}{Metadata request scheduling}

All metadata requests must be available at the beginning of the metadata 
request period, and are assigned to one of the reserved requests slots in 
the request periods. The number of required requests slots depends on
\begin{itemize}
\item the total number of metadata requests
\item the length of the metadata request period. Because all requests must 
  be available at the beginning of the period and the order of the requests
  within the period is not fixed, the length of the period determines the 
  request latency for the filesystem. The period length must be set to 
  a value which is acceptable for the filesystem, e.g. dependent on the 
  buffer requirements.
\item the number of streams. Each stream needs at least one request per 
  period.
\end{itemize}

\emph{to do: example calculation}

\section{Block Device Driver Interface}

\subsection{Block Device Abstraction}

The Block Device Driver Interface uses the same block device abstraction 
like the Linux kernel:
\begin{itemize}
\item the block size is 1kByte
\item logical block addressing, the blocks of a device are addressed 
  with numbers $<0, number of blocks - 1>$
\item partitions of a device are specified with a number scheme similar to 
  linux, the upper 4 bit of an 8 bit device id specify the device 
  (e.g. a harddisk), the lower 4 bit the partition number, partition number
  0 specifies the whole device.
\end{itemize}

\subsubsection{Real-time Requests}

\subsection{Driver RPC Interface}

Device drivers implement an asynchronous interface. Clients send requests
to the driver using the \emph{put\_requests} RPC call, this call immediately
returns after the requests are enqueued to the driver's request list. If one the 
requests is processed, the driver sends a notification which includes the 
status of the request (success / error) to the client.

\epsfigure{rpc}{5.5cm}{Block Device Driver RPC protocol}

Fig.~\ref{fig:rpc} shows the principle mechanism of the interface. Before 
a client can send requests to a driver, the driver must be opened. Explicitly  
open the driver enables the driver to allocate resources necessary to 
serve the new client, e.g. to create new instances of the service threads.
The \emph{open} function returns a handle which describes the new instance of 
the driver, among other things this handle  contains the ids of the threads 
which can be used by the client to send the requests to the driver and to wait 
for the processed notifications. After the driver is opened, the client can
send requests to the driver. A single \emph{put\_request}-call can contain 
several requests, but the driver sends for each request a separate processed 
notification. The connection to the driver must be closed explicitly with a 
\emph{close}-call. 

\subsubsection{RPC functions}

The RPC protocol contains following function:
\begin{description}
\item[\emph{open}] Open a new instance of the driver, the function returns a 
  handle which describes the new instance.
\item[close] Close driver instance.
\item[\emph{create\_stream}] Create a new real-time stream. The arguments 
  are the bandwidth, blocksize and quality parameter of the stream. If the 
  driver can serve this stream, it returns a handle for the stream which must 
  be used to mark the requests of that stream in the request list of 
  \emph{put\_request}.
\item[\emph{close\_stream}] Close a real-time stream.
\item[\emph{put\_requests}] Send a request list to the driver. Each request 
  contains a handle wich is used by the driver to identify the request in the
  precessed notification message.
\item[\emph{wait}] Wait for a processed notification. The function returns the 
  request handle specified in \emph{put\_requests} and a status code.
\end{description}
See Section~\ref{app:rpc} for the exact definition of the RPC protocol 
(IDL specification).

\subsubsection{Implementation Remarks}
\label{sec:rpc_remarks}

Because Flick cannot handle asynchronous interfaces, the driver RPC protocol
must be implemented using several threads to emulate the asynchronous
interface. Fig.~\ref{fig:threads} shows how the interface functions are mapped
to the different threads.

\epsfigure{threads}{5cm}{Interface Threads}

The main thread of the driver is used to open and close instances of the 
driver, its id must be registered at the DROPS nameserver to enable clients 
to find the driver. The driver's command and notification threads are used 
to do the request processing, a driver instance can create its own versions 
of those threads or share existing threads, it depends on the requirements of 
the driver. Clients need a separate wait thread, it calls the wait thread of 
the driver to wait for processed notifications.

\subsection{Application Interface}

The previous section defined the RPC interface which a block device driver 
must implement. To ease the use of the interface clients do not use the 
RPC stubs functions directly, instead the RPC client stubs are encapsulated 
by a client library and this library provides an API to the client. 
The main reasons to build a sparate client library are:
\begin{itemize}
\item hide the complexity of the RPC interface, especially thread structure 
  described in Section~\ref{sec:rpc_remarks} and
\item add functionality, e.g. a synchronous request handling.
\end{itemize}
The client API contains following functions:
\begin{description}
\item[\emph{int blk\_open\_driver(const char * name, blk\_driver\_t * driver)}]
  Open a new driver instance. \emph{name} is the name of the driver which must
  be registered at the nameserver, the function returns the handle of the 
  driver instance. The implementation requests the the id of the main thread of 
  the driver at the nameserver and calls the \emph{open} RPC functions.
\item[\emph{int blk\_close\_driver(blk\_driver\_t driver)}] 
  Close a driver instance.
\item[\emph{blk\_stream\_t create\_stream(
    blk\_driver\_t driver, unsigned long bandwidth, 
    unsigned long blk\_size,unsigned int q)}]
 Create a new real-time stream.
\item[\emph{int blk\_close\_stream(blk\_driver\_t driver, 
    blk\_stream\_t stream)}]
  Close real-time stream.
\item[\emph{int blk\_put\_requests(blk\_request\_t requests[],
    int num\_requests)}]
  Send requests to the driver. Besides the usual arguments (driver handle, 
  device id, block number, number of blocks, buffer address) the request 
  descriptor can contain a stream handle and a request number. The stream
  handle must be set to identify real-time requests, the request number
  specifies the position of the request in a real-time stream and is 
  used by the driver to calculate the request period in which the request 
  must be processed.
\item[\emph{int blk\_get\_status(blk\_request\_t * request)}]
  Return the status of the request. The status can be:
  \begin{itemize}
  \item \emph{unprocessed} request not yet processed by the driver
  \item \emph{done} request successfully finished by the driver
  \item \emph{error} error processing the request. One possible reason of an 
    error is that there was not time left to process a real-time request.
  \end{itemize}
  \emph{blk\_get\_status} can be called at any time, thus it is possible
  to poll the status of requests if the synchronization through the 
  semaphore is not appropriate.
\item[\emph{int blk\_do\_request(blk\_driver\_t driver, blk\_stream\_t stream,
    blk\_request\_t * request)}]
  Execute request synchronously. \emph{blk\_do\_request} blocks until the
  processed notification is received from the driver. The return value 
  indicates the request status (success/error). 
\end{description}
See Section~\ref{app:api} for the exact definition of the API (C header file).

A drawback of the current asynchronous request handling is that through the 
semaphore it is only signaled that a request is finished, but not which 
request. The client must check which of its outstanding requests was finished
using \emph{blk\_get\_status}. A different approach would be to specify a 
callback function which is called if the processed notification is received and
which gets the request handle as argument. The problem is that the callback
function is executed by the wait thread of the client, it is not clear if this
causes any problems.

\clearpage
\part{Appendix}

\section{Device Driver RPC Interface}
\label{app:rpc}

\section{User API}
\label{app:api}

\end{document}

