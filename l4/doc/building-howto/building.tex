\documentclass[twocolumn,letter,10pt]{article}
\usepackage[latin1]{inputenc}
\usepackage{html}
%begin{latexonly}
\usepackage[a4paper,dvips]{geometry}
\usepackage{fancyvrb}
%end{latexonly}

\tolerance = 1000
\emergencystretch = 10pt
\clubpenalty = 10000
\widowpenalty = 10000 \displaywidowpenalty = 10000
\setlength{\marginparwidth}{1cm}

\title{Building DROPS HOWTO}
\author{Frank Mehnert \qquad Jork L�ser \qquad Ronald Aigner\\l4-hackers@os.inf.tu-dresden.de }
\date{March 2006}

\newcommand{\llinux}{L$\!^4$Linux}
\newcommand{\code}[1]{\texttt{#1}}

\usepackage{hyperref}

\begin{document}
\maketitle

This document is also available in
\latexhtml{\htmladdnormallink{HTML}{index.html}}
{\htmladdnormallink{postscript}{building.ps} and 
\htmladdnormallink{pdf}{building.pdf}} format.

\section{Quickstart}

Although we recommend reading the few pages of this paper, here is the
abstract:

\begin{enumerate}

\item Get the following components from \code{http://tudos.{\hskip
      0pt}org/{\hskip 0pt}drops/{\hskip 0pt}download.html}:
  L4Env, \llinux, Fiasco.

\item Unpack the tar-balls in one directory. Memorize the directory.

\item Switch into \code{l4/} and enter

  \code{DROPS\_STDDIR=<dir>/drops $\backslash$\\}
  \code{make oldconfig}
  
  Substitute \code{<dir>} with the absolute pathname of the directory
  you kept in mind. Still in \code{l4/}, enter

  \code{make}

\end{enumerate}


\section{Involved Components}

DROPS is a collection of packages using services of internal or
external libraries and tools.

A DROPS package typically consists of a server, one or more client
libraries and associated interface headers. Often there also exist one
or more examples illustrating the usage of the package.

For C library support, DROPS utilizes one of two external C libraries dietlibc
or uClibc. The two C libraries are included in the L4Env as packages.
Some packages require the older OSKit libc. These packages are not actively
maintained.

Some DROPS packages require \llinux, the L4 port of Linux. \llinux{} is
available as another external module.

The internal tools include the BID makefile macros, the IDL compiler
dice, a dependency generation tool, a code transformation tool, a
configuration tool and others.

To build a DROPS package, all interface headers and libraries the
packet depends on must be available, as well as the C library and the
tools mentioned. For examples, the name server has the following
dependencies:
\begin{itemize}
\item the resource manager \code{roottask} (part of DROPS)
\item the logging server \code{log} (part of DROPS)
\item to build it requires the packages \code{l4util}, \code{l4sys},
\code{roottask}, \code{sigma0}, \code{log}, \code{events}, and \code{crtx}
(all part of DROPS)
\item a C library (either dietlibc or uClibc)
\item the BID make macros, and the tools \code{config}, \code{gendep},
\code{dice}, and the \code{bin} directory from \code{l4/tool}.
\end{itemize}

The building tools of DROPS allow for two locations of the required
packets (rmgr, log\dots), C libraries and tools, depending on how you
installed DROPS:

\begin{enumerate}
\item All packets and tools are available in source code. Their
   directories are specified relative to the \code{L4DIR}. Please see
   table \ref{tab:dirs} for the locations of the various components.
   
   By executing \code{make} in \code{L4DIR/pkg}, all libraries are
   symlinked into subdirectories of \code{L4DIR/lib} and all
   interface headers and IDL files are symlinked into subdirectories
   at \code{L4DIR/include}. We also refer to \code{L4DIR} as \textit{L4
     directory}. All binary files are symlinked into subdirectories
   at \code{L4DIR/bin}.

\item The packets and tools are already compiled and installed at a central
   place at \code{DROPS\_STDDIR}. Obviously, only the headers and IDL
   files of the packets are available in source code. We also refer to
   \code{DROPS\_STDDIR} as \textit{DROPS directory}.
\end{enumerate}

\begin{table}[h]
\center{
\begin{tabular}{|l|l|}
\hline
component               & default location     \\
\hline
DROPS packets           & \code{L4DIR/pkg}         \\
DROPS Tools             & \code{L4DIR/tool}        \\
\llinux{} v2.2          & \code{L4DIR/../linux22}  \\
\llinux{} v2.6		& \code{L4DIR/../l4linux-2.6} \\
\hline
\end{tabular}
\caption{Locations of source code relative to \code{L4DIR}}
\label{tab:dirs}
}
\end{table}

The first method allows to inspect and to change the source code, e.g.
for development or for debugging. The second method allows to use
pre-compiled libraries
from a central place without the need to download and compile the whole DROPS
source tree. For instance, the students at the Dresden OS group can easily
access the \textit{DROPS directory} at \code{/home/drops} which is
rebuild from the newest CVS version each night.

You can mix both methods of installation, i.e. some packets can be
installed into the \textit{DROPS directory} and some other packets can be
compiled from sources within the \textit{L4 directory}. If you have both
versions of one packet, the one in \code{L4DIR/pkg} will be used when
compiling other packets.


In the following section we show how to build a DROPS tree.



\section{How to build DROPS}


\subsection{Downloading and Unpacking}

The source code of the DROPS packages and the tools can be obtained from
\code{http://tudos.{\hskip 0pt}org/{\hskip 0pt}drops/{\hskip
0pt}download.html} as tar-balls or via remote-CVS -- see download instructions
at this web page.  Put all the tar-balls into one directory and unpack them.

If you are at the Dresden OS group, you can use the CVS repository at
\code{/home/cvs}. You can also use the installed components at
\code{/home/drops}. Check out the components you want to compile on your own.
Additionally, you need the BID macros at \code{L4DIR/mk}, the Makefile at
\code{L4DIR}. If you want to compile DROPS packages, you should also check out
the Makefile at \code{L4DIR/pkg}. Before building a package you need to config
your environment and target platform (see Section~\ref{sec:config}).  For
this, you need the \code{config} and \code{gendep} tools from the
\code{L4DIR/tool} directory.  Most packages use IDL to specify their
interfaces.  To compile the IDLs you need our IDL compiler \code{dice} also in
the \code{L4DIR/tool} directory. We set up some CVS modules, please see the
aforementioned download page for details.

\subsection{OS requirements}

The following external compilers and tools which are not part of the
DROPS distribution are required:

\begin{table}[h]
\center{
\begin{tabular}{|l|l|}
\hline
tool                    & supported versions    \\
\hline
GNU C                   & 3.x \\
GNU make                & $\geq$ 3.77           \\
Binutils                & $\geq$ 2.13.90        \\
GNU bash                & $\geq$ 2.05           \\
GNU find                &                       \\
awk                     & posix compatible      \\
GNU sed                 &                       \\
GNU perl                & $\geq$ 5.6.1          \\
flex                    &                       \\
byacc / bison           &                       \\
Doxygen (recommended)   & $\geq$ 1.2.15		\\
autoconf                & $\geq$ 2.59		\\
automake		& $\geq$ 1.4		\\
ncurses-dev(el)		&			\\
python-curses		&			\\
\hline
\end{tabular}
\caption{Required build tools}
\label{tab:tools}
}
\end{table}

We have tested DROPS with various versions of the GNU tools, but
cannot give a guarantee that it works with the versions installed on
your system. If you encounter a problem and cannot use the versions
mentioned in table \ref{tab:tools}, contact us and we will try to
solve the issue.

\subsection{Configuring}
\label{sec:config}

Prior to compilation, you must configure the DROPS environment. Go
into your \code{L4DIR} directory. If you uncompressed the tar-balls,
this should be \code{l4/}. A 

\code{l4/> make config}

starts an interactive configuration tool. You should at least adapt
the \code{DROPS\_STDDIR} setting at ``Paths and Directories'', as this
is the directory where the components will be installed on a
\code{make install}.

The default values meet most peoples requirements. Thus, instead of
doing the interactive configuration, you can also configure
non-interactively. Set the environment variable \code{DROPS\_STDDIR}
and use the make-target \emph{oldconfig}:

\code{l4/> DROPS\_STDDIR=yourdir make oldconfig}

If you are at Dresden OS group, a ``\code{make oldconfig}'' should do it.


\subsection{Compilation}

\subsubsection{The Easy way}

After configuring, go to the \code{L4DIR} directory and do a

\code{l4/> make}

This builds everything available as sources: the tools, C libraries,
Fiasco micro-kernel and the DROPS packages. It also builds the documentation.

To install the things into \code{DROPS\_STDDIR} (and obtain the same
setup as we have at Dresden), do a

\code{l4/> make install}

\subsubsection{Doing it separately}

If you want to restart the build process, e.g. after editing some
source files, you do not have to build everything again. Here is what
the Makefile in \code{L4DIR} does.

\begin{enumerate}
\item DROPS tools in \code{L4DIR/tool}
\begin{verbatim}
cd l4/tool ; make
\end{verbatim}
\item \llinux{} v2.2 (if available)
\begin{verbatim}
cd linux22 ; make -f Makefile.drops
\end{verbatim}
\item Fiasco micro-kernel
\begin{verbatim}
cd l4/kernel ; make
\end{verbatim}
\item DROPS packages
\begin{verbatim}
cd l4/pkg ; make
\end{verbatim}
\item DROPS package and tool documentation
\begin{verbatim}
cd l4/pkg ; make doc
cd l4/doc/html ; make doc
cd l4/tool; make doc
\end{verbatim}
\end{enumerate}

Generally, you can build the DROPS packages and tools separately,
i.e., you can switch into the directory of a specific DROPS package or
tool and issue the \code{make} command there. See Section \ref{sec:submakes}
for details.

Note, although the makefile in \code{L4DIR} builds the documentation
per default, the called sub-makefiles only build the documentation
when \code{doc} is explicitly mentioned as a target on the cmdline.

\subsubsection{Sub-makefiles and Sub-tree consistency}
\label{sec:submakes}

\textsl{The DROPS build system allows to bring a subtree in a properly
  built state by issuing \code{make} at the top of that subtree. \code{Make}
  generally not builds things upwards in the directory hierarchy.}

This is true for everything \emph{below} \code{L4DIR}. An exception is
\llinux, which has its own build system. \code{L4DIR}, serving as the project
root directory, is the other exception, as it integrates the external
components.

As a consequence, a \code{make} issued within a package X cannot help you
building other packages X may depend on. Or, if you are within the
\code{server/} directory of a package, \code{make} does not rebuild the
\code{lib/} directory of that package. In these cases, you have to
issue the \code{make} upwards the directory hierarchy.

As another consequence, you can be sure that \code{make} does not change
your file-system outside the current subtree. Exceptions are the
installation directories \code{L4DIR/bin}, \code{L4DIR/lib},
\code{L4DIR/include} and \code{L4DIR/doc}.

\subsubsection{Dependencies}

The DROPS build system automatically detects source code dependencies
by intercepting the file system calls during compilation. This turned
out the be both a robust and general solution.

It is generally not necessary to enforce rebuilding of DROPS source
code, e.g. by doing a \code{make clean} followed by a \code{make}. A
\code{make} rebuilds the depending files reliably, be it within one DROPS
package, across DROPS packages or code of external components such as
the C library.



\end{document}

% Localwords: CVS unset API APIs IDL gcc libc OSKit OSKits Dresden OS
% Local variables:
%  compile-command: "make"
% End:
