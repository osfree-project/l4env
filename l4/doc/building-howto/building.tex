\documentclass[twocolumn,10pt]{article}
\usepackage[a4paper,dvips]{geometry}
\usepackage{fancyvrb}

\title{Building DROPS HOWTO}
\author{Frank Mehnert}
\date{}

\begin{document}
\maketitle


\section{The \texttt{drops-gcc} Compiler}
The best way to build libraries or applications for DROPS is to use
the \texttt{drops-gcc} compiler. This wrapper script for gcc/g++
automaticly uses the proper header files and links against several
standard libraries.

To work correct, the \texttt{drops-gcc} compiler itself has to be
installed (\texttt{make} is insufficient, \texttt{make install} is
needed) and depends at least on an installed version of the OSKit 1.0
(either as a local or global -- see next chapter for details)


\subsection{Local Installation}
In the global installation, the \texttt{drops-gcc} compiler itself
and appendant libraries are installed in subdirectories of 
\texttt{/home/drops}. For local installations (e.g. on Laptops),
these files can be installed into an \textbf{user specified}
directory (e.g. \texttt{\$HOME/src/drops}). In the following text,
this directory is called the \textsl{local DROPS directory}.

\begin{description}
\item[\textsl{local DROPS directory}]
  Create that directory (e.g. \$HOME/src/drops):
\begin{verbatim}
  mkdir $HOME/src/drops
\end{verbatim}
  Edit the global config script for the L4 build tree 
  (e.g. \texttt{\$HOME/src/l4/Makeconf.local}) and set
  the \texttt{DROPS\_STDDIR} variable to point to that
  directory.

\item[OSKit 0.6 libraries]
Check out, compile and install the OSKit Version 1.0.
\begin{verbatim}
  cd $HOME/src
  cvs checkout oskit
  cd oskit
  ./configure --prefix=$HOME/src/drops
  make
  make install
\end{verbatim}

\item[OSKit 1.0 libraries]
Check out, compile and install the OSKit Version 1.0.
\begin{verbatim}
  cd $HOME/src
  cvs checkout oskit10
  cd oskit10
  ./configure --prefix=$HOME/src/drops
  make
  make install
\end{verbatim}

After installation, the checked out version of the OSKit 1.0 can
be removed.

\item[The \texttt{drops-gcc} compiler]
Checkout the \texttt{drops-gcc} compiler:
\begin{verbatim}
  cd $HOME/src
  cvs checkout l4/tool/gcc-wrap
\end{verbatim}
There are two individual settings for the local installation in
the \texttt{drops-gcc.in} script but there is \textbf{no need to change}
something here.

\begin{itemize}
\item \texttt{DROPSDIR} points to the \textsl{local DROPS directory}.
  The default setting has the value of make's \texttt{\$DROPS\_STDDIR}
  variable.
\item \texttt{L4DIR} points to the directory the L4 tree resides in 
  (e.g. \$HOME/src/l4). If this variable is not set, \textbf{all} 
  libraries the application should linked against must reside in the 
  \textsl{local DROPS directory}. In default, this setting has the
  value of make's \texttt{\$L4DIR} variable.
\end{itemize}

If you want to change one of these values then you have to edit the 
\texttt{drops-gcc.in} script. Again: The default settings should be
suitable in most cases.

Compile and install the compiler.
\begin{verbatim}
  cd $HOME/src
  cvs checkout l4/tool/gcc-wrap
  cd l4/tool/gcc-wrap
  make
  make install
\end{verbatim}

\item[Flick]
Checkout and compile Flick.
\begin{verbatim}
  cd $HOME/src
  cvs checkout l4/tool/flick
  cvs checkout l4/tool/flick-obj
  cd l4/tool/flick-obj
  make
\end{verbatim}

Optional, install Flick by
\begin{verbatim}
  make install
\end{verbatim}
With this command, Flick will be installed into your 
\textsl{local DROPS directory}.

\item[Common L4 Environment]
Checkout and compile the Common L4 Environment. You can do it
using
\begin{verbatim}
  cd $HOME/src/l4/pkg
  make l4env_update
  make l4env
\end{verbatim}

or checking out the following packages by hand:

L4 packages are needed: \texttt{l4sys, l4util, names, log, rmgr,
l4env, lock, thread, semaphore, l4rm, generic\_dm, generic\_ts,
generic\_fprov, oskit10\_support\_l4env\_full}.
\begin{verbatim}
  cd $HOME/src
  cvs checkout l4/pkg/<pkg_name>
  cd l4/pkg
  make pkg-links
  make lib
  make
\end{verbatim}

\item[Standard C++ library]
Checkout, compile and install the \texttt{libstdc++} library:
\begin{verbatim}
  cd $HOME/src
  cvs checkout l4/pkg/libstdc++
  cd l4/pkg/libstdc++
  ln -s ../pkg.Makefile.tmpl .
  make
  make install
\end{verbatim}

After installing, this package can be removed.

\item[Other L4 packages]
To compile other L4 packages, it is necessary to install the
L$^{4}$Linux sources:
\begin{verbatim}
  cd $HOME/src
  cvs checkout linux22
  cd linux22
  make oldconfig
  make dep
\end{verbatim}

\end{description}

\subsection{Usage}
The \texttt{dropsgcc} compiler has several build mode.

\begin{description}
\item[oskit10\_sigma0]
  When using this mode, the application will be linked against OSKit 1.0
  libraries and the \texttt{oskit10\_support} library making her a 
  \textsl{Sigma~0} client. The OSKit's FreeBSD C library is used.
  Dependant libraries:
  \begin{description}
  \item[OSKit]
  \texttt{liboskit\_startup, liboskit\_clientos, liboskit\_bootp,
  liboskit\_linux\_fs, liboskit\_diskpart, liboskit\_linux\_dev, 
  liboskit\_freebsd\_net, liboskit\_kern, liboskit\_lmm, liboskit\_amm, 
  liboskit\_freebsd\_c, liboskit\_com}
  \item[other]
  \texttt{librmgr, libgcc, liboskit10\_support}
  \end{description}

\item[oskit10\_l4env]
  The application is linked against OSKit 1.0 libraries and the Common L4
  Environment together with the \texttt{oskit10\_support\_l4env\_full}
  library. This is the default mode of \texttt{drops-gcc}. The OSKit's
  FreeBSD C library is used. Dependant libraries:
  \begin{description}
  \item[OSKit]
  \texttt{liboskit\_startup, liboskit\_clientos, liboskit\_bootp,
  liboskit\_linux\_fs, liboskit\_diskpart, liboskit\_linux\_dev,
  liboskit\_freebsd\_net, liboskit\_kern, liboskit\_lmm, liboskit\_amm,
  liboskit\_freebsd\_c, liboskit\_com}
  \item[Common L4 Environment]
  \texttt{libthread, libsemaphore, libl4rm, libgeneric\_dm, liblogserver,
  libnames, libl4env, libl4util}
  \item[other]
  \texttt{librmgr, libgcc, liboskit10\_support\_l4env\_full}
  \end{description}

\item[oskit10\_l4env\_tiny]
  The application is linked against the OSKit 1.0 and the Common L4
  Environment together with the \texttt{oskit10\_support\_l4env} library.
  Use this mode for simple servers (without C++ and special OSKit features,
  e.g. memory file system).
  There are many functions in this mode which don't work (e.g. 
  \texttt{sleep()}). Tested functions are memory management
  (\texttt{malloc()}/\texttt{free()} and console input/output
  (\texttt{printf()}, \texttt{getchar()}). Therefore it is recommended
  to use the \texttt{oskit10\_l4env} mode instead. The OSKit's tiny C
  library is used.
  Dependant libraries:
  \begin{description}
  \item[OSKit]
  \texttt{liboskit\_clientos, liboskit\_kern, liboskit\_lmm, liboskit\_c}
  \item[Common L4 Environment]
  \texttt{libthread, libsemaphore, libl4rm, libgeneric\_dm, liblogserver,
  libnames, liblenv, libl4util}
  \item[other]
  \texttt{librmgr, liboskit10\_support\_l4env}
  \end{description}

\end{description}

To specifiy the mode the project should built with, \textbf{either}

\begin{itemize}
\item add \texttt{-}\textsl{$<$mode\_name$>$} as command line parameter
  of \texttt{drops-gcc} \textbf{or}
\item export the \texttt{DROPS\_BUILD\_MODE} envionment variable
  from the Makefile.
\end{itemize}

To simplify this task, a new Makefile template
(\texttt{Makeconf.oskit10\_lenv}) was created.
The template does also define a \texttt{DROPS\_LIB}
variable which includes the libraries the application is linked
against for using as dependency.


\section{The \texttt{oskit10\_support\_l4env\_full} library}
The \texttt{oskit10\_support\_l4env\_full} library is the glue code
which allows applications to use code of the OSKit 1.0 together with
services of the Common L4 Environment.


\subsection{Global Heap Management}
The malloc()/free() functions of the OSKit access a memory chunk
located at a dataspace provided by a dataspace manager. The default
size of the memory chunk is 4MB (defined in file 
\textsl{base\_multiboot\_init\_mem.c}). At boot time, the size can be
specified using the command line parameter
\texttt{--heap}=\textsl{$<$size in bytes$>$}.

The memory allocator is protected by a simple locking mechanism.
For this, a semaphore of the \texttt{semaphore} library is used
(file \textsl{lock.c}).


\subsection{Accessing Physical Memory}
Some applications need access to physical memory regions. Until now,
such requests are sent to RMGR.

\begin{description}
\item[0x00000000-0x00001000] The \textsl{first page} holds the BIOS
  data area.
\item[0x000A0000-0x000FFFFF] The \textsl{adapter area} holds the BIOS
  ROM area.
\item[0x80000000-0xFFFFFFFF] The \textsl{device area} can be accessed
  in memory chunks of 4MB.
\item[TODO] Unmapping of physical memory?
\end{description}

\subsection{Accessing the multiboot info}
There are two global symbols defined for accessing the multiboot info
which is provided by the bootloader:

\begin{description}
\item[boot\_info] holds the multiboot info as provided by the
  old grub bootloader.

\item[grub\_boot\_info] holds the extended multiboot info as
  provided by newer (hacked) versions of GRUB. The size of the
  structure depends on \texttt{grub\_boot\_info.flags}. One important
  feature of newer GRUB version is the VESA video mode information.
  For an example how to use that information, see the \texttt{vesaview}
  server of the \texttt{smart\_mpeg package}.
\end{description}


\subsection{Helpers}

Some helper functions of the OSKit 1.0 are overwritten:
\begin{description}
\item[exit] 
  The exit function prints the exit code of the application. If
  the exit code was not equal zero, the user will be prompted for
  an key. Pressing s sets the task into sleep, pressing k enters
  the L4 kernel debugger and Return reboots immediatly.
\item[pc\_reset]
  After showing an appropriate message, the code jumps into the kernel
  debugger. On continue, the machine is rebootet.
\end{description}

\section{Example}
For an example how to use the \texttt{drops-gcc} compiler, see the
tftp package. Especially notice the Makefile.


\section{Changelog}
\begin{description}
\item[22.01.2001] initial release
\item[25.01.2001] first review by Jork
\item[26.01.2001] new targets for checking out and building the
  Common L4 Environment
\item[29.01.2001] added dependant libraries for each mode of 
  \texttt{drops-gcc}
\item[17.04.2001] changed drops-gcc.in so that it doesn't need to
  be patched anymore -- use \texttt{DROPS\_STDDIR} instead
\item[01.02.2002] added dependency to OSKit 0.6 and L4Linux


\end{description}


\end{document}

