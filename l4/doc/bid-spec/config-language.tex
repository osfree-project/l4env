

\subsection{Introduction}

Config Language is not 'bash'.

This document describes Config Language, the Linux Kernel Configuration
Language.  config.in and Config.in files are written in this language.

Although it looks, and usually acts, like a subset of the 'sh' language,
Config Language has a restricted syntax and different semantics.

Here is a basic guideline for Config Language programming: use only the
programming idioms that you see in existing Config.in files.  People often
draw on their shell programming experience to invent idioms that look
reasonable to shell programmers, but silently fail in Config Language.

Config Language is not 'bash'.



\subsection{Interpreters}

Four different configuration programs read Config Language:

\begin{tabular}{ll}
    scripts/Configure  & make config, make oldconfig\\
    scripts/Menuconfig & make menuconfig\\
    scripts/tkparse    & make xconfig\\
    mconfig            & ftp.kernel.org/pub/linux/kernel/people/hch/mconfig/\\
\end{tabular}

\textbf{'Configure'} is a bash script which interprets Config.in files by sourcing
them.  Some of the Config Language commands are native bash commands;
simple bash functions implement the rest of the commands.

\textbf{'Menuconfig'} is another bash script.  It scans the input files with a
small awk script, builds a shell function for each menu, sources the
shell functions that it builds, and then executes the shell functions
in a user-driven order.  Menuconfig uses 'lxdialog', a back-end utility
program, to perform actual screen output.  'lxdialog' is a C program
which uses curses.

\textbf{'scripts/tkparse'} is a C program with an ad hoc parser which translates
a Config Language script to a huge TCL/TK program.  'make xconfig'
then hands this TCL/TK program to 'wish', which executes it.

\textbf{'mconfig'} is the next generation of Config Language interpreters.  It is a
C program with a bison parser which translates a Config Language script
into an internal syntax tree and then hands the syntax tree to one of
several user-interface front ends.



\subsection{Statements}

A Config Language script is a list of statements.  There are 21 simple
statements; an 'if' statement; menu blocks; and a 'source' statement.

A '\verb"\"' at the end of a line marks a line continuation.

'\#' usually introduces a comment, which continues to the end of the line.
Lines of the form '\# \dots is not set', however, are not comments.  They
are semantically meaningful, and all four config interpreters implement
this meaning.

Newlines are significant.  You may not substitute semicolons for newlines.
The 'if' statement does accept a semicolon in one position; you may use
a newline in that position instead.

Here are the basic grammar elements.
\begin{itemize}

\item  A \textit{/prompt/} is a single-quoted string or a double-quoted string.
    If the word is double-quoted, it may not have any \$ substitutions.

\item A \textit{/word/} is a single unquoted word, a single-quoted string, or a
    double-quoted string.  If the word is unquoted or double quoted,
    then \$-substitution will be performed on the word.

\item A \textit{/symbol/} is a single unquoted word.  A symbol must have a name of
    the form \texttt{CONFIG\_*}.  scripts/mkdep.c relies on this convention in order
    to generate dependencies on individual \texttt{CONFIG\_*} symbols instead of
    making one massive dependency on include/linux/autoconf.h.

\item A \textit{/dep/} is a dependency.  Syntactically, it is a \textit{/word/}.
    At run
    time, a \textit{/dep/} must evaluate to "y", "m", "n", or "".

\item An \textit{/expr/} is a bash-like expression using the operators
    '=', '!=', '-a', '-o', and '!'.

\end{itemize}

Here are all the statements:

\begin{itemize}

\item Text statements:

\begin{tabular}{l}
 \texttt{mainmenu\_name}\quad\textit{/prompt/}\\
 \texttt{comment}\quad\textit{/prompt/}\\
 \texttt{text}\quad\textit{/prompt/}\\
\end{tabular}

\item  Ask statements:

\begin{tabular}{l}
 \texttt{bool}\quad\textit{/prompt/}\quad\textit{/symbol/}\\
 \texttt{hex}\quad\textit{/prompt/}\quad\textit{/symbol/}\quad\textit{/word/}\\
 \texttt{int}\quad\textit{/prompt/}\quad\textit{/symbol/}\quad\textit{/word/}\\
 \texttt{string}\quad\textit{/prompt/}\quad\textit{/symbol/}\quad\textit{/word/}\\
 \texttt{tristate}\quad\textit{/prompt/}\quad\textit{/symbol/}\\
\end{tabular}

\item  Define statements:

\begin{tabular}{l}
 \texttt{define\_bool}     \quad\textit{/symbol/} \quad\textit{/word/}\\
 \texttt{define\_hex}      \quad\textit{/symbol/} \quad\textit{/word/}\\
 \texttt{define\_int}      \quad\textit{/symbol/} \quad\textit{/word/}\\
 \texttt{define\_string}   \quad\textit{/symbol/} \quad\textit{/word/}\\
 \texttt{define\_tristate}  \quad\textit{/symbol/} \quad\textit{/word/}\\
\end{tabular}

\item  Dependent statements:

\begin{tabular}{l}
 \texttt{dep\_bool}    \quad\textit{/prompt/} \quad\textit{/symbol/} \quad\textit{/dep/} \dots\\
 \texttt{dep\_mbool}   \quad\textit{/prompt/} \quad\textit{/symbol/} \quad\textit{/dep/} \dots\\
 \texttt{dep\_hex}     \quad\textit{/prompt/} \quad\textit{/symbol/} \quad\textit{/word/} \quad\textit{/dep/} \quad\dots\\
 \texttt{dep\_int}     \quad\textit{/prompt/} \quad\textit{/symbol/} \quad\textit{/word/} \quad\textit{/dep/} \quad\dots\\
 \texttt{dep\_string}  \quad\textit{/prompt/} \quad\textit{/symbol/} \quad\textit{/word/} \quad\textit{/dep/} \quad\dots\\
 \texttt{dep\_tristate} \quad\textit{/prompt/} \quad\textit{/symbol/} \quad\textit{/dep/} \quad\dots\\
\end{tabular}

\item Unset statement:

\begin{tabular}{l}
 \texttt{unset} \quad\textit{/symbol/}\quad\dots\\
\end{tabular}

\item  Choice statements:

\begin{tabular}{l}
  \texttt{choice}          \quad\textit{/prompt/} \quad\textit{/word/} \quad\textit{/word/}\\
  \texttt{nchoice}         \quad\textit{/prompt/} \quad\textit{/symbol/} \quad\textit{/prompt/} \quad\textit{/symbol/}\quad\dots\\
\end{tabular}

\item If statements:

\begin{tabular}{l}
 \texttt{if}\quad [ \textit{/expr/} ] \quad;\quad \texttt{then}\\
          \qquad\textit{/statement/}\\
          \qquad\dots\\
        \texttt{fi}\\

 \texttt{if} \quad[ \textit{/expr/} ] \quad;\quad then\\
          \qquad\textit{/statement/}\\
          \qquad\dots\\
        \texttt{else}\\
          \qquad\textit{/statement/}\\
          \qquad\dots\\
        \texttt{fi}\\
\end{tabular}

\item Menu block:

\begin{tabular}{l}
   \texttt{mainmenu\_option} \texttt{next\_comment}\\
   \texttt{comment}\quad \textit{/prompt/}\\
         \qquad\textit{/statement/}\\
         \qquad\dots\\
        \texttt{endmenu}\\
\end{tabular}

\item Source statement:

\begin{tabular}{l}
 \texttt{source} \textit{/word/}
\end{tabular}

\end{itemize}



\subsubsection{\texttt{mainmenu\_name}\quad \textit{/prompt/}}

This verb is a lot less important than it looks.  It specifies the top-level
name of this Config Language file.

\begin{tabular}{ll}
Configure:  &ignores this line\\
Menuconfig: &ignores this line\\
Xconfig:    &uses \textit{/prompt/} for the label window.\\
mconfig:    &ignores this line (mconfig does a better job without it).\\
\end{tabular}

Example:
{\small\begin{verbatim}
  # arch/sparc/config.in
  mainmenu_name "Linux/SPARC Kernel Configuration"
\end{verbatim}}


\subsubsection{\texttt{comment} \quad\textit{/prompt/}}

This verb displays its prompt to the user during the configuration process
and also echoes it to the output files during output.  Note that the
prompt, like all prompts, is a quoted string with no dollar substitution.

The \texttt{comment} verb is not a Config Language comment.  It causes the
user interface to display text, and it causes output to appear in the
output files.

\begin{tabular}{ll}
Configure:  &implemented\\
Menuconfig: &implemented\\
Xconfig:    &implemented\\
mconfig:    &implemented\\
\end{tabular}

Example:
{\small\begin{verbatim}
  # drivers/net/Config.in
  comment 'CCP compressors for PPP are only built as modules.'
\end{verbatim}}


\subsubsection{\texttt{text} \quad\textit{/prompt/}}

This verb displays the prompt to the user with no adornment whatsoever.
It does not echo the prompt to the output file.  mconfig uses this verb
internally for its help facility.

\begin{tabular}{ll}
Configure:  &not implemented\\
Menuconfig: &not implemented\\
Xconfig:    &not implemented\\
mconfig:    &implemented\\
\end{tabular}

Example:
{\small\begin{verbatim}
  # mconfig internal help text
  text 'Here are all the mconfig command line options.'
\end{verbatim}}



\subsubsection{\texttt{bool}\quad \textit{/prompt/}\quad\textit{/symbol/}}

This verb displays \textit{/prompt/} to the user, accepts a value from the user,
and assigns that value to \textit{/symbol/}.  The legal input values are "n" and
"y".

Note that the \texttt{bool} verb does not have a default value.  People keep
trying to write Config Language scripts with a default value for \texttt{bool},
but \textbf{all} of the existing language interpreters discard additional values.
Feel free to submit a multi-interpreter patch to linux-kbuild if you
want to implement this as an enhancement.

\begin{tabular}{ll}
Configure:  &implemented\\
Menuconfig: &implemented\\
Xconfig:    &implemented\\
mconfig:    &implemented\\
\end{tabular}

Example:
{\small\begin{verbatim}
  # arch/i386/config.in
  bool 'Symmetric multi-processing support' CONFIG_SMP
\end{verbatim}}



\subsubsection{\texttt{hex}\quad\textit{/prompt/}\quad\textit{/symbol/}\quad\textit{/word/}}

This verb displays \textit{/prompt/} to the user, accepts a value from the user,
and assigns that value to \textit{/symbol/}.  Any hexadecimal number is a legal
input value.  \textit{/word/} is the default value.

The \texttt{hex} verb does not accept range parameters.

\begin{tabular}{ll}
Configure:  &implemented\\
Menuconfig: &implemented\\
Xconfig:    &implemented\\
mconfig:    &implemented\\
\end{tabular}

Example:
{\small\begin{verbatim}
  # drivers/sound/Config.in
  hex 'I/O base for SB Check from manual of the card' \
    CONFIG_SB_BASE 220
\end{verbatim}}


\subsubsection{\texttt{int}\quad \textit{/prompt/}\quad \textit{/symbol/}\quad \textit{/word/}}

This verb displays \textit{/prompt/} to the user, accepts a value from the user,
and assigns that value to \textit{/symbol/.  /word/} is the default value.
Any decimal number is a legal input value.

The \texttt{int} verb does not accept range parameters.

\begin{tabular}{ll}
Configure:  &implemented\\
Menuconfig: &implemented\\
Xconfig:    &implemented\\
mconfig:    &implemented\\
\end{tabular}

Example:
{\small\begin{verbatim}
  # drivers/char/Config.in
  int 'Maximum number of Unix98 PTYs in use (0-2048)' \
        CONFIG_UNIX98_PTY_COUNT 256
\end{verbatim}}


\subsubsection{\texttt{string}\quad \textit{/prompt/}\quad \textit{/symbol/}\quad \textit{/word/}}

This verb displays \textit{/prompt/} to the user, accepts a value from the user,
and assigns that value to \textit{/symbol/.  /word/} is the default value.  Legal
input values are any ASCII string, except for the characters '"' and '\verb"\"'.
Configure will trap an input string of "?" to display help.

The default value is mandatory.

\begin{tabular}{ll}
Configure:  &implemented\\
Menuconfig: &implemented\\
Xconfig:    &implemented\\
mconfig:    &implemented\\
\end{tabular}

Example:
{\small\begin{verbatim}
  # drivers/sound/Config.in
  string '  Full pathname of DSPxxx.LD firmware file' \
        CONFIG_PSS_BOOT_FILE /etc/sound/dsp001.ld
\end{verbatim}}



\subsubsection{\texttt{tristate}\quad \textit{/prompt/}\quad \textit{/symbol/}}

This verb displays \textit{/prompt/} to the user, accepts a value from the user,
and assigns that value to \textit{/symbol/}.  Legal values are "n", "m", or "y".

The value "m" stands for "module"; it indicates that \textit{/symbol/} should
be built as a kernel module.  The value "m" is legal only if the symbol
CONFIG\_MODULES currently has the value "y".

The \texttt{tristate} verb does not have a default value.

\begin{tabular}{ll}
Configure:  &implemented\\
Menuconfig: &implemented\\
Xconfig:    &implemented\\
mconfig:    &implemented\\
\end{tabular}

Example:
{\small\begin{verbatim}
  # fs/Config.in
  tristate 'NFS filesystem support' CONFIG_NFS_FS
\end{verbatim}}



\subsubsection{\texttt{define\_bool}\quad \textit{/symbol/}\quad \textit{/word/}}

This verb the value of \textit{/word/} to \textit{/symbol/}.  
Legal values are "n" or "y".

For compatibility reasons, the value of "m" is also legal, because it
will be a while before \texttt{define\_tristate} is implemented everywhere.

\begin{tabular}{ll}
Configure:  &implemented\\
Menuconfig: &implemented\\
Xconfig:    &implemented\\
mconfig:    &implemented\\
\end{tabular}

Example:
{\small\begin{verbatim}
  # arch/alpha/config.in
  if [ "$CONFIG_ALPHA_GENERIC" = "y" ]
  then
    define_bool CONFIG_PCI y
    define_bool CONFIG_ALPHA_NEED_ROUNDING_EMULATION y
  fi
\end{verbatim}}



\subsubsection{\texttt{define\_hex} \quad\textit{/symbol/}\quad \textit{/word/}}

This verb assigns the value of \textit{/word/} to \textit{/symbol/}.  Any hexadecimal
number is a legal value.

\begin{tabular}{ll}
Configure:  &implemented\\
Menuconfig: &implemented\\
Xconfig:    &implemented\\
mconfig:    &implemented\\
\end{tabular}

Example:
{\small\begin{verbatim}
  # Not from the corpus
  bool 'Specify custom serial port' CONFIG_SERIAL_PORT_CUSTOM
  if [ "$CONFIG_SERIAL_PORT_CUSTOM" = "y" ]; then
    hex 'Serial port number' CONFIG_SERIAL_PORT
  else
    define_hex CONFIG_SERIAL_PORT 0x3F8
  fi
\end{verbatim}}



\subsubsection{\texttt{define\_int} \quad\textit{/symbol/}\quad \textit{/word/}}

This verb assigns \textit{/symbol/} the value \textit{/word/}.  
Any decimal number is a legal value.

\begin{tabular}{ll}
Configure:  &implemented\\
Menuconfig: &implemented\\
Xconfig:    &implemented\\
mconfig:    &implemented\\
\end{tabular}

Example:
{\small\begin{verbatim}
  # drivers/char/ftape/Config.in
  define_int CONFIG_FT_ALPHA_CLOCK 0
\end{verbatim}}



\subsubsection{\texttt{define\_string}\quad \textit{/symbol/} \quad\textit{/word/}}

This verb assigns the value of \textit{/word/} to \textit{/symbol/}. 
Legal input values are any ASCII string, except for the characters '"' 
and '\verb"\"'.

\begin{tabular}{ll}
Configure:  &implemented\\
Menuconfig: &implemented\\
Xconfig:    &implemented\\
mconfig:    &implemented\\
\end{tabular}

Example:
{\small\begin{verbatim}
  # Not from the corpus
  define_string CONFIG_VERSION "2.2.0"
\end{verbatim}}



\subsubsection{\texttt{define\_tristate}\quad \textit{/symbol/} \quad\textit{/word/}}

This verb assigns the value of \textit{/word/} to \textit{/symbol/}.  
Legal input values are "n", "m", and "y".

As soon as this verb is implemented in all interpreters, please use it
instead of \texttt{define\_bool} to define tristate values.  This aids in static
type checking.

\begin{tabular}{ll}
Configure:  &implemented\\
Menuconfig: &implemented\\
Xconfig:    &implemented\\
mconfig:    &implemented\\
\end{tabular}

Example:
{\small\begin{verbatim}
  # drivers/video/Config.in
  if [ "$CONFIG_FB_AMIGA" = "y" ]; then
     define_tristate CONFIG_FBCON_AFB y
     define_tristate CONFIG_FBCON_ILBM y
    else
     if [ "$CONFIG_FB_AMIGA" = "m" ]; then
        define_tristate CONFIG_FBCON_AFB m
        define_tristate CONFIG_FBCON_ILBM m
     fi
  fi
\end{verbatim}}



\subsubsection{\texttt{dep\_bool} \quad\textit{/prompt/}\quad \textit{/symbol/}\quad \textit{/dep/} \quad\dots}

This verb evaluates all of the dependencies in the dependency list.
Any dependency which has a value of "y" does not restrict the input
range.  Any dependency which has an empty value is ignored.
Any dependency which has a value of "n", or which has some other value,
(like "m") restricts the input range to "n".  Quoting dependencies is not
allowed. Using dependencies with an empty value possible is not
recommended.  See also \texttt{dep\_mbool} below.

If the input range is restricted to the single choice "n", \texttt{dep\_bool}
silently assigns "n" to \textit{/symbol/}.  If the input range has more than
one choice, dep\_bool displays \textit{/prompt/} to the user, accepts a value
from the user, and assigns that value to \textit{/symbol/}.

\begin{tabular}{ll}
Configure:  &implemented\\
Menuconfig: &implemented\\
XConfig:    &implemented\\
mconfig:    &implemented\\
\end{tabular}

Example:
{\small\begin{verbatim}
  # drivers/net/Config.in
  dep_bool 'Aironet 4500/4800 PCI support 'CONFIG_AIRONET4500_PCI \
    $CONFIG_PCI
\end{verbatim}}

Known bugs:
\begin{itemize}
\item Xconfig does not write "\# foo is not set" to .config (as well as
  "\#undef foo" to autoconf.h) if command is disabled by its dependencies.
\end{itemize}


\subsubsection{\texttt{dep\_mbool} \quad\textit{/prompt/}\quad \textit{/symbol/}\quad \textit{/dep/} \quad\dots}

This verb evaluates all of the dependencies in the dependency list.
Any dependency which has a value of "y" or "m" does not restrict the
input range.  Any dependency which has an empty value is ignored.
Any dependency which has a value of "n", or which has some other value,
restricts the input range to "n".  Quoting dependencies is not allowed.
Using dependencies with an empty value possible is not recommended.

If the input range is restricted to the single choice "n", dep\_bool
silently assigns "n" to \textit{/symbol/}.  If the input range has more than
one choice, \texttt{dep\_bool} displays \textit{/prompt/} to the user, accepts a value
from the user, and assigns that value to \textit{/symbol/}.

Notice that the only difference between \texttt{dep\_bool} and \texttt{dep\_mbool}
is in the way of treating the "m" value as a dependency.

\begin{tabular}{ll}
Configure:  &implemented\\
Menuconfig: &implemented\\
XConfig:    &implemented\\
mconfig:    &implemented\\
\end{tabular}

Example:
{\small\begin{verbatim}
  # Not from the corpus
  dep_mbool 'Packet socket: mmapped IO' CONFIG_PACKET_MMAP \
    $CONFIG_PACKET
\end{verbatim}}

Known bugs:
\begin{itemize}
\item Xconfig does not write "\# foo is not set" to .config (as well as
  "\#undef foo" to autoconf.h) if command is disabled by its dependencies.
\end{itemize}


\subsubsection{\texttt{dep\_hex}\quad \textit{/prompt/}\quad \textit{/symbol/}\quad \textit{/word/}\quad \textit{/dep/}\quad \dots\newline
dep\_int\quad \textit{/prompt/} \quad\textit{/symbol/}\quad \textit{/word/}\quad \textit{/dep/}\quad \dots\newline
dep\_string \quad\textit{/prompt/}\quad \textit{/symbol/}\quad \textit{/word/}\quad \textit{/dep/}\quad \dots}

I am still thinking about the semantics of these verbs.

\begin{tabular}{ll}
Configure:  &not implemented\\
Menuconfig: &not implemented\\
XConfig:    &not implemented\\
mconfig:    &not implemented\\
\end{tabular}



\subsubsection{\texttt{dep\_tristate}\quad \textit{/prompt/} \quad\textit{/symbol/} \quad\textit{/dep/}\quad \dots}

This verb evaluates all of the dependencies in the dependency list.
Any dependency which has a value of "y" does not restrict the input range.
Any dependency which has a value of "m" restricts the input range to
"m" or "n".  Any dependency which has an empty value is ignored.
Any dependency which has a value of "n", or which has some other value,
restricts the input range to "n".  Quoting dependencies is not allowed.
Using dependencies with an empty value possible is not recommended.

If the input range is restricted to the single choice "n", dep\_tristate
silently assigns "n" to \textit{/symbol/}.  If the input range has more than
one choice, \texttt{dep\_tristate} displays \textit{/prompt/} to the user, accepts a value
from the user, and assigns that value to \textit{/symbol/}.

\begin{tabular}{ll}
Configure:  &implemented\\
Menuconfig: &implemented\\
Xconfig:    &implemented\\
mconfig:    &implemented\\
\end{tabular}

Example:
{\small\begin{verbatim}
  # drivers/char/Config.in
  dep_tristate 'Parallel printer support' CONFIG_PRINTER $CONFIG_PARPORT
\end{verbatim}}

Known bugs:
\begin{itemize}
\item Xconfig does not write "\# foo is not set" to .config (as well as
  "\#undef foo" to autoconf.h) if command is disabled by its dependencies.
\end{itemize}


\subsubsection{\texttt{unset}\quad \textit{/symbol/} \quad\dots}

This verb assigns the value "" to \textit{/symbol/}, but does not cause 
\textit{/symbol/}
to appear in the output.  The existence of this verb is a hack; it covers
up deeper problems with variable semantics in a random-execution language.

\begin{tabular}{ll}
Configure:  &implemented\\
Menuconfig: &implemented\\
Xconfig:    &implemented (with bugs)\\
mconfig:    &implemented\\
\end{tabular}

Example:
{\small\begin{verbatim}
  # arch/mips/config.in
  unset CONFIG_PCI
  unset CONFIG_MIPS_JAZZ
  unset CONFIG_VIDEO_G364
\end{verbatim}}



\subsubsection{\texttt{choice}\quad \textit{/prompt/}\quad\textit{/word/}\quad\textit{/word/}}

This verb implements a choice list or "radio button list" selection.
It displays \textit{/prompt/} to the user, as well as a group of sub-prompts
which have corresponding symbols.

When the user selects a value, the choice verb sets the corresponding
symbol to "y" and sets all the other symbols in the choice list to "n".

The second argument is a single-quoted or double-quoted word that
describes a series of sub-prompts and symbol names.  The interpreter
breaks up the word at white space boundaries into a list of sub-words.
The first sub-word is the first prompt; the second sub-word is the
first symbol.  The third sub-word is the second prompt; the fourth
sub-word is the second symbol.  And so on, for all the sub-words.

The third word is a literal word.  Its value must be a unique abbreviation
for exactly one of the prompts.  The symbol corresponding to this prompt
is the default enabled symbol.

Note that because of the syntax of the \texttt{choice} verb, the sub-prompts
may not have spaces in them.

\begin{tabular}{ll}
Configure:  &implemented\\
Menuconfig: &implemented\\
Xconfig:    &implemented\\
mconfig:    &implemented\\
\end{tabular}

Example:
{\small\begin{verbatim}
  # arch/i386/config.in
  choice '  PCI access mode'         \
    "BIOS      CONFIG_PCI_GOBIOS     \
     Direct    CONFIG_PCI_GODIRECT   \
     Any       CONFIG_PCI_GOANY"     Any
\end{verbatim}}



\subsubsection{nchoice\quad \textit{/prompt/}\quad\textit{/symbol/}\quad\textit{/prompt/}\quad\textit{/symbol/}\dots}

This verb has the same semantics as the choice verb, but with a sensible
syntax.

The first \textit{/prompt/} is the master prompt for the entire choice list.

The first \textit{/symbol/} is the default symbol to enable (notice that this
is a symbol, not a unique prompt abbreviation).

The subsequent \textit{/prompt/} and \textit{/symbol/} pairs are the prompts and symbols
for the choice list.

\begin{tabular}{ll}
Configure:  &not implemented\\
Menuconfig: &not implemented\\
XConfig:    &not implemented\\
mconfig:    &implemented\\
\end{tabular}



\subsubsection{\texttt{if}\quad [ \textit{/expr/} ]\quad; \quad\texttt{then}}

This is a conditional statement, with an optional \texttt{else} clause.  You may
substitute a newline for the semicolon if you choose.

\textit{/expr/} may contain the following atoms and operators.  Note that, 
unlike shell, you must use double quotes around every atom.

    \textit{/atom/}:
        "\dots"                 a literal
        "\$\dots"                       a variable

    \textit{/expr/}:
        \textit{/atom/  = /atom/}       true if atoms have identical value
        \textit{/atom/ != /atom/}       true if atoms have different value

    \textit{/expr/}:
        \textit{/expr/} -o \textit{/expr/}      true if either expression is true
        \textit{/expr/} -a \textit{/expr/}      true if both expressions are true
        ! \textit{/expr/}               true if expression is not true

Note that a naked \textit{/atom/} is not a valid \textit{/expr/}.  If you try to use it
as such:

{\small\begin{verbatim}
  # Do not do this.
  if [ "$CONFIG_EXPERIMENTAL" ]; then
    bool 'Bogus experimental feature' CONFIG_BOGUS
  fi
\end{verbatim}}

\dots then you will be surprised, because \texttt{CONFIG\_EXPERIMENTAL} never 
has a
value of the empty string!  It is always "y" or "n", and both of these
are treated as true (non-empty) by the bash-based interpreters Configure
and Menuconfig.

\begin{tabular}{ll}
Configure:  &implemented\\
Menuconfig: &implemented\\
XConfig:    &implemented, with bugs\\
mconfig:    &implemented\\
\end{tabular}

Xconfig has some known bugs, and probably some unknown bugs too:
\begin{itemize}
\item literals with an empty "" value are not properly handled.
\end{itemize}



\subsubsection{\texttt{mainmenu\_option} \quad\texttt{next\_comment}}

This verb introduces a new menu.  The next statement must have a comment
verb.  The \textit{/prompt/} of that comment verb becomes the title of the menu.
(I have no idea why the original designer didn't create a 'menu \dots' verb).

Statements outside the scope of any menu are in the implicit top menu.
The title of the top menu comes from a variety of sources, depending on
the interpreter.

\begin{tabular}{ll}
Configure:  &implemented\\
Menuconfig: &implemented\\
Xconfig:    &implemented\\
mconfig:    &implemented\\
\end{tabular}



\subsubsection{\texttt{endmenu}}

This verb closes the scope of a menu.

\begin{tabular}{ll}
Configure:  &implemented\\
Menuconfig: &implemented\\
Xconfig:    &implemented\\
mconfig:    &implemented\\
\end{tabular}



\subsubsection{\texttt{source} \quad\textit{/word/}}

This verb interprets the literal \textit{/word/} as a filename, and interpolates
the contents of that file.  The word must be a single unquoted literal
word.

Some interpreters interpret this verb at run time; some interpreters
interpret it at parse time.

Inclusion is textual inclusion, like the C preprocessor \#include facility.
The \texttt{source} verb does not imply a submenu or any kind of block nesting.

\begin{tabular}{ll}
Configure:  &implemented (run time)\\
Menuconfig: &implemented (parse time)\\
Xconfig:    &implemented (parse time)\\
mconfig:    &implemented (parse time)\\
\end{tabular}

