%\documentclass[10pt]{book}
%\documentclass[twocolumn,10pt]{article}
\documentclass[10pt]{article}
\usepackage[a4paper,dvips]{geometry}
\usepackage{fancyvrb}
\usepackage{times}
\usepackage[T1]{fontenc}
\usepackage[latin1]{inputenc}
\usepackage[dvips]{epsfig}
\usepackage{hyperref}

% more tolerance at 2nd word wrap
\tolerance = 1000
% enable third word wrap
\emergencystretch = 10pt
% Disable single lines at the start of a paragraph (Schusterjungen)
\clubpenalty = 10000
% Disable single lines at the end of a paragraph (Hurenkinder)
\widowpenalty = 10000 \displaywidowpenalty = 10000

\newcommand{\epsfigure}[3]
{
  \begin{figure}[ht!]
    \centering
    \begin{minipage}{#2}
      \centerline{\epsfig{file=#1,width=\columnwidth}}
      \caption{#3}
      \label{fig:#1}
    \end{minipage}
  \end{figure}
}

\title{
  {--- L4Env ---} \\
  {An Environment for L4 Applications}}

\author{
  {Operating Systems Research Group} \\
  {Technische Univerit�t Dresden} \\
  {drops@os.inf.tu-dresden.de}}

\begin{document}
\maketitle

\section*{Preface}

This document gives an introduction into L4Env, a programming
environment for application development on top of the L4 microkernel 
family. L4Env was developed as a part of the Dresden Real-Time
Operating System (DROPS) and is still under construction. 

The purpose of this document is to give an overview over the concepts
of L4Env. More information about building and using L4Env can be found
in the {\em Building DROPS HOWTO} and the {\em Building Infrastructure
  for DROPS (BID) Tutorial}. Further on, most components of L4Env
provide a reference manual.

We assume that the reader is familiar with the basic L4 concepts
(tasks, threads, IPC). 

\vfill
\tableofcontents
\clearpage

\section*{Notations}

\begin{tabbing}
xxxxxxxxxxxxx \= \kill
\textsc{Dice} \> Development tools which can be found in l4/tool \\
\tt{thread}   \> L4Env packages which can be found in l4/pkg \\
\end{tabbing}   

%%%%%%%%%%%%%%%%%%%%%%%%%%%%%%%%%%%%%%%%%%%%%%%%%%%%%%%%%%%%%%%%%%%%%%%%%%%
% Motivation
%%%%%%%%%%%%%%%%%%%%%%%%%%%%%%%%%%%%%%%%%%%%%%%%%%%%%%%%%%%%%%%%%%%%%%%%%%%

\section{Motivation}

Prior to L4Env, most L4 applications had their very own idea about
the environment (libraries, interfaces and so on) in which they were
executed. Almost every programmer had his own set of libraries he used
to build his applications, which results in huge problems if someone
tries to combine components developed by different authors. Frequent
problems were conflicting implementations of common functions (like
{\tt printf}) or conflicts caused by the lack of a central management
of resources like threads or virtual memory. 

The intention of L4Env is to define a set of functions which
describe a minimal environment. This minimal environment is available
for every L4 application. Hence, all applications and especially all
libraries can use these functions. Libraries which are intended to be
used by many different applications should only use these functions to
avoid dependencies to other libraries. 

Such an environment also decreases the dependencies to a certain L4
API or hardware architecture, making applications more portable.

%%%%%%%%%%%%%%%%%%%%%%%%%%%%%%%%%%%%%%%%%%%%%%%%%%%%%%%%%%%%%%%%%%%%%%%%%%%
% L4Env Concepts
%%%%%%%%%%%%%%%%%%%%%%%%%%%%%%%%%%%%%%%%%%%%%%%%%%%%%%%%%%%%%%%%%%%%%%%%%%%

\section{L4Env Concepts}

As a microkernel-based system L4Env is built of a set of L4 servers
which manage basic system resources like memory, tasks or I/O
resources. Various L4Env libraries make use of these servers to
give access to the resources and to provide further functionality like
thread management or synchronization primitives. Together with the
L4Env servers these libraries form the programming environment for
application development on L4.

\epsfigure{l4env}{9cm}{Main L4Env components}

Figure~\ref{fig:l4env} shows the main components of L4Env. In addition
to these servers and libraries L4Env contains various development
tools, most notably the \textsc{Dice} IDL compiler. The following
sections will give a short overview of the concepts behind the L4Env
components. 

%%%%%%%%%%%%%%%%%%%%%%%%%%%%%%%%%%%%%%%%%%%%%%%%%%%%%%%%%%%%%%%%%%%%%%%%%%%

\subsection{Memory Management: DMphys, \tt{l4rm}}

L4Env uses the concept of {\em dataspaces}~\cite{aron:dataspaces} to
provide memory to L4 applications. A dataspace is a container which
can contain arbitrary types of memory (e.g. phys. memory, paged
memory, files and so on) and is managed by a {\em dataspace manager}. 
L4Env contains DMphys ({\tt dm\_phys}) as the dataspace manager which
manages the available phys. memory of the system. The address space of
an application is constructed of several dataspaces, these dataspaces
can be managed by different dataspace managers. The {\em region
  mapper} ({\tt l4rm}) manages an address space. The region mapper has
two major tasks, first to allocate virtual memory regions of the
address space and second to invoke the appropriate dataspace manager
to resolve a pagefault on a virtual address.

\subsubsection*{Dataspace Manager}

The interface to dataspace managers is split into several parts:
\begin{description}
\item[{\tt dm\_generic}] provides the interface common for all
  dataspace managers of the system. This interface must be implemented
  by all dataspace managers to ensure a basic functionality regardless
  of the type of memory the dataspace manager handles. It contains for
  instance the function invoked by the region mapper to resolve a
  pagefault or the functions to manage the access rights to dataspaces.
\item[{\tt dm\_mem}] extends {\tt dm\_generic} with functions that
  are useful with dataspaces specifically containing memory.
\item[{\tt dm\_phys}] further extents the dataspace manager interface
  with functions to manage physical memory.
\end{description}

{\tt dm\_generic} provides the interface definition and a client
library to use the interface, it does not provide a dataspace manager
implementation. However, {\tt dm\_generic} contains a library ({\tt
  dsmlib}) which provides various support functions to build a
dataspace manager. {\tt dm\_mem} just provides the interface
definition and a client library. {\tt dm\_phys} provides in addition
to the interface definition and client library the implementation of
the dataspace manager DMphys. 

DMphys manages the available phys. memory of the system. It grabs
all the available memory from Sigma0 and provides this memory to
applications (or other dataspace managers) in terms of
dataspaces. DMphys supports the allocation of memory to different
memory pools, the usage of superpages and special requirements like
contiguous or aligned memory allocation. 

Both DMphys and all the dataspace manager interfaces are described in
the {\em L4Env Physical Memory Dataspace Manager Reference Manual}.

\subsubsection*{Region Mapper}

The region mapper {\tt l4rm} is a library which is linked to all L4Env
applications. It provides the pager for all other threads of the
task. The pager forwards pagefaults caused by any other thread of the
task to the dataspace manager which manages the dataspace which is
associated to the address of the
pagefault. Figure~\ref{fig:dataspaces} shows the relationship between
the region mapper and the dataspace managers.  

\epsfigure{dataspaces}{9cm}{Dataspaces and region mapper}

The region mapper provides functions to associate dataspaces with
areas of the address space of an application, to map dataspace pages
to the address space an to reserve areas of the address space. The
usage of the region mapper is described in the {\em L4 Region Mapper
  Reference Manual}. 


%%%%%%%%%%%%%%%%%%%%%%%%%%%%%%%%%%%%%%%%%%%%%%%%%%%%%%%%%%%%%%%%%%%%%%%%%%%

\subsection{Threads: \tt{thread}}

The L4Env thread library provides a basic abstraction to native L4
threads. The thread library provides
\begin{itemize}
\item functions to create and destroy threads,
\item shutdown callbacks,
\item management of thread local data,
\item priority handling, and
\item stack handling.
\end{itemize}
The intention of the thread library is not to provide a fully featured
thread API like Pthreads, it rather is the basis for native L4
applications and higher level thread abstractions.

The thread library provides a local thread ID which is independent
from the underlying L4 implementation. Stacks are allocated using the
default dataspace  manager provided by L4Env. The stacks are placed in
the address space of a task in a way that the ID of a thread can be
derived from the stack pointer, thus avoiding expensive system calls
({\tt l4\_myself()}). In special cases the stack can be still
specified manually. For more information about the usage of the thread
library see the {\em L4 Thread Library Reference Manual}.

%%%%%%%%%%%%%%%%%%%%%%%%%%%%%%%%%%%%%%%%%%%%%%%%%%%%%%%%%%%%%%%%%%%%%%%%%%%

\subsection{IPC: \textsc{Dice}}
Since writing IPC code from hand over and over is a tedious task, we adopted
a mechanism from distributed computing to hide the communication code: the 
IDL compiler. It translates an abstract interface description of a service
into the communication code to talk to the service. Our IDL compiler is 
\textsc{Dice} and it generates L4 IPC communication code. It provides the
programmer with a C interface to the server and hides all the L4 API specific
code inside the communication stubs.

\textsc{Dice} allows you to include C header files into the IDL file to
use your existing C types in the interface declaration. It generates code
for L4 version 2 and version X.0 (version X.2 support is under way). 
\textsc{Dice} generates assembler inline code for IPC syscalls at the client
side (this will be extended to generate assembler inline code at the
server side as well). With former IDL compilers you had to write the server
loop by hand. This code is generated by \textsc{Dice} as well. To provide
the flexibility of hand-written server loops, you can specify hooks which
are called at the critical points in the server code.

For more information about \textsc{Dice} please consult the \textsc{Dice}
user's manual.

%%%%%%%%%%%%%%%%%%%%%%%%%%%%%%%%%%%%%%%%%%%%%%%%%%%%%%%%%%%%%%%%%%%%%%%%%%%

\subsection{Synchronisation: \tt{semaphore,lock}}

L4Env provides basic, task-local synchronization
primitives. Synchronization primitives for inter-task synchronization 
are not supported, such primitives should be part of the application
API.

\subsubsection*{Semaphore}

The semaphore library provides a simple counting semaphore
implementation. The implementation causes nearly no overhead in case
of no contention on the semaphore. If a thread must be blocked, a
thread provided by the semaphore library is called. This thread
manages the wait queues for all semaphores, avoiding the use of {\tt
  cli/sti} by serializing the accesses to these queues. For details
about the usage see the {\em L4 Semaphore Reference Manual}.

\subsubsection*{Locks}

The lock library provides simple locks using the semaphore
library. For details about the usage see the {\em L4 Lock Reference
  Manual}.
 
%%%%%%%%%%%%%%%%%%%%%%%%%%%%%%%%%%%%%%%%%%%%%%%%%%%%%%%%%%%%%%%%%%%%%%%%%%%

\subsection{Console Input/Output: \tt{con}, \tt{log}}

\subsubsection{\texttt{log}}

\texttt{log} is package providing a basic textual console
\emph{logserver} as well as a set of macros useful for generating
logging output during program execution. The logserver and the macros
can be used independent of each other.

The \emph{logserver} serializes the output of programs to avoid them
to intermix with each other on the output media, i.e. the screen, the
serial console or a network connection. The server offers buffering of
the data to minimze the impact to timing caused by the output. The
logserver comes in two flavours. The simple one uses the L4 kernel
debugger for printing text. A network-flavoured version allows data to
be transfered over the network using a tcp-based protocol with a
telnet at the other end.

The \emph{log macros} allow to easily genearate verbose logging
messages which may contain the position (file and line) in the program
code. The macros are also available in a conditional flavor, with a
condition expression as their first argument. If the
condition expression is constant and false at compile-time, the whole
logging statement will be ignored by the compiler.

\subsubsection{\texttt{con}}
\texttt{con} is a graphical console server. It provides virtual consoles
which are multiplexed to one visible graphics screen. Only the server can
directly access the physical graphics memory. However, the graphics
memory can be mapped to a client's address space as long he owns the
virtual console which is currently in the foreground. This feature was
added due to performance reasons with the XFree86 stub.

Writing to a virtual console can only be performed by its owner. The user
can select one virtual console to be actually displayed. There is a 
small area at the bottom of the screen which cannot be changed by any
client. This area is reserved for displaying the ID of the current foreground
console. The owner of that virtual console receives input events from mouse
and keyboard. A client can open several virtual consoles. Each virtual
console acquires one communication thread of the \texttt{con}
server.

The client uses a simple protocol to communicate with \texttt{con}:
\textit{pSLIM} (``p''~=~pseudo) as an extension of the
\textit{\textbf{S}tateless \textbf{L}ow level \textbf{I}nterface
\textbf{M}achine} protocol -- proposed by SUN at SOSP 1999 \cite{pSLIM}. 
The original protocol supports the following commands:
\begin{description}
\item[\texttt{SET}] Set literal pixels of a rectangular region.
\item[\texttt{BITMAP}] Expand a bitmap to a fill rectangular region with a
  foreground color where the bitmap contains 1's and background color where
  the bitmap contains 0's.
\item[\texttt{FILL}] Fill a rectangular region  with one pixel value.
\item[\texttt{COPY}] Copy a rectangular region  of the framebuffer to 
  another location.
\item[\texttt{CSCS}] Color-space convert rectangular region from YUV to
  RGB with optional bilinear scaling.
\end{description}
Due to several reasons we have extended the protocol by functions for
\begin{itemize}
\item fast font rendering
\item mapping a framebuffer from the client to the server
\item mapping the physical graphics memory from the server to the client
\end{itemize}

Input events (mouse, keyboard) are received using an L4 port of the Linux
input system.

Furthermore, \texttt{con} has limited support for hardware acceleration of
some
graphics cards (fast copy, fast fill, backend scaler for converting
YUV~to~RGB.


%%%%%%%%%%%%%%%%%%%%%%%%%%%%%%%%%%%%%%%%%%%%%%%%%%%%%%%%%%%%%%%%%%%%%%%%%%%

\subsection{Starting Applications: \tt{loader,exec,tftp}}

\subsubsection{Program Loader}
The loader is responsible to load L4 applications at runtime. It is
supported by the executable interpreter which interprets the binary
and resolves unbound symbols (linking). The current implementation of
the interpreter \texttt{exec} is able to interpret ELF binaries \cite{ELF}
as produced by the BID system (\textit{Building infrastructure for DROPS}).

The loader is able to load binaries which depend on shared libraries.
Those binaries have a special dynamic section listing the shared objects
they depend of. Shared libraries should only contain position independent
code (\textit{PIC}) which allows to share the \texttt{text} section
between multiple address spaces and thus allowing to share the same
cache and TLB entries on some architectures.


\subsubsection{File Provider}
The file provider interface defines a straight-forward approach to
access files. A file can be opened and its image is provided as a
dataspace. Currently, we have two implementations:

\begin{itemize}
\item \texttt{tftp} is an L4 server which requests the file from the
  network using the \texttt{tftp} protocol. The network library was ported
  from GRUB. Several current network adapters and an interface to an
  external L4 network server (\texttt{oshkosh}, distributed separately)
  are supported.
\item \texttt{fprov-l4} (example of the loader package) is an L$^{4}$Linux
  program which requests files from the local file system of an
  L$^{4}$Linux machine. It is implemented as a Linux task which is
  aware of the L4 environment interface behind Linux.
\end{itemize}

Both implementations read the whole image of the file at once and copy
it into a new dataspace they allocated at a dataspace manager. Other
implementations could read parts of the file on demand.

%%%%%%%%%%%%%%%%%%%%%%%%%%%%%%%%%%%%%%%%%%%%%%%%%%%%%%%%%%%%%%%%%%%%%%%%%%%

\subsection{Tasks: \texttt{simple\_ts}}
The task server \texttt{simple\_ts} is responsible for creating and
destroying L4 tasks. Tasks are created in three steps: Firstly, the
client asks the task server to allocate a free task ID. Secondly, the client
registers this ID at the pager of the new task. Finally, the client
asks the task server to create the task. This protocol ensures that the
task doesn't start until its pager knows about it.

To be compatible with L$^{4}$Linux the task server allows to
transmit tasks creation rights to a client.


%%%%%%%%%%%%%%%%%%%%%%%%%%%%%%%%%%%%%%%%%%%%%%%%%%%%%%%%%%%%%%%%%%%%%%%%%%%

\subsection{I/O Resources: \tt{l4io}}

\texttt{l4io} manages I/O resources and provides an interface to access I/O
memory regions, I/O ports, ISA DMA channels and PCI configuration space. All
hardware specific PCI code is ported from Linux 2.4.  Furthermore \texttt{l4io}
enables the implementation of user-level interrupt handling based on
\emph{Omega0} described in \cite{loeser99:omega0} and supports interrupts
sharing.


%%%%%%%%%%%%%%%%%%%%%%%%%%%%%%%%%%%%%%%%%%%%%%%%%%%%%%%%%%%%%%%%%%%%%%%%%%%

\subsection{Miscellaneous}

\subsubsection{C-Library}

Currently, we use the OSKit C-library for the L4 environment. There are
several backend libraries (\texttt{oskit*\_support*}) to map low-level
functionality to L4 primitives.

We plan to move to another or own implementation in the future due to
several reasons (see open issues).


\subsubsection{Name Service: \tt{names}}
 
A simple root name service is part of the L4 environment. \texttt{names}
provides a mapping of names (strings) to L4 \texttt{thread\_id}s and vice
versa. L4 servers register a unique string plus the thread ID of their
worker thread. The belonging thread ID to a name can be requested by other
servers to find the services they depend on and to synchronize at startup.



\subsubsection{Utilities}

The {\tt l4util} library is a collection of useful functions not found
elsewhere. It includes
\begin{itemize}
\item atomic operations ({\tt cmpxchg}, \dots),
\item bit operations,
\item low-level hardware access (port I/O, PIC/APIC programming,
  performance counter),
\item command line parsing,
\item base64 encoding, and
\item pseudo-random number generator.
\end{itemize}

%%%%%%%%%%%%%%%%%%%%%%%%%%%%%%%%%%%%%%%%%%%%%%%%%%%%%%%%%%%%%%%%%%%%%%%%%%%
% Open Issues
%%%%%%%%%%%%%%%%%%%%%%%%%%%%%%%%%%%%%%%%%%%%%%%%%%%%%%%%%%%%%%%%%%%%%%%%%%%

\section{Open Issues}

\begin{itemize}
\item Most servers are single threaded (vulnerable by 
  \textit{denial-of-service-attacks}).
\item Semaphore library and thread priorities. 
  \begin{itemize}
    \item It is not ensured that we wakeup the thread with the highest
      priority. Using the \textit{not-switch-if-deceit} mechanism of
      Fiasco does not fully solve that problem.
  \item Lack of task-overlapped synchronization
  \end{itemize}
\item The loader/linker mechanism is quite complex.
  \begin{itemize}
    \item We use an executable interpreter to interpret and link the
      binaries. Another approach would be a linker library which links
      the binary in its address space. The design would be a little bit
      more straight forward. It also would be possible to do
      \textit{lazy linking}, that is resolve symbols at the first time
      they are accessed. This is probably not important for real-time
      applications.
    \item The loader is still the pager for some program sections
      (\texttt{libloader.s.so} for \textit{Loader-style} applications) or
      all program sections (for \textit{Sigma0-style} applications). It would
      be better to use an own server which is only responsible for this job.
    \item \textit{L4Env-infopage} mechanism for applications not fully
      used.
  \end{itemize}
\item Still no working solution for task termination: 
  \begin{itemize} 
    \item A task can only be killed explicit. A possible solution is to use
      a notification server (\textit{work-in-progress}). The \texttt{exit()}
      function of the task could send an appropriate event to someone.
    \item When a task should be killed, the loader asks all participating
      L4 servers (\texttt{names}, \texttt{rmgr}, \texttt{con}, 
      \texttt{exec}, \texttt{simple\_ts}) to kill all resources
      of that task. This could also be solved by an event server.
  \end{itemize}
\item No naming hierarchy in the name server.
\item No resource reservation. For example, each client can get as much
  memory from dm\_phys as we have physical memory.
\item Some open DICE issues (optimization, error handling)
  \begin{itemize}
  \item more optimization necessary
  \item better error handling at client and server side
  \end{itemize}
\item We don't have a copy-on-write dataspace manager. A dataspace manager
  which could hand out cache-coloured pages would also be useful.
\item We don't have a file provider for hard disks.
\item We still use the heavy-weight OSKit as lib-c.
  \begin{itemize}
  \item code bloat
  \item no support for different output channels \\
    (\texttt{fprintf(stderr, ...)})
  \item No support for ``command line tools''. Such a tool would not
    open an own graphical console but it would use the console of the
    starter application for input and output.
  \item several hacks (e.g. \texttt{log} library implements own
    \texttt{sprintf()})
  \end{itemize}
\item We currently support L4 \textit{version 2} and
  \textit{version X.0}. Support of L4 \textit{version
    X.2} alias \textit{version 4} may require some internal
  restructuring and maybe even API changes (e.g. threads can be
  only created by privileged threads).
\item We currently only support one hardware platform (x86) and only
  single-processor systems.
\item We still depend on the \texttt{rmgr}.
\item No support for resource withdrawing (e.g. IRQs).
\item Support for C++ is not fully tested.
\item \texttt{l4io} doesn't know about other buses than PCI and there
  is no bus hierarchy.
\end{itemize}

%%%%%%%%%%%%%%%%%%%%%%%%%%%%%%%%%%%%%%%%%%%%%%%%%%%%%%%%%%%%%%%%%%%%%%%%%%%
% References
%%%%%%%%%%%%%%%%%%%%%%%%%%%%%%%%%%%%%%%%%%%%%%%%%%%%%%%%%%%%%%%%%%%%%%%%%%%

\bibliographystyle{plain}
\bibliography{own}

\end{document}

% LocalWords:  Env microkernel dataspace DMphys pagefault superpages
